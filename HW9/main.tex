\def\duedate{\today}
\def\HWnum{9}
\documentclass[10pt,a4paper]{book}

% custom section formatting
\usepackage{titlesec}
\titleformat{\chapter}[display]
{\normalfont\Large\filcenter\sffamily}
{\titlerule[1pt]%
\vspace{1pt}%
\titlerule
\vspace{1pc}%
\LARGE\MakeUppercase{\chaptertitlename} \thechapter}
{1pc}
{\titlerule
\vspace{1pc}%
\Huge}

% appendix handling
\usepackage[toc,page]{appendix}
    
% encoding for file and font
\usepackage[utf8]{inputenc}
\usepackage[T1]{fontenc}

% math formatting/tools
\usepackage{amsmath}
\usepackage{amssymb}
\usepackage{mathtools}
\usepackage[arrowdel]{physics}

% unit formatting
\usepackage{siunitx}
\AtBeginDocument{\RenewCommandCopy\qty\SI}

% figure formatting/tools
\usepackage{graphicx}
\usepackage{float}
\usepackage{subcaption}
\usepackage{multirow}
\usepackage{import}
\usepackage{pdfpages}
\usepackage{transparent}
\usepackage{currfile}

\NewDocumentCommand\incfig{O{1} m}{
    \def\svgwidth{#1\textwidth}
    \import{./Figures/\currfiledir}{#2.pdf_tex}
}

\newcommand{\bef}{\begin{figure}[h!tb]\centering}
\newcommand{\eef}{\end{figure}}

\newcommand{\bet}{\begin{table}[h!tb]\centering}
\newcommand{\eet}{\end{table}}

% hyperlink references 
\usepackage{hyperref}
\hypersetup{
    colorlinks=true,
    linkcolor=blue,
    filecolor=magenta,
    urlcolor=cyan,
    pdftitle={Physics 1 Notes},
    pdfauthor={Richard Whitehill},
    pdfpagemode=FullScreen
}
\urlstyle{same}

\newcommand{\eref}[1]{Eq.~(\ref{eq:#1})}
\newcommand{\erefs}[2]{Eqs.~(\ref{eq:#1})--(\ref{eq:#2})}

\newcommand{\fref}[1]{Fig.~(\ref{fig:#1})}
\newcommand{\frefs}[2]{Fig.~(\ref{fig:#1})--(\ref{fig:#2})}

\newcommand{\aref}[1]{Appendix~(\ref{app:#1})}
\newcommand{\sref}[1]{Section~(\ref{sec:#1})}
\newcommand{\srefs}[2]{Sections~(\ref{sec:#1})-(\ref{sec:#2})}

\newcommand{\tref}[1]{Table~(\ref{tab:#1})}
\newcommand{\trefs}[2]{Table~(\ref{tab:#1})--(\ref{tab:#2})}

% tcolorbox formatting/definitions
\usepackage[most]{tcolorbox}
\usepackage{xcolor}
\usepackage{xifthen}
\usepackage{parskip}

\definecolor{peach}{rgb}{1.0,0.8,0.64}

\DeclareTColorBox[auto counter, number within=chapter]{defbox}{O{}}{
    enhanced,
    boxrule=0pt,
    frame hidden,
    borderline west={4pt}{0pt}{green!50!black},
    colback=green!5,
    before upper=\textbf{Definition \thetcbcounter \ifthenelse{\isempty{#1}}{}{: #1} \\ },
    sharp corners
}

\newcommand*{\eqbox}{\tcboxmath[
    enhanced,
    colback=black!10!white,
    colframe=black,
    sharp corners,
    size=fbox,
    boxsep=8pt,
    boxrule=1pt
]}

\newtcolorbox[auto counter, number within=chapter]{exbox}{
    parbox=false,
    breakable,
    enhanced,
    sharp corners,
    boxrule=1pt,
    colback=white,
    colframe=black,
    before upper= \textbf{Example \thetcbcounter:}\,,
    before lower= \textbf{Solution:}\,,
    segmentation hidden
}

\newtcolorbox{resbox}{
    enhanced,
    colback=black!10!white,
    colframe=black,
    boxrule=1pt,
    boxsep=0pt,
    top=2pt,
    ams nodisplayskip,
    sharp corners
}


\begin{document}

\prob{1 -- Chapter 7 \# 12}{

This problem deals with the phenomenon of neutrino oscillations in a simplified scenario, in which the neutrino flavors are two (electron and muon neutrinos) rather than three (electron, muon, and tau neutrinos) as observed in Nature.
We denote these two flavors with $\nu_{e}$ and $\nu_{\mu}$, respectively.
It turns out that an electron neutrino of momentum $\va*{p}$ is given by
\begin{eqnarray}
    \ket{\nu_{e};\va*{p}} = \cos{\theta} \ket{\nu_1;\va*{p}} + \sin{\theta} \ket{\nu_2;\va*{p}}
,\end{eqnarray}
where $\ket{\nu_{k};\va*{p}}$ with $k = 1,2$ is an eigenstate of the relativistic free-particle Hamiltonian
\begin{eqnarray}
    \hat{H}_{k} = c \sqrt{\hat{\va*{p}}^2 + (m_{k} c)^2}
.\end{eqnarray}
That is
\begin{eqnarray}
    \hat{H}_{k} \ket{\nu_{k};\va*{p}} = E_{k} \ket{\nu_{k};\va*{p}}, \quad E_{k} = c \sqrt{p^2 + (m_{k}c)^2}
,\end{eqnarray}
where $m_{k}$ and $c$ are, respectively, the mass and speed of light, and 
\begin{eqnarray}
    \bra{\nu_{k};\va*{p}}\ket{\nu_{l};\va*{q}} = \delta_{kl} \delta(\va*{p} - \va*{q})
.\end{eqnarray}
A muon neutrino of momentum $\va*{p}$ is given, in terms of the Hamiltonian eigenstates defined above, by the orthogonal linear combination
\begin{eqnarray}
    \ket{\nu_{\mu};\va*{p}} = -\sin{\theta} \ket{\nu_1;\va*{p}} + \cos{\theta} \ket{\nu_2;\va*{p}}
.\end{eqnarray}

(a) Assume that at time $t = 0$ the neutrino state is $\ket{\psi(0)} = \ket{\nu_{e};\va*{p}}$.
Calculate the probability $P_{\nu_{e} \rightarrow \nu_{\mu}}(t)$ that at time $t$ the neutrino is in state $\ket{\psi(t)} = \ket{\nu_{\mu};\va*{p}}$, that is the neutrino has undergone the flavor oscillation $\nu_{e} \rightarrow \nu_{\mu}$.
Calculatealso the survival probability $P_{\nu_e \rightarrow \nu_{\mu}}$.
Alternatively, assume that the neutrino state at time $t = 0$ is $\ket{\psi(0)} = \ket{\nu_{\mu};\va*{p}}$, and calculate $P_{\nu_{\mu} \rightarrow \nu_{e}}(t)$ and $P_{\nu_{\mu} \rightarrow \nu_{e}}(t)$.

(b) Assume that the momentum $\va*{p}$ (that is, the eigenvalue of the momentum operator) is such that $|\va*{p}| \gg m_{k}c$.
Show that to leading order in $\Delta m^2 = m_1^2 - m_2^2$ the probability for the conversion $\nu_{e} \rightarrow \nu_{\mu}$ can be written as
\begin{eqnarray}
    P_{\nu_{e} \rightarrow \nu_{\mu}}(L) = \sin^2(2 \theta) \sin^2\Big( \frac{\Delta m^2 L c^2}{4 \hbar p} \Big)
,\end{eqnarray}
where $L$ is the distance the neutrino has travelled in the time $t$.

}

\sol{}


\prob{2 -- Chapter 8 \# 3}{

A harmonic oscillator of mass $m$ and angular frequency $\omega$ is in the ground state with normalized wave function given by
\begin{eqnarray}
    \psi_0(x) = \Big( \frac{m \omega}{\pi \hbar} \Big)^{1/4} \exp( -\frac{m \omega}{2 \hbar} x^2 )
.\end{eqnarray}

(a) At $t = 0$ a measurement of the particle's position is made.
What is the probability that the particle be found in an interval $\dd{x}$ centered at $x = 0$?

(b) Assume that the measurement of the particle's position at $t = 0$ indeed yields the result $x = 0$.
What is the particle's wave function $\psi(x,0)$ immediately after this measurement?

(c) What is the particle's wave function $\psi(x,t)$ at a later time $t$?

(d) What is the probability that a measurement of the energy at time $t$ yields the result $\hbar \omega / 2$?

(e) Without doing any detailed calculation, explain why there is zero probability to measure an energy $( n + 1/2 )\hbar \omega$ with $n$ odd.

}

\sol{}


\prob{3 -- Chapter 8 \# 5}{

The Hamiltonian of two particles is given by
\begin{eqnarray}
    \hat{H} = \hat{H}_1 + \hat{H}_2, \quad \hat{H}_{i} = \frac{\hat{p}_{i}^2}{2m} + \frac{m \omega^2}{2} \hat{x}_{i}^2
,\end{eqnarray}
where $\hat{x}_{i}$ and $\hat{p}_{i}$ are the position and momentum operator of particle $i$.
At time $t = 0$ the system is in state
\begin{eqnarray}
    \ket{\psi(0)} = \frac{1}{2} ( \ket{0,0} + \ket{1,0} + \ket{0,1} + \ket{1,1} )
.\end{eqnarray}

(a) At time $t = 0$ the total energy is measured and the result $2 \hbar \omega$ is obtained: (i) Calculate the mean values of the position, momentum, and energy of particle 1 and (ii) At $t > 0$ the energy of particle 1 is measured. What results can be found and with what probabilities?
Same question for a measurement of the position of particle 1.

(b) Instead of measuring the total energy at $t = 0$, the energy of particle 2 is measured and the result $\hbar \omega / 2$ is obtained.
What happens to the answers to questions (a) and (b) above?


}

\sol{}


\prob{4 -- Chapter 8 \# 6}{

The Hamiltonian of two particles is given by
\begin{eqnarray}
    \hat{H} = \hat{H}_1 + \hat{H}_2, \quad \hat{H}_{i} = \frac{\hat{p}_{i}^2}{2m} + \frac{m \omega^2}{2} \hat{x}_{i}^2
,\end{eqnarray}
where $\hat{x}_{i}$ and $\hat{p}_{i}$ are the position and momentum operators of particle $i$.
Let $\ket{n_1,n_2}$ denote the common eigenstates of $\hat{H}_1$ and $\hat{H}_2$ with eigenvalues $\hbar \omega (n_1 + 1/2)$ and $\hbar \omega (n_2 + 1/2)$, respectively.
We define the two-particle exchange operator as
\begin{eqnarray}
    \hat{P}_{e}\ket{n_1,n_2} = \ket{n_2,n_1}
.\end{eqnarray}

(a) Show that $\hat{P}_{e}^{-1} = \hat{P}_{e}$ and that $\hat{P}_{e}$ is unitary.
What are the eigenvalues of $\hat{P}_{e}$?
Let $\hat{A}' = \hat{P}_{e} \hat{A} \hat{P}_{e}^{\dagger}$ be the observable resulting from the transformation by $\hat{P}_{e}$ of an arbitrary observable $\hat{A}$.
Show that the condition $\hat{A}' = \hat{A}$ is equivalent to $[ \hat{A}, \hat{P}_{e} ] = 0$.

(b) Show that
\begin{eqnarray}
    \hat{P}_{e} \hat{H}_{1} \hat{P}_{e}^{\dagger} = \hat{H}_{2}, \quad \hat{P}_{e} \hat{H}_{2} \hat{P}_{e}^{\dagger} = \hat{H}_{1} 
.\end{eqnarray}
Does $\hat{H}$ commute with $\hat{P}_{e}$?
Calculate the action of $\hat{P}_{e}$ on the observables $\hat{x}_{1,2}$ and $\hat{p}_{1,2}$.

(c) Construct a basis of common eigenstates of $\hat{H}$ and $\hat{P}_{e}$.
Do these two operators form a complete set of commuting observables?
What happens to the spectrum of $\hat{H}$ and the degeneracy of its eigenvalues if only those eigenvectors of $\hat{H}$ are retained such that they are odd under exchange?

}

\sol{}




\end{document}
