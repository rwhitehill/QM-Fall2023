\def\duedate{\today}
\def\HWnum{9}
% Document setup
\documentclass[12pt]{article}
\usepackage[margin=1in]{geometry}
\usepackage{fancyhdr}
\usepackage{lastpage}

\pagestyle{fancy}
\lhead{Richard Whitehill}
\chead{PHYS 621 -- HW \HWnum}
\rhead{\duedate}
\cfoot{\thepage \hspace{1pt} of \pageref{LastPage}}

% Encoding
\usepackage[utf8]{inputenc}
\usepackage[T1]{fontenc}

% Math/Physics Packages
\usepackage{amsmath}
\usepackage{amssymb}
\usepackage{mathtools}
\usepackage[arrowdel]{physics}
\usepackage{siunitx}

\AtBeginDocument{\RenewCommandCopy\qty\SI}

% Reference Style
\usepackage{hyperref}
\hypersetup{
    colorlinks=true,
    linkcolor=blue,
    filecolor=magenta,
    urlcolor=cyan,
    citecolor=green
}

\newcommand{\eref}[1]{Eq.~(\ref{eq:#1})}
\newcommand{\erefs}[2]{Eqs.~(\ref{eq:#1})--(\ref{eq:#2})}

\newcommand{\fref}[1]{Fig.~\ref{fig:#1}}
\newcommand{\frefs}[2]{Figs.~\ref{fig:#1}--\ref{fig:#2}}

\newcommand{\tref}[1]{Table~\ref{tab:#1}}
\newcommand{\trefs}[2]{Tables~\ref{tab:#1}-\ref{tab:#2}}

% Figures and Tables 
\usepackage{graphicx}
\usepackage{float}

\newcommand{\bef}{\begin{figure}[h!]\begin{center}}
\newcommand{\eef}{\end{center}\end{figure}}

\newcommand{\bet}{\begin{table}[h!]\begin{center}}
\newcommand{\eet}{\end{center}\end{table}}

% tikz
\usepackage{tikz}
\usetikzlibrary{calc}
\usetikzlibrary{decorations.pathmorphing}
\usetikzlibrary{decorations.markings}
\usetikzlibrary{arrows.meta}
\usetikzlibrary{positioning}

% tcolorbox
\usepackage[most]{tcolorbox}
\usepackage{xcolor}
\usepackage{xifthen}
\usepackage{parskip}

\newcommand*{\eqbox}{\tcboxmath[
    enhanced,
    colback=black!10!white,
    colframe=black,
    sharp corners,
    size=fbox,
    boxsep=8pt,
    boxrule=1pt
]}

% problem-solution macros
% \usepackage{adjustbox}
\usepackage{changepage}

\newtcolorbox{probbox}[1][]{
    breakable,
    enhanced,
    boxrule=0pt,
    frame hidden,
    borderline west={4pt}{0pt}{green!50!black},
    colback=green!5,
    before upper=\textbf{Problem #1) \,},
    % \textbf{Problem #1 \ifthenelse{\isempty{#1}}{}{: #1} \\ },
    sharp corners
}

% \newtcolorbox{ProblemBox}[1][]{%
%   breakable,
%   enhanced,
%   colback=black!10!white,
%   colframe=black,
%   title={\large #1 \hfill}
% }
\newcommand{\prob}[2]{
\begin{probbox}[#1]
#2
\end{probbox}
}

\newenvironment{solution}{\begin{adjustwidth}{8pt}{8pt}}{\end{adjustwidth}}
\newcommand{\sol}[1]{
\begin{solution}
#1
\end{solution}
}
% \textbf{#1)} #2}

% Miscellaneous Definitions/Settings
\newcommand{\reals}{\mathbb{R}}
\newcommand{\integers}{\mathbb{Z}}
\newcommand{\naturals}{\mathbb{N}}
\newcommand{\rationals}{\mathbb{Q}}
\newcommand{\complexs}{\mathbb{C}}

\setlength{\parskip}{\baselineskip}
\setlength{\parindent}{0pt}
\setlength{\headheight}{14.49998pt}
\addtolength{\topmargin}{-2.49998pt}


\begin{document}

\prob{1 -- Chapter 7 \# 12}{

This problem deals with the phenomenon of neutrino oscillations in a simplified scenario, in which the neutrino flavors are two (electron and muon neutrinos) rather than three (electron, muon, and tau neutrinos) as observed in Nature.
We denote these two flavors with $\nu_{e}$ and $\nu_{\mu}$, respectively.
It turns out that an electron neutrino of momentum $\va*{p}$ is given by
\begin{eqnarray}
    \ket{\nu_{e};\va*{p}} = \cos{\theta} \ket{\nu_1;\va*{p}} + \sin{\theta} \ket{\nu_2;\va*{p}}
,\end{eqnarray}
where $\ket{\nu_{k};\va*{p}}$ with $k = 1,2$ is an eigenstate of the relativistic free-particle Hamiltonian
\begin{eqnarray}
    \hat{H}_{k} = c \sqrt{\hat{\va*{p}}^2 + (m_{k} c)^2}
.\end{eqnarray}
That is
\begin{eqnarray}
    \hat{H}_{k} \ket{\nu_{k};\va*{p}} = E_{k} \ket{\nu_{k};\va*{p}}, \quad E_{k} = c \sqrt{p^2 + (m_{k}c)^2}
,\end{eqnarray}
where $m_{k}$ and $c$ are, respectively, the mass and speed of light, and 
\begin{eqnarray}
    \bra{\nu_{k};\va*{p}}\ket{\nu_{l};\va*{q}} = \delta_{kl} \delta(\va*{p} - \va*{q})
.\end{eqnarray}
A muon neutrino of momentum $\va*{p}$ is given, in terms of the Hamiltonian eigenstates defined above, by the orthogonal linear combination
\begin{eqnarray}
    \ket{\nu_{\mu};\va*{p}} = -\sin{\theta} \ket{\nu_1;\va*{p}} + \cos{\theta} \ket{\nu_2;\va*{p}}
.\end{eqnarray}

(a) Assume that at time $t = 0$ the neutrino state is $\ket{\psi(0)} = \ket{\nu_{e};\va*{p}}$.
Calculate the probability $P_{\nu_{e} \rightarrow \nu_{\mu}}(t)$ that at time $t$ the neutrino is in state $\ket{\psi(t)} = \ket{\nu_{\mu};\va*{p}}$, that is the neutrino has undergone the flavor oscillation $\nu_{e} \rightarrow \nu_{\mu}$.
Calculate also the survival probability $P_{\nu_e \rightarrow \nu_{\mu}}$.
Alternatively, assume that the neutrino state at time $t = 0$ is $\ket{\psi(0)} = \ket{\nu_{\mu};\va*{p}}$, and calculate $P_{\nu_{\mu} \rightarrow \nu_{e}}(t)$ and $P_{\nu_{\mu} \rightarrow \nu_{e}}(t)$.

(b) Assume that the momentum $\va*{p}$ (that is, the eigenvalue of the momentum operator) is such that $|\va*{p}| \gg m_{k}c$.
Show that to leading order in $\Delta m^2 = m_1^2 - m_2^2$ the probability for the conversion $\nu_{e} \rightarrow \nu_{\mu}$ can be written as
\begin{eqnarray}
    P_{\nu_{e} \rightarrow \nu_{\mu}}(L) = \sin^2(2 \theta) \sin^2\Big( \frac{\Delta m^2 L c^2}{4 \hbar p} \Big)
,\end{eqnarray}
where $L$ is the distance the neutrino has travelled in the time $t$.

}

\sol{

(a) If the neutrino state is in the electron neutrino eigenstate at the initial time, then in terms of the mass eigenstates (which are also energy eigenstates)
\begin{eqnarray}
    \ket{\psi(0)} = \cos{\theta} \ket{\nu_1;\va*{p}} + \sin{\theta} \ket{\nu_2;\va*{p}}
,\end{eqnarray}
and at an arbitary time $t$
\begin{eqnarray}
    \ket{\psi(t)} = \cos{\theta} e^{-i E_1 t / \hbar} \ket{\nu_1;\va*{p}} + \sin{\theta} e^{-i E_2 t / \hbar} \ket{\nu_2;\va*{p}}
,\end{eqnarray}
where $E_{1,2}$ are as given above.
We can then calculate the probability that the neutrino has undergone a flavor oscillation (i.e. that a flavor measurement of the neutrino gives $\nu_{\mu}$) as
\begin{eqnarray}
\begin{aligned}
    P_{e \rightarrow \mu}(t) &= |\bra{\nu_{\mu};\va*{p}}\ket{\psi(t)}|^2 = \Big| ( -\sin{\theta} \bra{\nu_1;\va*{p}} + \cos{\theta} \bra{\nu_2;\va*{p}} ) \ket{\psi(t)} \Big|^2 \\
                             &= | -\sin{\theta}\cos{\theta} e^{-i E_1 t/\hbar} + \cos{\theta} \sin{\theta} e^{-i E_2 t / \hbar} |^2 = \sin^2{\theta} \cos^2{\theta} | 1 - e^{i(E_1 - E_2) t / \hbar} |^2 \\
                             &= \sin^2{2 \theta} \sin^2\Bigg(\frac{[E_1 - E_2]t}{2 \hbar}\Bigg)
\end{aligned}
.\end{eqnarray}
From this, we can calulate the probability that the neutrino is measured to be of electronic flavor as
\begin{eqnarray}
\begin{aligned}
    P_{e \rightarrow e}(t) &= 1 - \sin^2{2\theta} \sin^2\Bigg( \frac{[E_1 - E_2]t}{2\hbar} \Bigg)
\end{aligned}
.\end{eqnarray}

Let us now assume that the neutrino state is initially in the muonic flavor eigenstate.
We can calculate the oscillation amplitude
\begin{eqnarray}
    P_{\mu \rightarrow e}(t) = P_{e \rightarrow \mu}(t)
\end{eqnarray}
by observing that the mass and flavor eigenstates are given by a rotation of bases through some mixing angle $\theta$.
It is clear then that we can take $\theta \rightarrow -\theta$ and permute flavors $e \rightarrow \mu$, ending up with the same description.
Thus, permuting flavor indices in the oscillation probability and taking $\theta \rightarrow -\theta$, we end up with the same result since $\sin^2{2\theta}$ is even in $\theta$.
It also follows from the argument that
\begin{eqnarray}
    P_{\mu \rightarrow \mu}(t) = P_{e \rightarrow e}
.\end{eqnarray}

(b) We take the first result demonstrated above.
If $|\va*{p}| \gg m_{k}c$, then we can write
\begin{eqnarray}
    E_{k} \approx c p \Big[ 1 + \frac{1}{2} \frac{(m_{k} c)^2}{p^2} \Big]
,\end{eqnarray}
and therefore the energy difference
\begin{eqnarray}
    E_1 - E_2 \approx \frac{\Delta m^2 c^3}{2p}
.\end{eqnarray}
To first order in $\Delta m^2$, the oscillation probability from electronic to muonic flavor for a neutrino with momentum $\va*{p}$ traveling a distance $L$ over time $t$ ($L \approx c t$ since the particle is ultra-relativistic)
\begin{eqnarray}
    P_{e \rightarrow \mu} = \sin^2{2 \theta} \sin^2\Bigg( \frac{\Delta m^2 L c^2}{4 \hbar p} \Bigg)
.\end{eqnarray}

}


\prob{2 -- Chapter 8 \# 3}{

A harmonic oscillator of mass $m$ and angular frequency $\omega$ is in the ground state with normalized wave function given by
\begin{eqnarray}
    \psi_0(x) = \Big( \frac{m \omega}{\pi \hbar} \Big)^{1/4} \exp( -\frac{m \omega}{2 \hbar} x^2 )
.\end{eqnarray}

(a) At $t = 0$ a measurement of the particle's position is made.
What is the probability that the particle be found in an interval $\dd{x}$ centered at $x = 0$?

(b) Assume that the measurement of the particle's position at $t = 0$ indeed yields the result $x = 0$.
What is the particle's wave function $\psi(x,0)$ immediately after this measurement?

(c) What is the particle's wave function $\psi(x,t)$ at a later time $t$?

(d) What is the probability that a measurement of the energy at time $t$ yields the result $\hbar \omega / 2$?

(e) Without doing any detailed calculation, explain why there is zero probability to measure an energy $( n + 1/2 )\hbar \omega$ with $n$ odd.

}

\sol{

(a) The probability density is given as
\begin{eqnarray}
    \rho(x) = |\psi_0(x)|^2 = \sqrt{ \frac{m \omega}{\pi \hbar} } \exp( -\frac{m\omega}{\hbar} x^2 ) \Rightarrow \rho(0) = \sqrt{\frac{m \omega}{\pi \hbar}}
.\end{eqnarray}

(b) Right after measurement of the position, which yields $x = 0$, the system collapses to the state
\begin{eqnarray}
    \psi(x,0) = \bra{\phi_{x}} \frac{\ket{\phi_{0}} \psi_0(x)}{\sqrt{|\bra{\phi_{0}}\ket{n=0}|^2}} = \frac{\psi_0(x)}{\psi_0(0)} \delta(x) = \delta(x)
,\end{eqnarray}
which is as expected sine the system collapses to the normalized eigenstate corresponding to the position $x = 0$.

(c) Before writing the full time-dependent wave-function, we must determine the energy decomposition
\begin{eqnarray}
    \psi(x,0) = \sum_{n=0}^{\infty} c_{n} \psi_{n}(x)
,\end{eqnarray}
where
\begin{eqnarray}
    c_{n} = \int_{-\infty}^{\infty} \psi_{n}^{*}(x) \delta(x) \dd{x} = \psi_{n}^{*}(0)
.\end{eqnarray}
Note that $c_{2n+1} = 0$ since $\psi_{2n+1}(0) = 0$, leaving only the even energy eigenstates:
\begin{eqnarray}
    \psi(x,0) = \sum_{n=0}^{\infty} \psi_{2n}^{*}(0) \psi_{2n}(x)
.\end{eqnarray}
The wave-function at arbitrary time $t$ is then just
\begin{eqnarray}
    \psi(x,t) = \sum_{n=0}^{\infty} \psi_{2n}^{*}(0) \psi_{2n}(x) e^{-i (2n + 1) \omega t / 2}
.\end{eqnarray}

(d) The probabilty that an energy measurement on this state gives the $E_{0} = \hbar \omega / 2$ at time $t$ is simply
\begin{eqnarray}
P_{0}(t) = | \psi_{0}(0) |^2 = \sqrt{\frac{m \omega}{\pi \hbar}}
.\end{eqnarray}

(e) This fact follows simply from the fact that $\psi(x,0)$ is even and therefore orthognal to odd energy eigenstates as explained in part (c).

}


\prob{3 -- Chapter 8 \# 5}{

The Hamiltonian of two particles is given by
\begin{eqnarray}
    \hat{H} = \hat{H}_1 + \hat{H}_2, \quad \hat{H}_{i} = \frac{\hat{p}_{i}^2}{2m} + \frac{m \omega^2}{2} \hat{x}_{i}^2
,\end{eqnarray}
where $\hat{x}_{i}$ and $\hat{p}_{i}$ are the position and momentum operator of particle $i$.
At time $t = 0$ the system is in state
\begin{eqnarray}
    \ket{\psi(0)} = \frac{1}{2} ( \ket{0,0} + \ket{1,0} + \ket{0,1} + \ket{1,1} )
.\end{eqnarray}

(a) At time $t = 0$ the total energy is measured and the result $2 \hbar \omega$ is obtained: (i) Calculate the mean values of the position, momentum, and energy of particle 1 and (ii) At $t > 0$ the energy of particle 1 is measured. What results can be found and with what probabilities?
Same question for a measurement of the position of particle 1.

(b) Instead of measuring the total energy at $t = 0$, the energy of particle 2 is measured and the result $\hbar \omega / 2$ is obtained.
What happens to the answers to questions (a) and (b) above?

}

\sol{

(a) It is useful to write the position, momentum, and Hamiltonian operators in terms of the creation and annihilation operators
\begin{align}
    \hat{x}_{k} &= \sqrt{\frac{\hbar}{2m\omega}} (\hat{a}^{\dagger}_{k} + \hat{a}_{k}) \\
    \hat{p}_{k} &= i \sqrt{\frac{\hbar m \omega}{2}} (\hat{a}^{\dagger}_{k} - \hat{a}_{k}) \\
    \hat{H}_{k} &= \hbar \omega \Big( \hat{a}^{\dagger}_{k} \hat{a}_{k} + \frac{1}{2} \Big)
.\end{align}
Recall the following properties:
\begin{align}
    \hat{a}^{\dagger} \ket{n} &= \sqrt{n+1} \ket{n+1} \\
    \hat{a} \ket{n} &= \sqrt{n} \ket{n-1} \\
    \hat{a}^{\dagger} a \ket{n} &= n \ket{n}
.\end{align}
(i) It follows then that the average position, momentum, and energy of particle 1 in the composite state after the measurement are
\begin{gather}
\begin{aligned}
    \expval{x_1} &= \frac{1}{2} \sqrt{\frac{\hbar}{2 m \omega}} ( \bra{1,0} + \bra{0,1} ) (a^{\dagger}_{1} + a ) ( \ket{1,0} + \ket{0,1} ) \\
               &= \frac{1}{2} \sqrt{\frac{\hbar}{2 m \omega}} ( \bra{1,0} + \bra{0,1} ) ( \sqrt{2} \ket{2,0} + \ket{1,1} + \ket{0,0} ) = 0
\end{aligned}
\\
\begin{aligned}
    \expval{p_1} &= \frac{i}{4} \sqrt{\frac{\hbar m \omega}{2}} ( \bra{1,0} + \bra{0,1} ) (a^{\dagger}_{1} - a_{1}) ( \ket{1,0} + \ket{0,1} ) \\
               &= \frac{i}{4} \sqrt{\frac{ \hbar m \omega}{2}} ( \bra{1,0} + \bra{0,1} ) ( \sqrt{2}\ket{2,0} + \ket{1,1} - \ket{0,0} ) = 0 \\
\end{aligned}
\\
\begin{aligned}
    \expval{H_1} = \frac{\hbar \omega}{2} [  ( 1 + 1/2 ) + ( 0 + 1/2 ) ] = \hbar \omega
\end{aligned}
\end{gather}
(ii) After the energy measurement, the state collapses to 
\begin{eqnarray}
    \ket{\psi(0)} = \frac{1}{\sqrt{2}} ( \ket{1,0} + \ket{0,1} )
,\end{eqnarray}
and if the system is allowed to evolve,
\begin{eqnarray}
    \ket{\psi(t)} = e^{-i 2 \omega t} \ket{\psi(0)}
.\end{eqnarray}
Notice that there is only a phase factor since $E_{nm} = \hbar \omega (n_1 + n_2 + 1)$ and $E_{10} = E_{01}$.
Thus, it is immediately clear that the ``physics'' is actually stationary in time.
That is, if we perform a measurement of the system's total energy, we must get $E = 2 \hbar \omega$ with certainty.
On the other hand, a position measurement can yield any real number with probability given by the square wave-function $\Big|\bra{\phi_{x}^{(1)},\phi_{x}^{(2)}} \ket{\psi(0)} \Big|^2$.

(b) If $E_2 = \hbar \omega / 2$ is measured at $t = 0$, the wave-function collapses to
\begin{eqnarray}
    \ket{\psi(0)} = \frac{1}{\sqrt{2}} ( \ket{0,0} + \ket{1,0} )
,\end{eqnarray}
and in this case, the system dynamics are non-trivial:
\begin{eqnarray}
    \ket{\psi(t)} = \frac{1}{\sqrt{2}} ( e^{-i \omega t} \ket{0,0} + e^{-i 2 \omega t} \ket{1,0} ) = \frac{e^{-i\omega t}}{\sqrt{2}} ( \ket{0,0} + e^{-i \omega t} \ket{1,0} )
.\end{eqnarray}
At time $t = 0$
\begin{align}
\begin{aligned}
    \expval{x_1} &= \frac{1}{2} \sqrt{\frac{\hbar}{2 m \omega}} ( \bra{0,0} + \bra{1,0} ) (a^{\dagger}_{1} + a_{1}) ( \ket{0,0} + \ket{1,0} ) \\
               &= \frac{1}{2} \sqrt{\frac{\hbar}{2 m \omega}} ( \bra{0,0} + \bra{1,0} ) ( \ket{1,0} + \sqrt{2} \ket{2,0} ) \\
               &= \frac{1}{2} \sqrt{\frac{\hbar}{2 m \omega}}
\end{aligned}
\\
\begin{aligned}
    \expval{p_1} &= \frac{1}{2} \sqrt{\frac{\hbar m \omega}{2}} ( \bra{0,0} + \bra{1,0} ) (a^{\dagger}_{1} - a_1) ( \ket{0,0} + \ket{1,0} ) \\
                 &= \frac{1}{2} \sqrt{\frac{\hbar m \omega}{2}} ( \bra{0,0} + \bra{1,0} ) ( \ket{1,0} + \sqrt{2} \ket{2,0} - \ket{0,0} ) \\
                 &= 0
\end{aligned}
\\
\expval{H_1} = \frac{1}{2} [ ( 0 + 1/2 ) + (1 + 1/2) ] \hbar \omega = \hbar \omega
\end{align}
Considering the time dependence, the possible energy values of particle 1 are $\hbar \omega / 2$ and $3 \hbar \omega / 2$ with equal probability, and we can get any real number for the position given by the square of the wave-function.

}


\prob{4 -- Chapter 8 \# 6}{

The Hamiltonian of two particles is given by
\begin{eqnarray}
    \hat{H} = \hat{H}_1 + \hat{H}_2, \quad \hat{H}_{i} = \frac{\hat{p}_{i}^2}{2m} + \frac{m \omega^2}{2} \hat{x}_{i}^2
,\end{eqnarray}
where $\hat{x}_{i}$ and $\hat{p}_{i}$ are the position and momentum operators of particle $i$.
Let $\ket{n_1,n_2}$ denote the common eigenstates of $\hat{H}_1$ and $\hat{H}_2$ with eigenvalues $\hbar \omega (n_1 + 1/2)$ and $\hbar \omega (n_2 + 1/2)$, respectively.
We define the two-particle exchange operator as
\begin{eqnarray}
    \hat{P}_{e}\ket{n_1,n_2} = \ket{n_2,n_1}
.\end{eqnarray}

(a) Show that $\hat{P}_{e}^{-1} = \hat{P}_{e}$ and that $\hat{P}_{e}$ is unitary.
What are the eigenvalues of $\hat{P}_{e}$?
Let $\hat{A}' = \hat{P}_{e} \hat{A} \hat{P}_{e}^{\dagger}$ be the observable resulting from the transformation by $\hat{P}_{e}$ of an arbitrary observable $\hat{A}$.
Show that the condition $\hat{A}' = \hat{A}$ is equivalent to $[ \hat{A}, \hat{P}_{e} ] = 0$.

(b) Show that
\begin{eqnarray}
    \hat{P}_{e} \hat{H}_{1} \hat{P}_{e}^{\dagger} = \hat{H}_{2}, \quad \hat{P}_{e} \hat{H}_{2} \hat{P}_{e}^{\dagger} = \hat{H}_{1} 
.\end{eqnarray}
Does $\hat{H}$ commute with $\hat{P}_{e}$?
Calculate the action of $\hat{P}_{e}$ on the observables $\hat{x}_{1,2}$ and $\hat{p}_{1,2}$.

(c) Construct a basis of common eigenstates of $\hat{H}$ and $\hat{P}_{e}$.
Do these two operators form a complete set of commuting observables?
What happens to the spectrum of $\hat{H}$ and the degeneracy of its eigenvalues if only those eigenvectors of $\hat{H}$ are retained such that they are odd under exchange?

}

\sol{

(a) It is clear that $\hat{P}_{e}$ is its own inverse since
\begin{eqnarray}
    \hat{P}_{e}^2 \ket{n_1,n_2} = \hat{P}_{e} \ket{n_2,n_1} = \ket{n_1,n_2}
.\end{eqnarray}
We also can see fairly easily that $\hat{P}_{e}$ is unitary since
\begin{eqnarray}
    \bra{n_1',n_2'} \hat{P}_{e}^{\dagger} \hat{P}_{e} \ket{n_1,n_2} = \bra{n_2',n_1'} \ket{n_2,n_1} = \delta_{n_2,n_2'} \delta_{n_1,n_1'}
,\end{eqnarray}
meaning that $\hat{P}_{e}^{\dagger} \hat{P}_{e} = \hat{\id}$.
We can actually use these two facts to show that
\begin{eqnarray}
    \hat{P}_{e}^{\dagger} \hat{P}_{e}^2 = \hat{P}_{e} = \hat{P}_{e}^{\dagger}
.\end{eqnarray}

The eigenvalues of $\hat{P}_{e}$ are $\pm 1$.
If $\hat{A}' = \hat{A}$, then
\begin{gather}
    \hat{P}_{e} \hat{A} \hat{P}_{e}^{\dagger} = \hat{A} \nonumber \\
    \hat{P}_{e} \hat{A} = \hat{A} \hat{P}_{e} \nonumber \\
    \hat{A} \hat{P}_{e} - \hat{P}_{e} \hat{A} = [\hat{A}, \hat{P}_{e}] = 0
.\end{gather}

(b) Operators are equal if their matrix elements are the same:
\begin{eqnarray}
\begin{aligned}
    \bra{n_1',n_2'} P_{e} \hat{H}_1 \hat{P}_{e}^{\dagger} \ket{n_1,n_2} &= \bra{n_2',n_1'} H_1 \ket{n_2,n_1} \\
                                                                        &= \hbar \omega ( n_2 + 1/2 ) \delta_{n_2,n_2'} \delta_{n_1,n_1'} \\
                                                                        &= \bra{n_1',n_2'} \hat{H}_2 \ket{n_1,n_2}
\end{aligned}
.\end{eqnarray}
Since this has to be true for any matrix element, $\hat{P}_{e} \hat{H}_1 \hat{P}_{e}^{\dagger} = \hat{H}_2$.
The manipulations are the same for the second identity.
From these, it is immediate that
\begin{eqnarray}
    \hat{P}_{e} \hat{H} \hat{P}_{e}^{\dagger} = \hat{P}_{e} \hat{H}_{1} \hat{P}_{e}^{\dagger} + \hat{P}_{e} \hat{H}_2 \hat{P}_{e}^{\dagger} = \hat{H}_{2} + \hat{H}_{1} = \hat{H}
,\end{eqnarray}
and therefore $[ \hat{H},\hat{P}_{e} ] = 0$.
The action of $\hat{P}_{e}$ on the position and momentum operators is the same as on the Hamiltonian, and furthermore, any operator that acts solely on the individual particle Hilbert spaces has the same form.
The operator $\hat{P}_{e}$ therefore essentially exhanges the identities of the particles.

(c) We can simply construct the common eigenstates as
\begin{eqnarray}
    \frac{1}{\sqrt{2}} ( \ket{n_1,n_2} \pm \ket{n_2,n_1} )
,\end{eqnarray}
corresponding to total energy eigenvalue $\hbar \omega (n_1 + n_2 + 1)$ and permutation eigenvalue $\pm 1$.
It should be clear that these operators do not form a complete set of commuting observables since the specification of both eigenvalues does not uniquely specify the state.
If we retain only eigenstates which are odd under exchange, there is still degeneracy in the spectrum, but not quite as much for even $n_1 + n_2$.
We see a form of the Pauli exclusion principle here, which is that if the system is in a state which is odd under exchange, then the individual particles cannot be in the same state with respect to their individual Hamiltonians.
That is $( \ket{n,n} - \ket{n,n} )/\sqrt{2} = 0$.

}




\end{document}
