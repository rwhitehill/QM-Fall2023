\def\duedate{\today}
\def\HWnum{2}
% Document setup
\documentclass[12pt]{article}
\usepackage[margin=1in]{geometry}
\usepackage{fancyhdr}
\usepackage{lastpage}

\pagestyle{fancy}
\lhead{Richard Whitehill}
\chead{PHYS 621 -- HW \HWnum}
\rhead{\duedate}
\cfoot{\thepage \hspace{1pt} of \pageref{LastPage}}

% Encoding
\usepackage[utf8]{inputenc}
\usepackage[T1]{fontenc}

% Math/Physics Packages
\usepackage{amsmath}
\usepackage{amssymb}
\usepackage{mathtools}
\usepackage[arrowdel]{physics}
\usepackage{siunitx}

\AtBeginDocument{\RenewCommandCopy\qty\SI}

% Reference Style
\usepackage{hyperref}
\hypersetup{
    colorlinks=true,
    linkcolor=blue,
    filecolor=magenta,
    urlcolor=cyan,
    citecolor=green
}

\newcommand{\eref}[1]{Eq.~(\ref{eq:#1})}
\newcommand{\erefs}[2]{Eqs.~(\ref{eq:#1})--(\ref{eq:#2})}

\newcommand{\fref}[1]{Fig.~\ref{fig:#1}}
\newcommand{\frefs}[2]{Figs.~\ref{fig:#1}--\ref{fig:#2}}

\newcommand{\tref}[1]{Table~\ref{tab:#1}}
\newcommand{\trefs}[2]{Tables~\ref{tab:#1}-\ref{tab:#2}}

% Figures and Tables 
\usepackage{graphicx}
\usepackage{float}

\newcommand{\bef}{\begin{figure}[h!]\begin{center}}
\newcommand{\eef}{\end{center}\end{figure}}

\newcommand{\bet}{\begin{table}[h!]\begin{center}}
\newcommand{\eet}{\end{center}\end{table}}

% tikz
\usepackage{tikz}
\usetikzlibrary{calc}
\usetikzlibrary{decorations.pathmorphing}
\usetikzlibrary{decorations.markings}
\usetikzlibrary{arrows.meta}
\usetikzlibrary{positioning}

% tcolorbox
\usepackage[most]{tcolorbox}
\usepackage{xcolor}
\usepackage{xifthen}
\usepackage{parskip}

\newcommand*{\eqbox}{\tcboxmath[
    enhanced,
    colback=black!10!white,
    colframe=black,
    sharp corners,
    size=fbox,
    boxsep=8pt,
    boxrule=1pt
]}

% problem-solution macros
% \usepackage{adjustbox}
\usepackage{changepage}

\newtcolorbox{probbox}[1][]{
    breakable,
    enhanced,
    boxrule=0pt,
    frame hidden,
    borderline west={4pt}{0pt}{green!50!black},
    colback=green!5,
    before upper=\textbf{Problem #1) \,},
    % \textbf{Problem #1 \ifthenelse{\isempty{#1}}{}{: #1} \\ },
    sharp corners
}

% \newtcolorbox{ProblemBox}[1][]{%
%   breakable,
%   enhanced,
%   colback=black!10!white,
%   colframe=black,
%   title={\large #1 \hfill}
% }
\newcommand{\prob}[2]{
\begin{probbox}[#1]
#2
\end{probbox}
}

\newenvironment{solution}{\begin{adjustwidth}{8pt}{8pt}}{\end{adjustwidth}}
\newcommand{\sol}[1]{
\begin{solution}
#1
\end{solution}
}
% \textbf{#1)} #2}

% Miscellaneous Definitions/Settings
\newcommand{\reals}{\mathbb{R}}
\newcommand{\integers}{\mathbb{Z}}
\newcommand{\naturals}{\mathbb{N}}
\newcommand{\rationals}{\mathbb{Q}}
\newcommand{\complexs}{\mathbb{C}}

\setlength{\parskip}{\baselineskip}
\setlength{\parindent}{0pt}
\setlength{\headheight}{14.49998pt}
\addtolength{\topmargin}{-2.49998pt}


\begin{document}

\prob{1}{

(a) Show that the wave function at time $t$ for a free particle of mass $m$ can be written as
\begin{eqnarray}
\Psi(\va*{r},t) = \int \dd[3]{\va*{r}_{0}} G(\va*{r}-\va*{r}_{0},t-t_0) \Psi(\va*{r}_{0},t_{0})
,\end{eqnarray}
where $\Psi(\va*{r},t_{0})$ is the wave function at the initial time $t_0$ and the function $G(\va*{r}-\va*{r}_{0},t-t_0)$, known as the free-particle Green's function, reads
\begin{eqnarray}
    G(\va*{r} - \va*{r}_{0}, t - t_0) = \int \frac{\dd[3]{\va*{p}}}{(2 \pi \hbar)^3} e^{i[\va*{p} \cdot (\va*{r} - \va*{r}_{0}) - E_{p} (t - t_0)]/\hbar}, \quad E_{p} = \frac{p^2}{2m}
.\end{eqnarray}
\textbf{Hint}: The free-particle wave funcction can be generally written as the superposition (wave packet)
\begin{eqnarray}
    \Psi(\va*{r},t) = \int \frac{\dd[3]{\va*{p}}}{(2 \pi \hbar)^{3/2}} f(\va*{p}) e^{i(\va*{p} \cdot \va*{r} - E_{p}t)/\hbar}
.\end{eqnarray}

(b) Obtain the explicit expression for the Green's function.
\textbf{Hint}: Use the following integral
\begin{eqnarray}
   \int_{-\infty}^{\infty} e^{-\alpha^2(x - \beta)^2} = \frac{\sqrt{\pi}}{\alpha}
,\end{eqnarray}
where $\alpha$ and $\beta$ are generally complex numbers with $-\pi/4 < {\rm arg} ~ \alpha < \pi/4$ for convergence.

}

\sol{

(a) Plugging in the expressions for $G(\va*{r} - \va*{r}_{0}, t - t_0)$ and $\Psi(\va*{r}_{0},t_0)$, we have
\begin{eqnarray}
   \begin{aligned}
       \int \dd[3]{\va*{r}_{0}} ~ &G(\va*{r} - \va*{r}_{0},t - t_0) \Psi(\va*{r}_{0},t_0) \\
       &= \int \dd[3]{\va*{r}_{0}} \Bigg[ \int \frac{\dd[3]{\va*{p}}}{(2 \pi \hbar)^{3}} e^{i[ \va*{p} \cdot (\va*{r} - \va*{r}_{0}) - E_{p} (t - t_0) ] / \hbar} \Bigg] \Bigg[ \int \frac{\dd[3]{\va*{q}}}{(2 \pi \hbar)^{3/2}} f(\va*{q}) e^{i( \va*{q} \cdot \va*{r}_{0} - E_{q} t_0) / \hbar } \Bigg] \\
                       &= \int \frac{\dd[3]{\va*{p}}}{(2 \pi \hbar)^{3/2}} \dd[3]{\va*{q}} \frac{\dd[3]{\va*{r}_{0}}}{(2 \pi \hbar)^3} e^{i (\va*{q} - \va*{p}) \cdot \va*{r}_{0} / \hbar} f(\va*{q}) e^{i( \va*{p} \cdot \va*{r} - E_{p} t)/\hbar} e^{i( E_{p} - E_{q} )t_0/\hbar} \\
                       &= \int \frac{\dd[3]{\va*{p}}}{(2 \pi \hbar)^{3/2}} \dd[3]{\va*{q}} \delta^{(3)}(\va*{q} - \va*{p}) f(\va*{q}) e^{i( \va*{p} \cdot \va*{r} - E_{p} t)/\hbar} e^{i( E_{p} - E_{q} )t_0/\hbar} \\
                       &= \eqbox{ \int \frac{\dd[3]{\va*{p}}}{(2 \pi \hbar)^{3/2}} f(\va*{p}) e^{i ( \va*{p} \cdot \va*{r} - E_{p} t ) / \hbar} = \Psi(\va*{r},t) }
   .\end{aligned} 
\end{eqnarray}

(b) Notice that we can write
\begin{eqnarray}
    \begin{aligned}
    G(\va*{r} - \va*{r}_0, t - t_0) &= \int \frac{\dd[3]{\va*{p}}}{(2 \pi \hbar)^{3}} e^{i[ \va*{p} \cdot (\va*{r} - \va*{r}_{0}) - ( \va*{p}^2/2m ) (t - t_0) ] / \hbar} \\
                                    &= G(x - x_0, t - t_0) G(y - y_0, t - t_0) G(z - z_0, t - t_0)
    ,\end{aligned}
\end{eqnarray}
where
\begin{eqnarray}
    G(x - x_0, t - t_0) = \int \frac{\dd{p_{x}}}{2 \pi \hbar} e^{i[ p_{x} (x - x_0) - (p_{x}^2/2m)(t - t_0) ] / \hbar}
\end{eqnarray}
Thus, we can solve our three-dimensional problem by splicing together three copies of the solution to a one-dimensional problem.
Before moving forward, let us denote $\Delta x = x - x_0$ and $\Delta t = t - t_0$ to make the writing slightly less cumbersome (it is also slightly more illustrative).
Doing the integrations, we find
\begin{eqnarray}
    G(x - x_0, t - t_0) = \frac{1}{2 \pi \hbar} \int \dd{p_{x}} e^{-i \Delta t( p_{x}^2 - 2m \frac{\Delta x}{\Delta t} p_{x} ) / 2m}
.\end{eqnarray}


}


\prob{2}{

Show that the probability density and probability current density at position $\va*{r}_{0}$ can be expressed as expectation values of the operators $\rho(\va*{r}_{0})$ and $\va*{j}(\va*{r}_{0})$, defined as
\begin{eqnarray}
    \rho(\va*{r}_{0}) = \delta(\va*{r} - \va*{r}_{0}), \quad \va*{j}(\va*{r}_{0}) = \frac{1}{2m} [\va*{p} \delta(\va*{r} - \va*{r}_{0}) + \delta(\va*{r} - \va*{r}_{0})\va*{p}]
,\end{eqnarray}
where $\va*{r}$ and $\va*{p}$ are the position and momentum operators.
Derive the expressions for these densities in both coordinate and momentum space.

}




\end{document}
