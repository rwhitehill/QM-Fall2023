\def\duedate{\today}
\def\HWnum{2}
\documentclass[10pt,a4paper]{book}

% custom section formatting
\usepackage{titlesec}
\titleformat{\chapter}[display]
{\normalfont\Large\filcenter\sffamily}
{\titlerule[1pt]%
\vspace{1pt}%
\titlerule
\vspace{1pc}%
\LARGE\MakeUppercase{\chaptertitlename} \thechapter}
{1pc}
{\titlerule
\vspace{1pc}%
\Huge}

% appendix handling
\usepackage[toc,page]{appendix}
    
% encoding for file and font
\usepackage[utf8]{inputenc}
\usepackage[T1]{fontenc}

% math formatting/tools
\usepackage{amsmath}
\usepackage{amssymb}
\usepackage{mathtools}
\usepackage[arrowdel]{physics}

% unit formatting
\usepackage{siunitx}
\AtBeginDocument{\RenewCommandCopy\qty\SI}

% figure formatting/tools
\usepackage{graphicx}
\usepackage{float}
\usepackage{subcaption}
\usepackage{multirow}
\usepackage{import}
\usepackage{pdfpages}
\usepackage{transparent}
\usepackage{currfile}

\NewDocumentCommand\incfig{O{1} m}{
    \def\svgwidth{#1\textwidth}
    \import{./Figures/\currfiledir}{#2.pdf_tex}
}

\newcommand{\bef}{\begin{figure}[h!tb]\centering}
\newcommand{\eef}{\end{figure}}

\newcommand{\bet}{\begin{table}[h!tb]\centering}
\newcommand{\eet}{\end{table}}

% hyperlink references 
\usepackage{hyperref}
\hypersetup{
    colorlinks=true,
    linkcolor=blue,
    filecolor=magenta,
    urlcolor=cyan,
    pdftitle={Physics 1 Notes},
    pdfauthor={Richard Whitehill},
    pdfpagemode=FullScreen
}
\urlstyle{same}

\newcommand{\eref}[1]{Eq.~(\ref{eq:#1})}
\newcommand{\erefs}[2]{Eqs.~(\ref{eq:#1})--(\ref{eq:#2})}

\newcommand{\fref}[1]{Fig.~(\ref{fig:#1})}
\newcommand{\frefs}[2]{Fig.~(\ref{fig:#1})--(\ref{fig:#2})}

\newcommand{\aref}[1]{Appendix~(\ref{app:#1})}
\newcommand{\sref}[1]{Section~(\ref{sec:#1})}
\newcommand{\srefs}[2]{Sections~(\ref{sec:#1})-(\ref{sec:#2})}

\newcommand{\tref}[1]{Table~(\ref{tab:#1})}
\newcommand{\trefs}[2]{Table~(\ref{tab:#1})--(\ref{tab:#2})}

% tcolorbox formatting/definitions
\usepackage[most]{tcolorbox}
\usepackage{xcolor}
\usepackage{xifthen}
\usepackage{parskip}

\definecolor{peach}{rgb}{1.0,0.8,0.64}

\DeclareTColorBox[auto counter, number within=chapter]{defbox}{O{}}{
    enhanced,
    boxrule=0pt,
    frame hidden,
    borderline west={4pt}{0pt}{green!50!black},
    colback=green!5,
    before upper=\textbf{Definition \thetcbcounter \ifthenelse{\isempty{#1}}{}{: #1} \\ },
    sharp corners
}

\newcommand*{\eqbox}{\tcboxmath[
    enhanced,
    colback=black!10!white,
    colframe=black,
    sharp corners,
    size=fbox,
    boxsep=8pt,
    boxrule=1pt
]}

\newtcolorbox[auto counter, number within=chapter]{exbox}{
    parbox=false,
    breakable,
    enhanced,
    sharp corners,
    boxrule=1pt,
    colback=white,
    colframe=black,
    before upper= \textbf{Example \thetcbcounter:}\,,
    before lower= \textbf{Solution:}\,,
    segmentation hidden
}

\newtcolorbox{resbox}{
    enhanced,
    colback=black!10!white,
    colframe=black,
    boxrule=1pt,
    boxsep=0pt,
    top=2pt,
    ams nodisplayskip,
    sharp corners
}


\begin{document}

\prob{1}{

(a) Show that the wave function at time $t$ for a free particle of mass $m$ can be written as
\begin{eqnarray}
    \label{eq:prob1-prop}
\Psi(\va*{r},t) = \int \dd[3]{\va*{r}_{0}} G(\va*{r}-\va*{r}_{0},t-t_0) \Psi(\va*{r}_{0},t_{0})
,\end{eqnarray}
where $\Psi(\va*{r},t_{0})$ is the wave function at the initial time $t_0$ and the function $G(\va*{r}-\va*{r}_{0},t-t_0)$, known as the free-particle Green's function, reads
\begin{eqnarray}
    \label{eq:free-greens}
    G(\va*{r} - \va*{r}_{0}, t - t_0) = \int \frac{\dd[3]{\va*{p}}}{(2 \pi \hbar)^3} e^{i[\va*{p} \cdot (\va*{r} - \va*{r}_{0}) - E_{p} (t - t_0)]/\hbar}, \quad E_{p} = \frac{p^2}{2m}
.\end{eqnarray}
\textbf{Hint}: The free-particle wave funcction can be generally written as the superposition (wave packet)
\begin{eqnarray}
    \Psi(\va*{r},t) = \int \frac{\dd[3]{\va*{p}}}{(2 \pi \hbar)^{3/2}} f(\va*{p}) e^{i(\va*{p} \cdot \va*{r} - E_{p}t)/\hbar}
.\end{eqnarray}

(b) Obtain the explicit expression for the Green's function.
\textbf{Hint}: Use the following integral
\begin{eqnarray}
   \int_{-\infty}^{\infty} e^{-\alpha^2(x - \beta)^2} = \frac{\sqrt{\pi}}{\alpha}
,\end{eqnarray}
where $\alpha$ and $\beta$ are generally complex numbers with $-\pi/4 < {\rm arg} ~ \alpha < \pi/4$ for convergence.

}

\sol{

(a) Plugging in the expressions for $G(\va*{r} - \va*{r}_{0}, t - t_0)$ and $\Psi(\va*{r}_{0},t_0)$, we have
\begin{eqnarray}
   \begin{aligned}
       \int \dd[3]{\va*{r}_{0}} ~ &G(\va*{r} - \va*{r}_{0},t - t_0) \Psi(\va*{r}_{0},t_0) \\
       &= \int \dd[3]{\va*{r}_{0}} \Bigg[ \int \frac{\dd[3]{\va*{p}}}{(2 \pi \hbar)^{3}} e^{i[ \va*{p} \cdot (\va*{r} - \va*{r}_{0}) - E_{p} (t - t_0) ] / \hbar} \Bigg] \Bigg[ \int \frac{\dd[3]{\va*{q}}}{(2 \pi \hbar)^{3/2}} f(\va*{q}) e^{i( \va*{q} \cdot \va*{r}_{0} - E_{q} t_0) / \hbar } \Bigg] \\
                       &= \int \frac{\dd[3]{\va*{p}}}{(2 \pi \hbar)^{3/2}} \dd[3]{\va*{q}} \frac{\dd[3]{\va*{r}_{0}}}{(2 \pi \hbar)^3} e^{i (\va*{q} - \va*{p}) \cdot \va*{r}_{0} / \hbar} f(\va*{q}) e^{i( \va*{p} \cdot \va*{r} - E_{p} t)/\hbar} e^{i( E_{p} - E_{q} )t_0/\hbar} \\
                       &= \int \frac{\dd[3]{\va*{p}}}{(2 \pi \hbar)^{3/2}} \dd[3]{\va*{q}} \delta^{(3)}(\va*{q} - \va*{p}) f(\va*{q}) e^{i( \va*{p} \cdot \va*{r} - E_{p} t)/\hbar} e^{i( E_{p} - E_{q} )t_0/\hbar} \\
                       &= \eqbox{ \int \frac{\dd[3]{\va*{p}}}{(2 \pi \hbar)^{3/2}} f(\va*{p}) e^{i ( \va*{p} \cdot \va*{r} - E_{p} t ) / \hbar} = \Psi(\va*{r},t) }
   ,\end{aligned} 
\end{eqnarray}
verifying the validity of \eref{free-greens}

As a derivation of $G$ from the form of $\Psi(\va*{r},t)$, we first determine $f(\va*{p})$ as a function of $\Psi(\va*{r},t_0)$, where $t_0$ is some initial time:
\begin{eqnarray}
    \begin{aligned}
        \tilde{\Psi}(\va*{q},t_{0}) &= \int \frac{\dd[3]{\va*{r}}}{(2 \pi \hbar)^{3/2}} e^{- i \va*{q} \cdot \va*{r} / \hbar} \Psi(\va*{r},t_{0}) = \int \frac{\dd[3]{\va*{r}}}{(2 \pi \hbar)^{3/2}} e^{-i \va*{q} \cdot \va*{r} / \hbar} \Bigg[ \int \frac{\dd[3]{\va*{p}}}{(2 \pi \hbar)^{3/2}} f(\va*{p}) e^{i(\va*{p} \cdot \va*{r} - E_{p} t_{0}) / \hbar} \Bigg] \\
                                    &= f(\va*{q}) e^{-i E_{q} t_0 / \hbar} \Rightarrow f(\va*{q}) = e^{i E_{q} t_0 / \hbar} \int \frac{\dd[3]{\va*{r}}}{(2 \pi \hbar)^{3/2}} e^{-i \va*{q} \cdot \va*{r} / \hbar} \Psi(\va*{r},t_0)
    \end{aligned}
.\end{eqnarray}
We can then substitute this into the wave-packet definition of $\Psi$ and arrive at
\begin{eqnarray}
    \eqbox{
    \begin{aligned}
        \Psi(\va*{r},t) &= \int \frac{\dd[3]{\va*{p}}}{(2 \pi \hbar)^{3/2}} \Bigg[ e^{i E_{p} t_0 / \hbar} \int \frac{\dd[3]{\va*{r}_{0}}}{(2 \pi \hbar)^{3/2}} e^{-i \va*{p} \cdot \va*{r}_{0} / \hbar} \Psi(\va*{r}_{0},t_0) \Bigg] e^{i(\va*{p} \cdot \va*{r} - E_{p} t) / \hbar} \\
                        &= \int \dd[3]{\va*{r}_{0}} \underbrace{ \Bigg[ \int \frac{\dd[3]{\va*{p}}}{(2 \pi \hbar)^3} e^{i [ \va*{p} \cdot (\va*{r} - \va*{r}_{0}) - E_{p} (t - t_0) ] / \hbar} \Bigg] }_{G(\va*{r} - \va*{r}_{0}, t - t_0)} \Psi(\va*{r}_{0}, t_0) 
    .\end{aligned}
}
\end{eqnarray}


We can perform a more general derivation than we previously undertook, which gives the Green's function for the free particle as a special case.
The interpretation of the expression \eref{prob1-prop} is that the function $G$ tells us how to evolve the initial state $\Psi(\va*{r}_{0},t_0)$ to a state $\Psi(\va*{r},t)$ via an integral transform, where $G$ is the kernel of such transform.
Indeed, another name for $G$ is simply \textit{the propagator}.
We can solve the time-dependent Schr\"{o}dinger equation, treating $H$ almost as a constant, giving
\begin{eqnarray}
    \pdv{\ket{\Psi}}{t} = -\frac{i}{\hbar}H \ket{\Psi} \Rightarrow \ket{\Psi(t)} = \underbrace{ e^{-\frac{i}{\hbar} H (t - t_0)}}_{{\cal U}(t,t_0)} \ket{\Psi(t_0)}
.\end{eqnarray}
Hence, $\cal U$ is the unitary time evolution operator for the wave function.
We can take this a step further by expanding $\ket{\Psi}$ in the position basis (i.e. $\Psi(\va*{r},t) = \bra{\va*{r}}\ket{\Psi(t)}$).
\begin{eqnarray}
    \Psi(\va*{r},t) = \int \dd[3]{\va*{r}_0} \bra{\va*{r}} \mathcal{U}(t - t_0) \ket{\va*{r}_{0}} \Psi(\va*{r}_{0},t)
.\end{eqnarray}
Thus, $G$ is simply a position matrix element of the time evolution operator.

For a free particle,
\begin{eqnarray}
    \mathcal{U}(t - t_0) = e^{-i (\va*{p}^2/2 m \hbar) (t - t_0) }
,\end{eqnarray}
and therefore, inserting two complete sets of momentum states, we have
\begin{eqnarray}
    \begin{aligned}
        G(\va*{r},\va*{r}_{0};t-t_0) &= \bra{\va*{r}} \mathcal{U} \ket{\va*{r}_{0}} = \int \dd[3]{\va*{p}_{1}} \dd[3]{\va*{p}_{2}} \frac{e^{i \va*{p}_{1} \cdot \va*{r} / \hbar}}{(2 \pi \hbar)^{3/2}} \delta(\va*{p}_{1} - \va*{p}_{2}) e^{-\frac{i}{\hbar} \frac{\va*{p}_{2}^2}{2m} (t - t_0)} \frac{e^{-i \va*{p}_{2} \cdot \va*{r}_{0} / \hbar}}{(2 \pi \hbar)^{3/2}} \\
                                     &= \eqbox{ \int \frac{\dd[3]{\va*{p}}}{(2 \pi \hbar)^3} e^{i [ \va*{p} \cdot (\va*{r} - \va*{r}_{0}) - E_{p}(t - t_0) ]/\hbar} = G(\va*{r} - \va*{r}_{0},t-t_0) }
    .\end{aligned}
\end{eqnarray}
Note that we have used $\bra{\va*{r}}\ket{\va*{p}} = \exp{i \va*{p} \cdot \va*{r} / \hbar}/(2 \pi \hbar)^{3/2}$, which is the coordinate space momentum eigenfunction defined by $- \frac{i}{\hbar} \grad f_{\va*{p}}(\va*{r}) = \va*{p} f_{\va*{p}}(\va*{r})$, and $\bra{\va*{p}}\ket{\va*{p}_{0}} = \delta(\va*{p} - \va*{p}_{0})$.

(b) Notice that we can write
\begin{eqnarray}
    \begin{aligned}
    G(\va*{r} - \va*{r}_0, t - t_0) &= \int \frac{\dd[3]{\va*{p}}}{(2 \pi \hbar)^{3}} e^{i[ \va*{p} \cdot (\va*{r} - \va*{r}_{0}) - ( \va*{p}^2/2m ) (t - t_0) ] / \hbar} \\
                                    &= G(x - x_0, t - t_0) G(y - y_0, t - t_0) G(z - z_0, t - t_0)
    ,\end{aligned}
\end{eqnarray}
where
\begin{eqnarray}
    G(x - x_0, t - t_0) = \int \frac{\dd{p_{x}}}{2 \pi \hbar} e^{i[ p_{x} (x - x_0) - (p_{x}^2/2m)(t - t_0) ] / \hbar}
\end{eqnarray}
Thus, we can solve our three-dimensional problem by splicing together three copies of the solution to a one-dimensional problem.
Before moving forward, let us denote $\Delta x = x - x_0$ and $\Delta t = t - t_0$ to make the writing slightly less cumbersome.
Doing the integrations, we find
\begin{eqnarray}
    \begin{aligned}
        G(x - x_0, t - t_0) &= \frac{1}{2 \pi \hbar} \int \dd{p_{x}} e^{-i \Delta t( p_{x}^2 - 2m \frac{\Delta x}{\Delta t} p_{x} ) / 2m \hbar} \\
        &= \frac{1}{2 \pi \hbar} e^{-i \Delta t [- ( m \Delta x / \Delta t )^2] /2m \hbar} \int \dd{p_{x}} e^{-i \Delta t ( p_{x} - m \Delta x / \Delta t )^2 / 2m \hbar } \\
        &= \frac{1}{2 \pi \hbar} e^{i m \Delta x^2 / 2 \hbar \Delta t} \sqrt{ \frac{2 \pi \hbar m}{i \Delta t} } = \sqrt{ \frac{m}{2 i \pi \hbar \Delta t} } e^{i m \Delta x^2/2 \hbar \Delta t}
    .\end{aligned}
\end{eqnarray}
Splicing the results for each coordinate direction together, we have
\begin{eqnarray}
    \eqbox{ G(\va*{r} - \va*{r}_{0},t-t_0) = \Bigg( \frac{m}{2 i \pi \hbar (t - t_0)} \Bigg)^{3/2} \exp[ \frac{i m}{2 \hbar (t - t_0)} (\va*{r} - \va*{r}_{0})^2 ] }
.\end{eqnarray}

}


\prob{2}{

Show that the probability density and probability current density at position $\va*{r}_{0}$ can be expressed as expectation values of the operators $\rho(\va*{r}_{0})$ and $\va*{j}(\va*{r}_{0})$, defined as
\begin{eqnarray}
    \rho(\va*{r}_{0}) = \delta(\va*{r} - \va*{r}_{0}), \quad \va*{j}(\va*{r}_{0}) = \frac{1}{2m} [\va*{p} \delta(\va*{r} - \va*{r}_{0}) + \delta(\va*{r} - \va*{r}_{0})\va*{p}]
,\end{eqnarray}
where $\va*{r}$ and $\va*{p}$ are the position and momentum operators.
Derive the expressions for these densities in both coordinate and momentum space.

}

\sol{

We can simply take the expectation values of the operators above:
\begin{eqnarray}
    \eqbox{ \int \dd[3]{\va*{r}_{0}} \psi^{*}(\va*{r}_{0},t) \delta(\va*{r} - \va*{r}_{0}) \psi(\va*{r}_{0},t) = \psi^{*}(\va*{r},t)\psi(\va*{r},t) = | \psi(\va*{r},t) |^2 = \rho(\va*{r},t) }
\end{eqnarray}
and
\begin{eqnarray}
   \begin{aligned}
       \int &\dd[3]{\va*{r}_{0}} \psi^{*}(\va*{r}_{0},t) \frac{1}{2m}[ \va*{p} \delta(\va*{r} - \va*{r}_{0}) + \delta(\va*{r} - \va*{r}_{0}) \va*{p}] \psi(\va*{r}_{0},t) \\
            &= \frac{1}{2m} \Bigg[ \int \dd[3]{\va*{r}_0} \psi^{*}(\va*{r}_0,t) \va*{p} \delta(\va*{r} - \va*{r}_{0}) \psi(\va*{r}_{0},t) + \psi^{*}(\va*{r},t)\va*{p}\psi(\va*{r},t) \Bigg]
   .\end{aligned}
\end{eqnarray}
All that remains is to evaluate the first term in this expression.
Recall that the momentum operator is proportional to the gradient differential operator and hence respects the product rule.
That is,
\begin{eqnarray}
    \begin{aligned}
        \int &\dd[3]{\va*{r}_{0}} \psi^{*}(\va*{r}_0,t) \va*{p} \delta(\va*{r} - \va*{r}_{0}) \psi(\va*{r}_{0},t) \\
             &= \va*{p}[ \psi^{*}(\va*{r}_{0},t) \delta(\va*{r} - \va*{r}_0) \psi(\va*{r}_{0},t) ]_{r \rightarrow \infty} - \int \dd[3]{\va*{r}_{0}} \psi(\va*{r}_{0},t) \delta(\va*{r} - \va*{r}_{0}) \va*{p} \psi^{*}(\va*{r}_{0},t) \\
        &= -\psi(\va*{r},t) \va*{p} \psi^{*}(\va*{r},t)
    \end{aligned}
\end{eqnarray}
since the boundary term in the partial integration is identically zero given that $\psi \rightarrow 0$ as $r \rightarrow \infty$.
Putting this back into the original expression, we find
\begin{eqnarray}
    \eqbox{
   \begin{aligned}
       \int &\dd[3]{\va*{r}_{0}} \psi^{*}(\va*{r}_{0},t) \frac{1}{2m}[ \va*{p} \delta(\va*{r} - \va*{r}_{0}) - \delta(\va*{r} - \va*{r}_{0}) \va*{p}] \psi(\va*{r}_{0},t) \\
            &= \frac{1}{2m} \Big[ \psi^{*} \va*{p} \psi - \psi \va*{p} \psi^{*} \Big] = \frac{\hbar}{2 m i} \Big[ \psi^{*} \grad \psi - \psi \grad \psi^{*} \Big] = \va*{j}(\va*{r},t)
   .\end{aligned}
}
\end{eqnarray}

We also derive the expressions for the probability and probability current densities dependent on the momentum space variables.
First, the probability density\footnote{Note: the explicit time dependence is dropped but it is still there implicitly. It is just a bit cumbersome to write it everytime since our manipulations are only in coordinate and position space not time.}
\begin{eqnarray}
    \eqbox{
    \begin{aligned}
        \rho &= \psi^{*} \psi = \Bigg[ \int \frac{\dd[3]{\va*{p}}}{(2 \pi \hbar)^{3/2}} e^{-i \va*{p} \cdot \va*{r} / \hbar} \tilde{\psi}^{*}(\va*{p}) \Bigg] \Bigg[ \int \frac{\dd[3]{\va*{q}}}{(2 \pi \hbar)^{3/2}} e^{i \va*{q} \cdot \va*{r} / \hbar} \tilde{\psi}(\va*{q}) \Bigg] \\
             &= \int \frac{\dd[3]{\va*{p}} \dd[3]{\va*{q}}}{(2 \pi \hbar)^3} e^{-i (\va*{p} - \va*{q}) \cdot \va*{r} / \hbar} \tilde{\psi}^{*}(\va*{p}) \tilde{\psi}(\va*{q})
    \end{aligned}
}
.\end{eqnarray}

Now, we address the current density.
\begin{eqnarray}
    \begin{aligned}
        \psi^{*} \grad \psi &= \Bigg[ \int \frac{\dd[3]{\va*{p}}}{(2 \pi \hbar)^{3/2}} e^{-i \va*{p} \cdot \va*{r} / \hbar} \tilde{\psi}^{*}(\va*{p}) \Bigg] \grad \Bigg[ \int \frac{\dd[3]{\va*{q}}}{(2 \pi \hbar)^{3/2}} e^{i \va*{q} \cdot \va*{r} / \hbar} \tilde{\psi}(\va*{q}) \Bigg] \\
                            &= \frac{i}{\hbar} \int \frac{\dd[3]{\va*{p} \dd[3]{\va*{q}}}}{(2 \pi \hbar)^3} e^{-i \va*{p} \cdot \va*{r} / \hbar} \tilde{\psi}^{*}(\va*{p}) \, \va*{q} \, e^{i \va*{q} \cdot \va*{r} / \hbar} \tilde{\psi}(\va*{q}) \\
                            &= \frac{i}{\hbar} \int \frac{\dd[3]{\va*{p}} \dd[3]{\va*{q}}}{(2 \pi \hbar)^3} \va*{q} e^{-i (\va*{p} - \va*{q}) \cdot \va*{r} / \hbar} \tilde{\psi}^{*}(\va*{p}) \tilde{\psi}(\va*{q})
    .\end{aligned}
\end{eqnarray}

Note that we only have to work out the first term in brackets since it is the complex conjugate of the first (and therefore \textit{vice versa}).
This gives,
\begin{eqnarray}
   \eqbox{ 
    \begin{aligned}
        \va*{j} &= \frac{1}{2 m} \int \frac{\dd[3]{\va*{p}} \dd[3]{\va*{q}}}{(2 \pi \hbar)^3} \Big[ \va*{q} e^{-i (\va*{p} - \va*{q}) \cdot \va*{r} / \hbar} \tilde{\psi}^{*}(\va*{p}) \tilde{\psi}(\va*{q}) + \va*{q} e^{i (\va*{p} - \va*{q}) \cdot \va*{r} / \hbar} \tilde{\psi}(\va*{p}) \tilde{\psi}^{*}(\va*{q}) \big] \\
                &= \frac{1}{2 m} \int \frac{\dd[3]{\va*{p}} \dd[3]{\va*{q}}}{(2 \pi \hbar)^3} \big[ \va*{q} + \va*{p} \big] e^{-i (\va*{p} - \va*{q}) \cdot \va*{r} / \hbar} \tilde{\psi}^{*}(\va*{p}) \tilde{\psi}(\va*{q})
    ,\end{aligned}
}
\end{eqnarray}
where we have exchanged exchanged the dummy integration variables $\va*{p}$ and $\va*{q}$ in the second term.

}


\prob{3}{

Suppose that a system, described by the normalized wave function $\psi(x)$, has average values of the position and momentum operators given by, respectively, $\expval{x}$ and $\expval{p}$.
Now, consider the system being described by the wave function
\begin{eqnarray}
    \overline{\psi}(x) = e^{-i \expval{p} x / \hbar} \psi(x + \expval{x})
.\end{eqnarray}
Show that $\overline{\psi}(x)$ is normalized and calculate the expectation values of $x$ and $p$ in this case.

}

\sol{

We first show that $\overline{\psi}$ is normalized as follows:
\begin{eqnarray}
    \eqbox{ \int_{-\infty}^{\infty} | \overline{\psi}(x) |^2 \dd{x} = \int_{-\infty}^{\infty} | \psi(x + \expval{x}) |^2 \dd{x} = \int_{-\infty}^{\infty} | \psi(\bar{x}) |^2 \dd{\overline{x}} = 1}
,\end{eqnarray}
where we have defined $\overline{x} = x + \expval{x}$.

Next, we calculate $\expval{x}$ and $\expval{p}$, which are not much different from the calculation of the normalization above.
In the calculation of $\expval{x}$, the phase factor cancels since $e^{-i \varphi} e^{i \varphi} = 1$, and when we shift variables to $\overline{x} = x + \expval{x}$, we find
\begin{eqnarray}
    \eqbox{
   \begin{aligned}
       \expval{x}_{\overline{\psi}} &= \int_{-\infty}^{\infty} x |\psi(x + \expval{x})|^2 \dd{x} = \int_{-\infty}^{\infty} (\overline{x} - \expval{x}) | \psi(\overline{x}) |^2 \dd{\overline{x}} \\
                                    &= \int_{-\infty}^{\infty} \bar{x} \psi(\overline{x}) \dd{\overline{x}} - \expval{x} \int_{-\infty}^{\infty} | \psi(\overline{x}) |^2 \dd{\overline{x}} = 0
   .\end{aligned} 
}
\end{eqnarray}
and
\begin{eqnarray}
    \eqbox{
    \begin{aligned}
        \expval{p}_{\overline{\psi}} &= -i \hbar \int_{-\infty}^{\infty} e^{i \expval{p} x / \hbar} \psi(x + \expval{x}) \pdv{x} e^{-i \expval{p} x / \hbar} \psi(x + \expval{x}) \dd{x} \\
                                     &= -i \hbar \int_{-\infty}^{\infty} e^{i \expval{p} x / \hbar} \psi(x + \expval{x}) \Big[ -\frac{i \expval{p}}{\hbar} \psi(x + \expval{x}) + \pdv{x} \psi(x + \expval{x}) \Big] e^{- i \expval{p} x / \hbar} \dd{x} \\
                                     &= - \expval{p} + \int_{-\infty}^{\infty} \psi(\overline{x}) \Big( -i \hbar \pdv{\overline{x}} \Big) \psi(\overline{x}) \dd{\overline{x}} = 0
    .\end{aligned}
}
\end{eqnarray}
}


\prob{4}{

The time-dependent Schr\"{o}dinger equation in $r$-space for a particle under the influence of a potential $V(\va*{r})$ -- a real function of $\va*{r}$ -- given by
\begin{eqnarray}
    i \hbar \pdv{t}\Psi(\va*{r},t) = \Big[ -\frac{\hbar^2}{2m} \laplacian + V(\va*{r}) \Big] \Psi(\va*{r},t)
.\end{eqnarray}
Verify that there exists a probability current given by 
\begin{eqnarray}
    \va*{j}(\va*{r},t) = \frac{\hbar}{2 m i} [ \Psi^{*}(\va*{r},t) \grad \Psi(\va*{r},t) - \Psi(\va*{r},t) \grad \Psi^{*}(\va*{r},t) ]
,\end{eqnarray}
such that
\begin{eqnarray}
    \pdv{t} \rho(\va*{r},t) + \div{\va*{j}(\va*{r},t)} = 0
,\end{eqnarray}
where $\rho(\va*{r},t) = | \Psi(\va*{r},t) |^2$.

}

\sol{

The probability density for a particle with wave function $\Psi(\va*{r},t)$ is just $\rho(\va*{r},t) = | \Psi(\va*{r},t) |^2$.
In the case of the free particle, we derived the probability current by differentiating this probability density with respect to time.
It also followed from this derivation that the probability density and current satisfied a conserivation equation.
We do the same here, adding in a potential term to the Schr\"{o}dinger equation as follows (and essentially show that the terms with the potential cancel):
\begin{eqnarray}
    \label{eq:rho-deriv}
    \pdv{\rho}{t} = \Psi^{*} \pdv{\Psi}{t} + \Psi \pdv{\Psi^{*}}{t}
.\end{eqnarray}
The time derivative of the wave function (and its complex conjugate) is given by the Schr\"{o}dinger equation to be
\begin{align}
    \pdv{\Psi}{t} &= \phantom{-} \frac{1}{i \hbar} \Big[ - \frac{\hbar^2}{2m} \laplacian + V(\va*{r}) \Big] \Psi(\va*{r},t) \\
    \pdv{\Psi^{*}}{t} &= -\frac{1}{i \hbar} \Big[ - \frac{\hbar^2}{2m} \laplacian + V(\va*{r}) \Big] \Psi^{*}(\va*{r},t)
,\end{align}
and \eref{rho-deriv} simplifies to
\begin{eqnarray}
    \eqbox{
    \begin{aligned}
        \pdv{\rho}{t} &= \frac{1}{i \hbar} \Big[ \Psi^{*} \Big( -\frac{\hbar^2}{2m} \laplacian \Psi + V(\va*{r}) \Psi \Big) - \Psi \Big( -\frac{\hbar^2}{2m} \laplacian \Psi^{*} + V(\va*{r}) \Psi^{*} \Big) \Big] \\
                      &= - \frac{\hbar}{2 m i} \Big[ \Psi^{*} \laplacian \Psi - \Psi \laplacian \Psi^{*} \Big] = - \div{ \underbrace {\frac{\hbar}{2 m i} \Big[ \Psi^{*} \grad \Psi - \Psi \grad \Psi^{*} \Big]}_{\va*{j}(\va*{r},t)} } \\
                      &\Rightarrow \pdv{p(\va*{r},t)}{t} + \div{\va*{j}(\va*{r},t)} = 0
    .\end{aligned}
}
\end{eqnarray}


}



\end{document}
