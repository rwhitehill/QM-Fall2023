\def\duedate{\today}
\def\HWnum{12}
\documentclass[10pt,a4paper]{book}

% custom section formatting
\usepackage{titlesec}
\titleformat{\chapter}[display]
{\normalfont\Large\filcenter\sffamily}
{\titlerule[1pt]%
\vspace{1pt}%
\titlerule
\vspace{1pc}%
\LARGE\MakeUppercase{\chaptertitlename} \thechapter}
{1pc}
{\titlerule
\vspace{1pc}%
\Huge}

% appendix handling
\usepackage[toc,page]{appendix}
    
% encoding for file and font
\usepackage[utf8]{inputenc}
\usepackage[T1]{fontenc}

% math formatting/tools
\usepackage{amsmath}
\usepackage{amssymb}
\usepackage{mathtools}
\usepackage[arrowdel]{physics}

% unit formatting
\usepackage{siunitx}
\AtBeginDocument{\RenewCommandCopy\qty\SI}

% figure formatting/tools
\usepackage{graphicx}
\usepackage{float}
\usepackage{subcaption}
\usepackage{multirow}
\usepackage{import}
\usepackage{pdfpages}
\usepackage{transparent}
\usepackage{currfile}

\NewDocumentCommand\incfig{O{1} m}{
    \def\svgwidth{#1\textwidth}
    \import{./Figures/\currfiledir}{#2.pdf_tex}
}

\newcommand{\bef}{\begin{figure}[h!tb]\centering}
\newcommand{\eef}{\end{figure}}

\newcommand{\bet}{\begin{table}[h!tb]\centering}
\newcommand{\eet}{\end{table}}

% hyperlink references 
\usepackage{hyperref}
\hypersetup{
    colorlinks=true,
    linkcolor=blue,
    filecolor=magenta,
    urlcolor=cyan,
    pdftitle={Physics 1 Notes},
    pdfauthor={Richard Whitehill},
    pdfpagemode=FullScreen
}
\urlstyle{same}

\newcommand{\eref}[1]{Eq.~(\ref{eq:#1})}
\newcommand{\erefs}[2]{Eqs.~(\ref{eq:#1})--(\ref{eq:#2})}

\newcommand{\fref}[1]{Fig.~(\ref{fig:#1})}
\newcommand{\frefs}[2]{Fig.~(\ref{fig:#1})--(\ref{fig:#2})}

\newcommand{\aref}[1]{Appendix~(\ref{app:#1})}
\newcommand{\sref}[1]{Section~(\ref{sec:#1})}
\newcommand{\srefs}[2]{Sections~(\ref{sec:#1})-(\ref{sec:#2})}

\newcommand{\tref}[1]{Table~(\ref{tab:#1})}
\newcommand{\trefs}[2]{Table~(\ref{tab:#1})--(\ref{tab:#2})}

% tcolorbox formatting/definitions
\usepackage[most]{tcolorbox}
\usepackage{xcolor}
\usepackage{xifthen}
\usepackage{parskip}

\definecolor{peach}{rgb}{1.0,0.8,0.64}

\DeclareTColorBox[auto counter, number within=chapter]{defbox}{O{}}{
    enhanced,
    boxrule=0pt,
    frame hidden,
    borderline west={4pt}{0pt}{green!50!black},
    colback=green!5,
    before upper=\textbf{Definition \thetcbcounter \ifthenelse{\isempty{#1}}{}{: #1} \\ },
    sharp corners
}

\newcommand*{\eqbox}{\tcboxmath[
    enhanced,
    colback=black!10!white,
    colframe=black,
    sharp corners,
    size=fbox,
    boxsep=8pt,
    boxrule=1pt
]}

\newtcolorbox[auto counter, number within=chapter]{exbox}{
    parbox=false,
    breakable,
    enhanced,
    sharp corners,
    boxrule=1pt,
    colback=white,
    colframe=black,
    before upper= \textbf{Example \thetcbcounter:}\,,
    before lower= \textbf{Solution:}\,,
    segmentation hidden
}

\newtcolorbox{resbox}{
    enhanced,
    colback=black!10!white,
    colframe=black,
    boxrule=1pt,
    boxsep=0pt,
    top=2pt,
    ams nodisplayskip,
    sharp corners
}


\begin{document}

\prob{1 -- Chapter 11 \# 3}{

The state $\ket{\psi}$ is in the subspace spanned by the eigenstates of $\hat{\mathbf{J}}^2$ having eigenvalue $2\hbar^2$.
Suppose $\ket{\psi}$ is also a normalized eigenstate of $\vu*{n} \cdot \hat{\mathbf{J}}$ with eigenvalue $+\hbar$; here, $\vu*{n}$ is the unit vector with components $(\sin{\theta}\cos{\phi},\sin{\theta}\sin{\phi},\cos{\theta})$.
Obtain $\ket{\psi}$ as a linear combination of the eigenstates $\ket{m}$ of $\hat{J}_{z}$ with $m = 0, \pm 1$.

}

\sol{

We can write
\begin{eqnarray}
    \ket{\psi} = \sum_{m = -1}^{1} c_{m} \ket{m}
\end{eqnarray}
since $j = 1$ and hence the range of the sum is limited to $m=-1,0,1$ since $m$ can only take on values between $-j$ and $j$ separated by one unit.
From the definition of the dot product, we can write
\begin{eqnarray}
    \vu*{n} \vdot \va*{J} = \sin{\theta} \cos{\phi} J_1 + \sin{\theta} \sin{\phi} J_2 + \cos{\theta} J_3
.\end{eqnarray}
We can use the fact that $\ket{\psi}$ is an eigenvector of $\vu*{n} \cdot \va*{J}$ to write
\begin{eqnarray}    
    (\vu*{n} \vdot \va*{J}) \ket{\psi} = \hbar \ket{\psi} \Rightarrow \sum_{m=-1}^{1} c_{m} \Big[ \sin{\theta} \cos{\phi} J_1 \ket{m} + \sin{\theta} \sin{\phi} J_2 \ket{m} + \hbar ( m \cos{\theta} - 1 ) \ket{m} \Big] = 0
.\end{eqnarray}
At this point, we need to know the action of $J_1$ and $J_2$ on an eigenstate of $J_3$.
Recall the raising and lowering operators $J_{\pm} = J_1 \pm i J_2$ with $J_{+} \ket{m} = \hbar \sqrt{j(j+1) - m(m+1)} \ket{m+1}$ and $J_{-} \ket{m} = \hbar \sqrt{j(j+1) - m(m-1)} \ket{m - 1}$.
Solving for the Cartesian operators in terms of the raising and lowering operators, we have
\begin{eqnarray}
\begin{aligned}
    J_1 &= \frac{1}{2} \Big( J_{+} + J_{-} \Big) \\
    J_2 &= \frac{1}{2i} \Big( J_{+} - J_{-} \Big)
.\end{aligned}
\end{eqnarray}
Thus,
\begin{eqnarray}
    \cos{\phi} J_1 + \sin{\phi} J_2 = \frac{1}{2} \Big( e^{-i\phi} J_{+} + e^{i\phi} J_{-} \Big)
.\end{eqnarray}
Putting this into the sum above, we have
\begin{eqnarray}
    \sum_{m=-1}^{1} c_{m} \Big[ \frac{1}{2} e^{-i\phi} \sqrt{2 - m(m+1)} \ket{m+1} + \frac{1}{2} e^{i\phi} \sqrt{2 - m(m-1)} \ket{m-1} + (m \cos{\theta} - 1) \ket{m} \Big] = 0
.\end{eqnarray}
We can expand this in full glory, which gives:
\begin{eqnarray}
\begin{aligned}
    &\phantom{+} c_{-1} \Big[ \frac{1}{\sqrt{2}} e^{-i\phi} \ket{0} - (1 + \cos{\theta}) \ket{-1}  \Big] \\
    &+ c_{0} \Big[ \frac{1}{\sqrt{2}} e^{-i\phi} \ket{1} + \frac{1}{\sqrt{2}} e^{i\phi} \ket{-1} - \ket{0}  \Big] \\
    &+ c_1 \Big[ \frac{1}{\sqrt{2}} e^{i\phi} \ket{0} + (\cos{\theta} - 1) \ket{1} \Big] = 0
.\end{aligned}
\end{eqnarray}
Grouping terms and equating the coefficients in front of the eigenstates of $J_3$ to zero, we find
\begin{gather}
    -(1 + \cos{\theta}) c_{-1} + \frac{1}{\sqrt{2}} e^{i\phi} c_0 = 0 \\
    \frac{1}{\sqrt{2}} e^{-i\phi} c_{-1} - c_0 + \frac{1}{\sqrt{2}} e^{i\phi} c_1 = 0 \\
    \frac{1}{\sqrt{2}} e^{-i\phi} c_{0} + (\cos{\theta} - 1) c_{1} = 0
.\end{gather}
Notice that we have three equations and three unknowns here.
Per usual, though, they are redundant (i.e. linearly dependent).
Let us choose our undeterimined coefficient to be $c_0$, giving
\begin{eqnarray}
    \ket{\psi} = -c_0 \Big[ \frac{e^{i\phi}}{\sqrt{2}(1 + \cos{\theta})} \ket{-1} + \ket{0} + \frac{e^{-i\phi}}{\sqrt{2}(1 - \cos{\theta})} \ket{1} \Big]
.\end{eqnarray}
Finally, we fix $c_0$ via normalization:
\begin{eqnarray}
    |c_0|^2 \Big[ \frac{1}{2(1 + \cos{\theta})^2} + 1 + \frac{1}{2(1 - \cos{\theta})^2} \Big] = |c_0|^2 \frac{1 + \cos^2{\theta} + 2 \sin^{4}{\theta}}{\sin^{4}{\theta}} = 1
.\end{eqnarray}
Thus,
\begin{eqnarray}
    \eqbox{ \ket{\psi} = \frac{1}{\sqrt{1 + \cos^2{\theta} + 2 \sin^{4}{\theta}}} \Big[ \frac{1 - \cos{\theta}}{\sqrt{2}} e^{i\phi} \ket{-1} + \sin^2{\theta} \ket{0} + \frac{1 + \cos{\theta}}{\sqrt{2}} e^{-i\phi} \ket{1} \Big] }
.\end{eqnarray}

Notice that the result above is consistent with the following sanity check.
If $\vu*{n} = \vu*{z}$, then $\vu*{n} \vdot \va*{J} = J_3$ and $\ket{\psi} = \ket{1}$, and the above formula yields this result as needed (with an irrelevant phase factor that we can simply toss away in this special case).


}


\prob{2 -- Chapter 11 \# 4}{

The eigenstates of the orbital angular momentum satisfy the eigenvalue equations
\begin{eqnarray}
    \hat{\mathbf{L}}^2 \ket{\psi_{lm}} = l (l+1) \hbar^2 \ket{\psi_{lm}},\quad \hat{L}_{z} \ket{\psi_{lm}} = m \hbar \ket{\psi_{lm}}
,\end{eqnarray}
where $l = 0,1,2,\ldots$ and $m = -l,\ldots,l$.
In the r-representation, the corresponding wave functions are the spherical harmonics,
\begin{eqnarray}
    Y_{lm}(\theta,\phi) = \bra{\va*{r}}\ket{\psi_{lm}}
.\end{eqnarray}

\vspace{1em}

(a) Using the expression of $L_{z}$ as a differential operator, solve the differential equation implied by the (second) eigenvalue equation above to show that the $\phi$ dependence of $Y_{lm}(\theta,\phi)$ is proportional to $e^{im\phi}$.

(b) Using the condition $L_{+} \ket{\psi_{ll}} = 0$ and the fact that $Y_{ll}(\theta,\phi) = F_{l}(\theta) e^{i l \phi}$, show that
\begin{eqnarray}
    Y_{l,l}(\theta,\phi) = c_{l} \sin^{l}{\theta} e^{il\phi}
,\end{eqnarray}
where $c_{l}$ is a normalization factor.

(c) Assume that the orbital angular momentum could also take on half-integer values, say $l = 1/2$.
Construct the ``spherical harmonic $Y_{1/2,-1/2}(\theta,\phi)''$ by (i) applying the lowering operator $L_{-}$ to $Y_{1/2,1/2}(\theta,\phi) \propto \sin^{1/2}{\theta} e^{i\phi/2}$ and (ii) by solving the differential equation resulting from $L_{-} Y_{1/2,-1/2}(\theta,\phi) = 0$.
Show that the two procedures are problematic and yield contradictory results.
Thus, half-integer values cannot occur for the orbital angular momentum operator.

}

\sol{

(a) The eigenvalue equation for the $z$-component of the orbital angular momentum has the form
\begin{eqnarray}
    L_{z} f(\phi) = - i \hbar \pdv{f(\phi)}{\phi} = \hbar m f(\phi)
\end{eqnarray}
in coordinate space, which has the solution we all know and love:
\begin{eqnarray}
    f(\phi) \propto e^{i m \phi}
.\end{eqnarray}
Note that the spherical harmonics are factorized into azimuthal and polar functions as $Y_{lm}(\theta,\phi) = c_{lm} P(\theta) f(\phi)$, where $c_{lm}$ is a normalizing constant.

(b) For fixed $l$, we can construct the spherical harmonics using the properties of the raising and lowering operators, which take the form
\begin{align}
    L_{\pm} &= \hbar e^{\pm i \phi} \Big[ \pm \pdv{\theta} + i \cot{\theta} \pdv{\phi} \Big]
.\end{align}
If we apply $L_{+}$ on $Y_{ll}$, we get zero:
\begin{eqnarray}
    \hbar e^{i \phi} \Big( \pdv{\theta} + i \cot{\theta} \pdv{\phi} \Big) P(\theta) e^{i l \phi} = \hbar e^{i(l+1)\phi} \Big( \dv{P(\theta)}{\theta} - l \cot{\theta} P(\theta) \Big) = 0
.\end{eqnarray}
We thus have a first-order separable differential equation for $\theta$, which has solution
\begin{eqnarray}
    P(\theta) \propto \sin^{l}{\theta}
,\end{eqnarray}
meaning that
\begin{eqnarray}
    Y_{ll}(\theta,\phi) = c_{l} \sin^{l}{\theta} e^{i l \phi}
.\end{eqnarray}

(c) Suppose that $l = 1/2$, that is that $l$ can take on half-integer values.
Then
\begin{eqnarray}
    Y_{1/2,1/2} = c_{l} \sin^{1/2}{\theta} e^{i\phi/2}
.\end{eqnarray}
Thus,
\begin{eqnarray}
\begin{aligned}
    Y_{1/2,-1/2} &\propto e^{-i\phi} \Big( -\pdv{\theta} + i \cot{\theta} \pdv{\phi} \Big) \sqrt{ \sin{\theta} } e^{i\phi/2} \\
                 &= e^{-i\phi/2} \Big( -\frac{1}{2} \cot{\theta} \sqrt{\sin{\theta}} - \frac{1}{2} \cot{\theta} \sqrt{\sin{\theta}}  \Big) \\
                 &= - \cot{\theta} \sqrt{\sin{\theta}} e^{-i\phi/2}
.\end{aligned}
\end{eqnarray}

Next, let's find the form of $Y_{1/2,-1/2}$ from $L_{-} Y_{1/2,-1/2} = 0$:
\begin{eqnarray}
    -\dv{P}{\theta} + \frac{1}{2} \cot{\theta} P = 0
.\end{eqnarray}
Thus we find from this that
\begin{eqnarray}
    Y_{1/2,-1/2} \propto \sqrt{\sin{\theta}} e^{-i\phi/2}
,\end{eqnarray}
which contradicts our findings from the method above, meaning that in fact $l$ cannot be $1/2$.

}


\prob{1 -- Chapter 11 \# 6}{

Let $a_r$ and $a_{r}^{\dagger}$ with $r = 1,2$ be the annihilation and creation operators of a two-dimensional harmonic oscillator, satisfying
\begin{eqnarray}
    [a_{r},a_{s}] = [a_{r}^{\dagger},a_{s}^{\dagger}] = 0, \quad [a_{r},a_{s}^{\dagger}] = \delta_{rs}
.\end{eqnarray}
We define
\begin{eqnarray}
    S = \frac{1}{2} \Big( a_1^{\dagger} a_1 + a_2^{\dagger} a_2 \Big)
,\end{eqnarray}
and
\begin{eqnarray}
    \hat{J}_1 = \frac{1}{2} \Big( a_2^{\dagger}a_1 + a_1^{\dagger} a_2 \Big), \quad \hat{J}_{2} = \frac{i}{2} \Big( a_2^{\dagger} a_1 - a_1^{\dagger} a_2 \Big), \quad \hat{J}_3 = \frac{1}{2} \Big( a_1^{\dagger} a_1 - a_2^{\dagger} a_2 \Big)
,\end{eqnarray}
and the $\hat{J}_{i}$ may be considered as the Cartesian components of a certain vector operator.

\vspace{1em}

(a) Show that the components of $\va*{J}$ as defined above satisfy the commutation relations characteristic of an angular momentum up to factors of $\hbar$, that is,
\begin{eqnarray}
    [\hat{J}_{i},\hat{J}_{j}] = i \epsilon_{ijk} \hat{J}_{k}
\end{eqnarray}
and that
\begin{eqnarray}
    \va*{J}^2 = S (S + \id)
.\end{eqnarray}

(b) Hereafter $\va*{J}$ will be considered to be the angular momentum of the system.
We denote the eigenvalues of $\va*{J}^2$ and $J_3$ by $j(j+1)$ and $m$, respectively.
Show that $\va*{J}^2$ and $J_3$ form a complete set of commuting observables, and that $j$ may take all integral or half-integral values $\geq 0$.

(c) Show that the states
\begin{eqnarray}
    \frac{1}{\sqrt{(j+m)!(j-m)!}} (a_1^{\dagger})^{j+m} (a_2^{\dagger})^{j-m} \ket{0,0}
\end{eqnarray}
form a basis of common eigenstates of $\va*{J}^2$ and $J_3$.

}

\sol{

(a) We will determine the commutation relations via brute force:
\begin{gather}
\begin{aligned} 
    [J_1,J_2] &= \frac{i}{4} \Big[ \Big( a_2^{\dagger} a_1 + a_1^{\dagger} a_2 \Big) \Big( a_2^{\dagger} a_1 - a_1^{\dagger} a_2 \Big) - \Big( a_2^{\dagger} a_1 - a_1^{\dagger} a_2 \Big) \Big( a_2^{\dagger} a_1 + a_1^{\dagger} a_2 \Big) \Big] \\
                &= \frac{i}{2} \Big[ a_1^{\dagger} a_1 a_2 a_2^{\dagger} - a_2^{\dagger} a_2 a_1 a_1^{\dagger} \Big] \\
                &= \frac{i}{2} \Big[ a_1^{\dagger} a_1 - a_2^{\dagger} a_2   \Big] = i J_3
\end{aligned}
\\
\begin{aligned}
    [J_2,J_3] &= \frac{i}{4} \Big[ \Big( a_2^{\dagger} a_1 - a_1^{\dagger} a_2 \Big) \Big( a_1^{\dagger} a_1 - a_2^{\dagger} a_2 \Big) - \Big( a_1^{\dagger} a_1 - a_2^{\dagger} a_2 \Big) \Big( a_2^{\dagger} a_1 - a_1^{\dagger} a_2 \Big) \Big] \\
              &= \frac{i}{2} \Big[ a_2^{\dagger} a_1 + a_1^{\dagger} a_2 + (a_2^{\dagger} a_2 a_2^{\dagger} - a_2^{\dagger} a_2^{\dagger} a_2) a_1 \Big] = i J_1
\end{aligned}
\\
\begin{aligned}
    [J_3,J_1] &= \frac{1}{4} \Big[ \Big( a_1^{\dagger} a_1 - a_2^{\dagger} a_2 \Big) \Big( a_2^{\dagger} a_1 + a_1^{\dagger}a_2 \Big) - \Big( a_2^{\dagger} a_1 + a_1^{\dagger} a_2 \Big) \Big( a_1^{\dagger} a_1 - a_2^{\dagger} a_2 \Big) \Big] \\
              &= \frac{1}{2} \Big[ a_1^{\dagger} a_2 - a_2^{\dagger} a_1 \Big] = i \frac{i}{2} \Big[ a_2^{\dagger} a_1 - a_1^{\dagger} a_2 \Big] = i J_2
.\end{aligned}
\end{gather}
Finally, we bake in the antisymmetric nature of the commutation relations, we can write generically $[J_{i},J_{j}] = i \epsilon_{ijk} J_{k}$.

Next, we find
\begin{eqnarray}
\begin{aligned}
    \va*{J}^2 &= J_1^2 + J_2^2 + J_3^2 \\
              &= \frac{1}{4} \Big[ a_1 a_1 a_2^{\dagger} a_2^{\dagger} + a_1^{\dagger} a_1^{\dagger} a_2 a_2 + a_1 a_1^{\dagger} a_2^{\dagger} a_2 + a_1^{\dagger} a_1 a_2 a_2^{\dagger} \\
              &\phantom{\frac{1}{4}\Big[} - a_1 a_1 a_2^{\dagger} a_2^{\dagger} - a_1^{\dagger} a_1^{\dagger} a_2 a_2 + a_1 a_1^{\dagger} a_2^{\dagger} a_2 + a_1^{\dagger} a_1 a_2 a_2^{\dagger} \\
              &\phantom{\frac{1}{4} \Big[} + a_1^{\dagger} a_1 a_1^{\dagger} a_1 + a_2^{\dagger} a_2 a_2^{\dagger} a_2 - 2 a_1^{\dagger} a_1 a_2^{\dagger} a_2 \Big] \\
              &= \frac{1}{4} \Big[ 2 a_1 a_1^{\dagger} a_2^{\dagger} a_2 + 2 a_1^{\dagger} a_1 a_2 a_2^{\dagger} + a_1^{\dagger} a_1 a_1^{\dagger} a_1 + a_2^{\dagger} a_2 a_2^{\dagger} a_2 - 2a_1^{\dagger} a_1 a_2^{\dagger} a_2 \Big] \\
              &= \frac{1}{4} \Big[ 2a_2^{\dagger}a_2 + 2 a_1^{\dagger} a_1 a_2^{\dagger} a_2 + 2 a_1^{\dagger} a_1 + (a_1^{\dagger} a_1)^2 + (a_2^{\dagger} a_2)^2 \Big] \\
              &= \Big[ S + S^2 \Big] = S (S + \id)
.\end{aligned}
\end{eqnarray}

(b) Let us now consider $\va*{J}$ to be the angular momentum of the system, denoting the eigenvalue of $J_3$ by $m$, $S$ by $j$ and therefore $\va*{J}^2$ by $j(j+1)$.
It is clear that $\va*{J}^2$ and $J_3$ form a set of commuting observables (since they obey the fundamental commutation relations of angular momentum operators sans a factor $\hbar$, which is irrelevant for the commutator $[\va*{J}^2,J_3]$).
Thus, we can find a common orthonormal eigenbasis of $\va*{J}^2$ and $J_3$.
The next question is whether any state of this two-dimensional harmonic oscillator can be expanded in the common eigenbasis of $\va*{J}^2$ and $J_3$, or equivalently, if
\begin{eqnarray}
    \sum_{j,m} \ket{j,m} \bra{j,m} \stackrel{?}{=} \id
.\end{eqnarray}

We know that the eigenstates of the Hamiltonians for particles 1 and 2, $H_1$ and $H_2$ respectively, form a complete set of commuting observables, or equivalently, the number operators $N_1$ and $N_2$ form a complete set of commuting observables since the eigenstates of $N_1$ and $H_1$ and also $N_2$ and $H_2$ are the same, respectively.
Thus,
\begin{eqnarray}
    \sum_{n_1,n_2} \ket{n_1,n_2} \bra{n_1,n_2} = \id
.\end{eqnarray}
Now, observe that the eigenstates $\ket{n_1,n_2}$ are also eigenstates of $\va*{J}^2$ and $J_3$:
\begin{eqnarray}
\begin{aligned}
    \va*{J}^2\ket{n_1,n_2} &= S(S + 1) \ket{n_1,n_2} = \frac{1}{2}(n_1 + n_2) \Big[ \frac{1}{2}(n_1 + n_2) + 1 \Big] \ket{n_1,n_2} \\
    J_3 \ket{n_1,n_2} &= \frac{1}{2} (n_1 - n_2) \ket{ n_1,n_2}
.\end{aligned}
\end{eqnarray}
That is, $\ket{n_1,n_2}$ corresponds to $j = (n_1+n_2)/2$ and $m = (n_1-n_2)/2$.
Hence, $j$ can take on any half-integer or integral value, and $m$ can range from $-j$ to $j$ in integer steps, and thus the completeness statement for the states $\ket{n_1,n_2}$ is also a completeness statement for $\ket{j,m}$.

(c) From our work in part (b), we found $n_1 = j+m$ and $n_2 = j-m$.
We know that
\begin{eqnarray}
    \ket{n_1,n_2} = \frac{1}{\sqrt{n_1! n_2!}} (a_1^{\dagger})^{n_1} (a_2^{\dagger})^{n_2} \ket{0,0}
,\end{eqnarray}
so
\begin{eqnarray}
    \ket{n_1,n_2} = \frac{1}{\sqrt{(j+m)!(j-m)!}} (a_1^{\dagger})^{j+m} (a_2^{\dagger})^{j-m} \ket{0,0}
.\end{eqnarray}
These states forms a basis of common eigenstates of $\va*{J}^2$ and $J_3$ since any combination of $n_1$ and $n_2$ specifies a corresponding unique combination of $j$ and $m$.

}


%\prob{4 -- Chapter 11 \# 12 (\textit{optional})}{}

%\sol{}


\end{document}
