\def\duedate{\today}
\def\HWnum{5}
\documentclass[10pt,a4paper]{book}

% custom section formatting
\usepackage{titlesec}
\titleformat{\chapter}[display]
{\normalfont\Large\filcenter\sffamily}
{\titlerule[1pt]%
\vspace{1pt}%
\titlerule
\vspace{1pc}%
\LARGE\MakeUppercase{\chaptertitlename} \thechapter}
{1pc}
{\titlerule
\vspace{1pc}%
\Huge}

% appendix handling
\usepackage[toc,page]{appendix}
    
% encoding for file and font
\usepackage[utf8]{inputenc}
\usepackage[T1]{fontenc}

% math formatting/tools
\usepackage{amsmath}
\usepackage{amssymb}
\usepackage{mathtools}
\usepackage[arrowdel]{physics}

% unit formatting
\usepackage{siunitx}
\AtBeginDocument{\RenewCommandCopy\qty\SI}

% figure formatting/tools
\usepackage{graphicx}
\usepackage{float}
\usepackage{subcaption}
\usepackage{multirow}
\usepackage{import}
\usepackage{pdfpages}
\usepackage{transparent}
\usepackage{currfile}

\NewDocumentCommand\incfig{O{1} m}{
    \def\svgwidth{#1\textwidth}
    \import{./Figures/\currfiledir}{#2.pdf_tex}
}

\newcommand{\bef}{\begin{figure}[h!tb]\centering}
\newcommand{\eef}{\end{figure}}

\newcommand{\bet}{\begin{table}[h!tb]\centering}
\newcommand{\eet}{\end{table}}

% hyperlink references 
\usepackage{hyperref}
\hypersetup{
    colorlinks=true,
    linkcolor=blue,
    filecolor=magenta,
    urlcolor=cyan,
    pdftitle={Physics 1 Notes},
    pdfauthor={Richard Whitehill},
    pdfpagemode=FullScreen
}
\urlstyle{same}

\newcommand{\eref}[1]{Eq.~(\ref{eq:#1})}
\newcommand{\erefs}[2]{Eqs.~(\ref{eq:#1})--(\ref{eq:#2})}

\newcommand{\fref}[1]{Fig.~(\ref{fig:#1})}
\newcommand{\frefs}[2]{Fig.~(\ref{fig:#1})--(\ref{fig:#2})}

\newcommand{\aref}[1]{Appendix~(\ref{app:#1})}
\newcommand{\sref}[1]{Section~(\ref{sec:#1})}
\newcommand{\srefs}[2]{Sections~(\ref{sec:#1})-(\ref{sec:#2})}

\newcommand{\tref}[1]{Table~(\ref{tab:#1})}
\newcommand{\trefs}[2]{Table~(\ref{tab:#1})--(\ref{tab:#2})}

% tcolorbox formatting/definitions
\usepackage[most]{tcolorbox}
\usepackage{xcolor}
\usepackage{xifthen}
\usepackage{parskip}

\definecolor{peach}{rgb}{1.0,0.8,0.64}

\DeclareTColorBox[auto counter, number within=chapter]{defbox}{O{}}{
    enhanced,
    boxrule=0pt,
    frame hidden,
    borderline west={4pt}{0pt}{green!50!black},
    colback=green!5,
    before upper=\textbf{Definition \thetcbcounter \ifthenelse{\isempty{#1}}{}{: #1} \\ },
    sharp corners
}

\newcommand*{\eqbox}{\tcboxmath[
    enhanced,
    colback=black!10!white,
    colframe=black,
    sharp corners,
    size=fbox,
    boxsep=8pt,
    boxrule=1pt
]}

\newtcolorbox[auto counter, number within=chapter]{exbox}{
    parbox=false,
    breakable,
    enhanced,
    sharp corners,
    boxrule=1pt,
    colback=white,
    colframe=black,
    before upper= \textbf{Example \thetcbcounter:}\,,
    before lower= \textbf{Solution:}\,,
    segmentation hidden
}

\newtcolorbox{resbox}{
    enhanced,
    colback=black!10!white,
    colframe=black,
    boxrule=1pt,
    boxsep=0pt,
    top=2pt,
    ams nodisplayskip,
    sharp corners
}


\begin{document}

\prob{1 -- Chapter 3 \# 3}{

In the $r$-representation, the Schr\"{o}dinger equation for a particle under the influence of a potential $V(\va*{r})$ is given by
\begin{eqnarray}
    i\hbar \pdv{t} \Psi(\va*{r},t) = \Bigg[ -\frac{\hbar^2}{2m} \laplacian + V(\va*{r}) \Bigg] \Psi(\va*{r},t)
.\end{eqnarray}

(a) By using the properties of the Fourier transform show that the Schrodinger equation in the $p$-representation can be wrriten as
\begin{eqnarray}
    \Big( i\hbar \pdv{t} - \frac{\va*{p}^2}{2m} \Big) \tilde{\Psi}(\va*{p},t) = \frac{1}{(2 \pi \hbar)^{3/2}} \int \dd[3]{\va*{p}'} \tilde{V}(\va*{p} - \va*{p}') \tilde{\Psi}(\va*{p}',t)
,\end{eqnarray}
where
\begin{eqnarray}
    \tilde{\Psi}(\va*{p},t) = \int \frac{\dd[3]{\va*{r}}}{(2 \pi \hbar)^{3/2}} e^{-i \va*{p} \cdot \va*{r} / \hbar} \Psi(\va*{r},t), \quad \tilde{\va*{p}} = \int \frac{\dd[3]{\va*{r}}}{(2 \pi \hbar)^{3/2}} e^{-i \va*{p} \cdot \va*{r} / \hbar} V(\va*{r})
.\end{eqnarray}
Assume that $V(\va*{r})$ vanishes as $|\va*{r}| \rightarrow \infty$ fast enough for its Fourier transform to exist.
Note that in the $p$-space Schr\"{o}dinger equation as given above, the term with the potential reduces to a convolution product involving the Fourier transforms $\tilde{V}(\va*{p})$ and $\tilde{\Psi}(\va*{p},t)$.

(b) Show that the $p$-space Schr\"{o}dinger equation can also be written as
\begin{eqnarray}
    i \hbar \pdv{t} \tilde{\Psi}(\va*{p},t) = \Big[ \frac{\va*{p}^2}{2m} + V(i\hbar \grad_{p}) \Big] \tilde{\Psi}(\va*{p},t)
\end{eqnarray}
as given in the notes.

}

\sol{

(a) We take the Fourier transform of the S.E.:
\begin{eqnarray}
    \int \frac{\dd[3]{\va*{r}}}{(2 \pi \hbar)^{3/2}} e^{-i \va*{p} \cdot \va*{r} / \hbar} \Bigg\{ i\hbar \pdv{\Psi(\va*{r},t)}{t} - \Bigg[ - \frac{\hbar^2}{2m} \laplacian \Psi(\va*{r},t) + V(\va*{r}) \Psi(\va*{r},t) \Bigg] \Bigg\} = 0
.\end{eqnarray}
The first two terms are quite simple to determine.
For the first, notice that there is not time-dependence in the integration except for the wave function, allowing us to pull it out of the integral, and for the second term, we can integrate by parts twice, giving
\begin{eqnarray}
    \pdv{\tilde{\Psi}(\va*{p},t)}{t} - \Bigg[ - \frac{\va*{p}^2}{2m} \tilde{\Psi}(\va*{p},t) + \int \frac{\dd[3]{\va*{r}}}{(2 \pi \hbar)^{3/2}} e^{-i \va*{p} \cdot \va*{r} /\hbar} V(\va*{r}) \Psi(\va*{r},t) \Bigg] = 0
.\end{eqnarray}
All that remains now is to evaluate the last term, which is the Fourier transform of a product of functions.
Note that we know that the fourier transform of a convolution of functions $f$ and $g$ is just the product of the individual fourier transforms of $f$ and $g$.
Because of the symmetric nature of the Fourier transform, we may expect that a similar result for the Fourier transform of a product of functions.
Namely, the fourier transform of this product is just a convolution in the conjugate space, which we now prove.
Let $f,g \in L^2(\reals^3)$.
Thus,
\begin{eqnarray}
    \begin{aligned}
        \int &\frac{\dd[3]{\va*{r}}}{(2 \pi \hbar)^{3/2}} e^{-i \va*{p} \cdot \va*{r} / \hbar} f(\va*{r}) g(\va*{r}) \\
                                                         &= \int \frac{\dd[3]{\va*{r}}}{(2 \pi \hbar)^{3/2}} e^{-i \va*{p} \cdot \va*{r} / \hbar} \Bigg[ \int \frac{\dd[3]{\va*{q}}}{(2 \pi \hbar)^{3/2}} e^{i \va*{q} \cdot \va*{r} / \hbar} \tilde{f}(\va*{q}) \Bigg] \Bigg[ \int \frac{\dd[3]{\va*{l}}}{(2 \pi \hbar)^{3/2}} e^{i \va*{l} \cdot \va*{r} / \hbar} \tilde{g}(\va*{l}) \Bigg] \\
                                                         &= \int \frac{\dd[3]{\va*{q}} \dd[3]{\va*{l}}}{(2 \pi \hbar)^{3/2}} \tilde{f}(\va*{q}) \tilde{g}(\va*{l}) \underbrace{\Bigg[ \int \frac{\dd[3]{\va*{r}}}{(2 \pi \hbar)^{3/2}} \frac{e^{i(\va*{q} + \va*{l} - \va*{p}) \cdot \va*{r} / \hbar}}{(2 \pi \hbar)^{3}} \Bigg]}_{\delta^{(3)}(\va*{l} - [\va*{p} - \va*{q}])} \\
                                                         &= \int \frac{\dd[3]{\va*{q}}}{(2 \pi \hbar)^{3/2}} \tilde{f}(\va*{q}) \tilde{g}(\va*{p} - \va*{q}) = \tilde{f} \star \tilde{g}
    ,\end{aligned}
\end{eqnarray}
which is the result we anticipated.
Note that the convolution is commutative in $f$ and $g$ (i.e. $\tilde{f} \star \tilde{g} = \tilde{g} \star \tilde{f}$).
Hence, we have
\begin{eqnarray}
    \eqbox{ \pdv{\tilde{\Psi}(\va*{p},t)}{t} = -\frac{\va*{p}^2}{2m} \tilde{\Psi}(\va*{p},t) + \tilde{\Psi} \star \tilde{V} }
.\end{eqnarray}

(b) One way of arriving at this result is knowing that the S.E. is true in any particular basis.
That is, for a state $\ket{\psi}$, we have
\begin{eqnarray}
    i \hbar \pdv{\ket{\psi}}{t} = \Big[ \frac{p^2}{2m} + V(x) \Big] \ket{\psi}
.\end{eqnarray}
If we project this onto the $p$-basis, then we immediately have
\begin{eqnarray}
    \eqbox{ i \hbar \pdv{\tilde{\psi}}{t} = \Big[ \frac{\va*{p}^2}{2m} + V(i \hbar \grad_{p}) \Big] \tilde{\psi} }
.\end{eqnarray}

}


\prob{2 -- Chapter 4 \# 2}{

This problem deals with the solution of the S.E. in momentum space.

(a) Solve directly in momentum space for the bound-state energy and wave function in an attractive $\delta$-function potential, given in coordinate space by $V(x) = -V_0 \delta(x)$ with $V_0 > 0$.

(b) From the momentum space wave function so obtained, calculate the average kinetic energy of the particle.
Repeat the calculation of this average kinetic energy but in coordinate space, using the wave function you derived in the previous problem.
Note that the first derivative of the coordinate-space bound-state wave function is discontinuous at $x = 0$.

}

\sol{

(a) If our potential $V(x) = -V_0 \delta(x)$, then
\begin{eqnarray}
    \tilde{V}(p) = \int \frac{\dd{x}}{\sqrt{2 \pi \hbar}} e^{-i p x /\hbar} \Big[ - V_0 \delta(x) \Big] = -\frac{V_0}{\sqrt{2 \pi \hbar}}
.\end{eqnarray}
Note that we could have repeated the process from part (a) on the time-independent S.E. to arrive at
\begin{eqnarray}
    \Big[ E - \frac{p^2}{2m} \Big] \tilde{\psi}(p) = - \frac{V_0}{\sqrt{ 2 \pi \hbar}} \int \frac{\dd{p}}{\sqrt{2 \pi \hbar}} \tilde{\psi}(p)
.\end{eqnarray}
Notice that the r.h.s is a constant with respect to $p$, meaning
\begin{eqnarray}
    \tilde{\psi}(p) = \frac{\mathcal{N}}{|E| + p^2/2m}
,\end{eqnarray}
where we have used the fact that $E < 0$ for a bound state.
The energy $E$ is determined using 
\begin{gather}
    \frac{V_0}{2 \pi \hbar} \int_{-\infty}^{\infty} \frac{1}{|E| + p^2/2m} = 1 \Rightarrow \sqrt{\frac{2 m}{E}} = \frac{2 \hbar}{V_0} \\
    E = -\frac{V_0^2}{4 \hbar^2} (2m) = -\frac{m V_0^2}{2 \hbar^2}
.\end{gather}
Thus,
\begin{eqnarray}
    \tilde{\psi}(p) = \frac{\mathcal{N}}{p_0^2 + p^2}
,\end{eqnarray}
where we have defined $p_0^2 = 2m E$.
Normalizing gives
\begin{eqnarray}
    \eqbox{ \tilde{\psi}(p) = \sqrt{\frac{2 p_0^3}{\pi}} \frac{1}{p_0^2 + p^2} }
,\end{eqnarray}
which agrees with the result found in a previous homework where we explicitly took the Fourier transform of the coordinate space result.

(b) The average kinetic energy of a particle is given by
\begin{eqnarray}
    \eqbox{ \expval{T} = \frac{1}{2m} \expval{p^2} = \frac{1}{2m} \frac{2 p_0^3}{\pi} \int_{-\infty}^{\infty} \frac{p^2}{(p_0^2 + p^2)^2} \dd{p} = \frac{p_0^3}{\pi m} \frac{\pi}{2 p_0} = \frac{p_0^2}{2m} = E }
.\end{eqnarray}

We can verify this result by repeating the calculation in coordinate space:
\begin{eqnarray}
    \begin{aligned}
        \expval{T} &= \frac{p_0}{\hbar} \int_{-\infty}^{\infty} e^{-p_0 |x|/\hbar} \Big( -\frac{\hbar^2}{2m} \dv[2]{x} e^{- p_0 |x| /\hbar} \Big)  \dd{x} \\
                   &= -\frac{p_0 \hbar}{2m} \Bigg[ \int_{-\infty}^{0} e^{p_0 x /\hbar} \dv[2]{x} e^{p_0 x /\hbar} \dd{x} + \int_{0}^{\infty} e^{-p_0 x /\hbar} \dv[2]{x} e^{-p_0 x /\hbar} \Bigg] \\
                   &= -\frac{p_0 \hbar}{m} \Big( -\frac{p_0}{\hbar} \Big)^2 \int_{0}^{\infty} e^{-2p_0 x / \hbar} \dd{x} = \frac{p_0^3}{m \hbar} \frac{\hbar}{2 p_0} = \frac{p_0^2}{2m} = E
    .\end{aligned}
\end{eqnarray}

}


\prob{3 -- Chapter 5 \# 4}{

Consider a particle in one-dimension under the influence of the even potential given by
\begin{eqnarray}
   V(x) = \begin{cases}
       V_0 & |x| \leq a \\
       0 & |x| > a
   ,\end{cases}
\end{eqnarray}
where $V_0 > 0$.
Assume that the particle's energy $E$ is in the range $0 < E < V_0$.
Write the general solution as 
\begin{eqnarray}
   \psi(x) = \begin{cases}
       Ae^{ikx} + Be^{-ikx} & x < -a \\
       C e^{-\kappa x} + D e^{\kappa x} & |x| < a \\
       Fe^{ikx} + Ge^{-ikx} & x > a
   ,\end{cases}
\end{eqnarray}
where $k = \sqrt{\epsilon}$, $\kappa = \sqrt{v_0 - \epsilon}$, $\epsilon = 2mE/\hbar^2$, and $v_0 = 2mV_0/\hbar^2$.

(a) By imposing the appropriate boundary conditions at $x = \pm a$, show that the coefficients $A$ and $B$ are related to the coefficients $F$ and $G$ in the following way
\begin{eqnarray}
    \begin{pmatrix}
    A \\ B
    \end{pmatrix}
    = 
    \begin{bmatrix}
        (\cosh{2 \kappa a} + i(\alpha/2) \sinh{2 \kappa a})e^{2ika} & i(\beta/2) \sinh{2 \kappa a} \\

        -i(\beta/2) \sinh{2 \kappa a} & (\cosh{2 \kappa a} - i(\alpha/2) \sinh{2 \kappa a})e^{-2ika} \\
    \end{bmatrix}
    \begin{pmatrix}
    F \\ G
    \end{pmatrix}
,\end{eqnarray}
where
\begin{eqnarray}
   \alpha = \frac{\kappa}{k} - \frac{k}{\kappa},~ \beta = \frac{\kappa}{k} + \frac{k}{\kappa},~\beta^2 - \alpha^2 = 4
.\end{eqnarray}

(b) From the matrix relation obtained in part (a) above, show that the transmission coefficient $T$ for a high and wide barrier with $\kappa a \gg 1$ is approximately given by 
\begin{eqnarray}
    T \approx 16 e^{-4 \kappa a} \Big( \frac{\kappa k}{k^2 + \kappa^2} \Big)^2
.\end{eqnarray}

(c) Consider the case of a very narrow but very high barrier such that $a V_0$ is finite.
Assume $V_0 \gg E$ $\kappa \gg k$, and $\kappa a \ll 1$, but $\kappa^2 a$ is finite.
Show that the transmission coefficient under these conditions is the same as that obtained in the repulsive $\delta$-function potential $V(x) = 2aV_0\delta(x)$, namely
\begin{eqnarray}
    T = \frac{\epsilon}{\epsilon + (2mV_0a/\hbar^2)^2}
.\end{eqnarray}


}

\sol{

(a) The S.E. under this potential reads
\begin{align}
    \psi''(x) = \begin{cases}
        [v_0 - \epsilon] \psi(x) & |x| < a \\
        -\epsilon \psi(x) & |x| > a
    ,\end{cases}
\end{align}
where $v_0 = 2mV_0/\hbar^2$ and $\epsilon = 2mE/\hbar^2$.
This has solution
\begin{eqnarray}
    \psi(x) = \begin{cases}
        A e^{ikx} + Be^{-ikx} & x < a \\
        C e^{-\kappa x} + D e^{\kappa x} & |x| < a \\
        F e^{ikx} + G e^{-ikx} & x > a
    ,\end{cases}
\end{eqnarray}
where $k^2 = \epsilon$ and $\kappa^2 = v_0 - \epsilon$
We have the following boundary conditions
\begin{align}
    \psi(-a^{-}) = \psi(-a^{+}) &\Rightarrow A e^{-ika} + B e^{ika} = C e^{\kappa a} + D e^{-\kappa a} \\
    \psi'(-a^{-}) = \psi'(-a^{+}) &\Rightarrow ik(A e^{-ika} - B e^{ika}) = \kappa (-C e^{\kappa a} + De^{-\kappa a}) \\
    \psi(a^{-}) = \psi(a^{+}) &\Rightarrow Ce^{-\kappa a} + D e^{\kappa a} = F e^{ika} + Ge^{-ika} \\
    \psi'(a^{-}) = \psi'(a^{+}) &\Rightarrow \kappa ( -C e^{-\kappa a} + D e^{\kappa a} ) = ik ( F e^{ika} - G e^{-ika} )
.\end{align}
Let us solve for $C$ and $D$ in terms of $A$ and $B$:
\begin{align}
C e^{-\kappa a} &= \frac{e^{-ika} (\kappa - ik)}{2 \kappa e^{2\kappa a}} A + \frac{e^{ika} (\kappa + ik)}{2 \kappa e^{2 \kappa a}} B \\
    De^{\kappa a} &= \frac{e^{-ika} (\kappa + ik)}{2 \kappa e^{-2\kappa a}} A + \frac{e^{ika} (\kappa - ik)}{2 \kappa e^{-2 \kappa a}} B
.\end{align}
The last two equations from the BCs are then
\begin{align}
    \frac{e^{-ika}}{2\kappa} \Bigg[ \frac{\kappa - ik}{e^{2 \kappa a}} + \frac{\kappa + ik}{e^{-2\kappa a}} \Bigg] A + \frac{e^{ika}}{2\kappa} \Bigg[ \frac{\kappa - ik}{e^{-2 \kappa a}} + \frac{\kappa + ik}{e^{2 \kappa a}} \Bigg] B &= F e^{ika} + G e^{-ika} \\
    \frac{e^{-ika}}{2} \Bigg[ \frac{-\kappa + ik}{e^{2 \kappa a}} + \frac{\kappa + ik}{e^{-2\kappa a}} \Bigg] A + \frac{e^{ika}}{2} \Bigg[ \frac{\kappa - ik}{e^{-2 \kappa a}} - \frac{\kappa + ik}{e^{2 \kappa a}} \Bigg] B &= ik( F e^{ika} - G e^{-ika} )
.\end{align}
Observe that these equations reduce to
\begin{align}
    e^{-ika} \Big[ \cosh(2 \kappa a) + i \frac{k}{\kappa} \sinh(2 \kappa a) \Big] A + e^{ika} \Big[ \cosh(2 \kappa a) &- i \frac{k}{\kappa} \sinh(2 \kappa a) \Big] B \nonumber \\
    &= Fe^{ika} + G e^{-ika} \\
    e^{-ika} \Big[ \cosh(2 \kappa a) - i \frac{\kappa}{k} \sinh(2 \kappa a) \Big] A - e^{ika} \Big[ \cosh(2 \kappa a) &+ i \frac{\kappa}{k} \sinh(2 \kappa a) \Big] B \nonumber \\
    &= Fe^{ika} - Ge^{-ika}
.\end{align}
Solving for $F$ and $G$ gives
\begin{align}
    F &= \Big[ \cosh(2 \kappa a) - i (\alpha/2) \sinh(2 \kappa a) \Big] e^{-2ika} A - i (\beta/2) \sinh(2 \kappa a) B \\
    G &= i (\beta/2) \sinh(2 \kappa a) A + \Big[ \cosh(2 \kappa a) + i(\alpha/2) \sinh(2 \kappa a) \Big] e^{2ika} B
,\end{align}
and rewriting in matrix form
\begin{eqnarray}
    \begin{pmatrix}
    F \\ G
    \end{pmatrix}
    = \begin{bmatrix}
        \Big( \cosh(2 \kappa a) - i (\alpha/2) \sinh(2 \kappa a) \Big) e^{-2ika} & - i (\beta/2) \sinh(2 \kappa a) \\
        i (\beta/2) \sinh(2 \kappa a) & \Big( \cosh(2 \kappa a) + i(\alpha/2) \sinh(2 \kappa a) \Big) e^{2ika}
    \end{bmatrix}
    \begin{pmatrix}
    A \\ B
    \end{pmatrix}
.\end{eqnarray}
Notice that the matrix on the r.h.s is unitary\footnote{
Justification: (1) Without resorting to wave packets, notice that taking $x \rightarrow -x$, we have $A^{*} = G$ and $B^{*} = F$, meaning $|F|^2 + |G|^2 = |B^{*}|^2 + |A^{*}|^2 = |A|^2 + |B|^2$. (2) Using wave packets, we would see that $j(-\infty) = j(\infty)$, where $j$ is the probability current denity. Thus, $|A|^2 + |B|^2 = |F|^2 + |G|^2$. (3) Without the physical intuition, one could also just do the algebra brute force to take the inverse of the matrix, although this could be quite an unwieldy algebraic problem -- a minefield lying in wait for mistakes.
}, meaning we can write
\begin{eqnarray}
    \eqbox{
    \begin{pmatrix}
    A \\ B
    \end{pmatrix}
    = \begin{bmatrix}
        \Big( \cosh(2 \kappa a) + i (\alpha/2) \sinh(2 \kappa a) \Big) e^{2ika} & i (\beta/2) \sinh(2 \kappa a) \\
        -i (\beta/2) \sinh(2 \kappa a) & \Big( \cosh(2 \kappa a) - i(\alpha/2) \sinh(2 \kappa a) \Big) e^{-2ika}
    \end{bmatrix}
    \begin{pmatrix}
    F \\ G
    \end{pmatrix}
}
.\end{eqnarray}

(b) The transmission coefficient (in the limit $\kappa a \gg 1$ -- noting that $\cosh{x} \approx \sinh{x} \approx e^{x}/2$ when $x \gg 1$) is given by
\begin{eqnarray}
    \eqbox{
    \begin{aligned}
        T &= \Big| \frac{1}{\cosh(2\kappa a) + i(\alpha/2) \sinh(2 \kappa a)} e^{2 i k a} \Big|^2 \approx 16 e^{-4 \kappa a} \frac{1}{4 + \alpha^2} \\
          &= 16 e^{-4 \kappa a} \Big[ 4 + \Big( \frac{\kappa - k}{k \kappa} \Big)^2 \Big]^{-1} = 16 e^{-4 \kappa a} \Big( \frac{\kappa k}{k+k} \Big)^{2}
    .\end{aligned}
}
\end{eqnarray}
Note that this implies that the probability of transmission for such a potential is exponentially suppressed.

(c) If we now consider the case where our square barrier approaches a delta barrier ($a \rightarrow 0$, $v_0 \rightarrow \infty$, $2aV_0 = {\rm constant}$), the transmission coefficient becomes
\begin{eqnarray}
    \eqbox{
    \begin{aligned}
        T &= \frac{1}{\cosh^2(2 \kappa a ) + (\alpha/2)^2 \sinh^2(2 \kappa a)} = \frac{1}{\cosh^2(2 \sqrt{\kappa^2 a} \sqrt{a}) + \frac{\alpha^2}{4} \sinh^2(2 \sqrt{\kappa^2 a} \sqrt{a})} \\
          &= \frac{1}{1 + \frac{\kappa^2}{4k^2} (4 \kappa^2 a^2) + \ldots} \approx \frac{\epsilon}{\epsilon + v_0^2 a^2} = \frac{\epsilon}{\epsilon + (2mV_0a/\hbar^2)^2}
    ,\end{aligned}
}
\end{eqnarray}
which is the result we expected.

}


\prob{4 -- Chapter 5 \# 5}{

Consider a particle of mass $m$ under the action of a potential given by
\begin{eqnarray}
    V(x) = \frac{\hbar^2}{2m} [ v_0 \theta(-x) - w_0 \delta(x) ]
,\end{eqnarray}
where $\theta(x)$ is the Heaviside step function.

(a) Show that the presence of the potential step of height $v_0$ does not alter the boundary conditions that the wave function and its first derivative must satisfy at $x = 0$.
Is there a bound state?
If there is, what is its energy?

(b) Assume $\epsilon > v_0$.
Calculate the reflection and transmission coefficients.
What are these coefficients in the limit $\epsilon \gg v_0,w_0$?

(c) Now, assume $0 < \epsilon < v_0$.
Calculate the reflection and transmission coefficients in this case.
Explain why you could have anticipated the result.
Compute the time delay associated with the relected wave packet in this case.

}

\sol{

(a) Observe the following
\begin{eqnarray}
    \eqbox{
    \begin{aligned}
        \psi'(0^{+}) - \psi'(0^{-}) &= v_0 \int_{-\epsilon}^{\epsilon} \psi(x) \theta(-x) \dd{x} - w_0 \psi(0) \\
                                    &= v_0 \int_{-\epsilon}^{0} \psi(x) \dd{x} - w_0 \psi(0) = -w_0 \psi(0)
    \end{aligned}
}
\end{eqnarray}
as we take $\epsilon \rightarrow 0$.

Now, we analyze the energy spectrum.
Clearly, bound states may only exist for $E < 0$.
In this regime, the wave function is given as
\begin{eqnarray}
    \psi(x) = \begin{cases}
        A e^{\kappa_{-} x} & x < 0 \\
        B e^{-\kappa_{+} x} & x > 0
    ,\end{cases}
\end{eqnarray}
where $\kappa_{-} = \sqrt{v_0 + |\epsilon|}$, $\kappa_{+} = \sqrt{|\epsilon|}$.
Note that we have already used the condition that $\psi \rightarrow 0$ as $|x| \rightarrow \infty$ in order to gurantee normalizability.
Imposing the BCs at $x = 0$, we find
\begin{gather}
    A = B \\
    -\kappa_{+}B - \kappa_{-}A = -w_0 A
.\end{gather}
Therefore,
\begin{eqnarray}
    \sqrt{v_0 + |\epsilon|} + \sqrt{|\epsilon|} = w_0 \Leftrightarrow \sqrt{1 + |\epsilon| / v_0} + \sqrt{|\epsilon| / v_0} = \frac{w_0}{\sqrt{v_0}}
.\end{eqnarray}
Notice that the l.h.s. is positive for all $x = |\epsilon|/v_0 > 0$, and furthermore $\sqrt{1 + x} + \sqrt{x} > 1$, implying that $w_0 > \sqrt{v_0}$ for the existence of a bound state (only one though!).

If this condition is satisfied, we find that
\begin{eqnarray}
    \eqbox{ \epsilon = -\Big( \frac{w_0^2 - v_0}{2 w_0} \Big)^2 }
.\end{eqnarray}

Finally, notice that we may have scattering states in the energy regime $\epsilon > 0$, where the solutions for any given energy are doubly degenerate for $\epsilon > v_0$.

(b) If we are in the region $\epsilon > v_0$, we can write down the wave function as follows
\begin{eqnarray}
    \psi(x) = \begin{cases}
        e^{i K x} + Be^{-iKx} & x < 0 \\
        C e^{ikx} & x > 0
    ,\end{cases}
\end{eqnarray}
where $K = \sqrt{\epsilon - v_0}$ and $k = \sqrt{\epsilon}$.
Note that we have separated the ``right'' and ``left'' moving waves in the region $x > 0$ into two separate solutions.
The BCs at $x = 0$ give
\begin{gather}
   1 + B = C \\
    ik C - iK (1 - B) = -w_0 C
.\end{gather}
Solving for $B$ gives
\begin{eqnarray}
    B = \frac{(K - k) + iw_0}{(K+k) - iw_0}
.\end{eqnarray}
Thus, the reflection coefficient is just
\begin{eqnarray}
    R = |B|^2 = \frac{(K-k)^2 + w_0^2}{(K+k)^2 + w_0^2}
,\end{eqnarray}
and therefore,
\begin{eqnarray}
    \eqbox{ T = 1 - R = \frac{4 k K}{(K + k)^2 + w_0^2} }
.\end{eqnarray}

If we consider the case where $\epsilon \gg v_0$, then $K \approx k$ and
\begin{eqnarray}
    T \approx \frac{4 \epsilon}{w_0^2 + 4 \epsilon}
.\end{eqnarray}
Furthermore, if we have $\epsilon \gg w_0$, then
\begin{eqnarray}
    \eqbox{ T \approx 1 - \frac{w_0^2}{4 \epsilon} \rightarrow 1 }
\end{eqnarray}
as $\epsilon \rightarrow \infty$.

(c) If we have $0 < \epsilon < v_0$, then
\begin{eqnarray}
    \psi(x) = \begin{cases}
        A e^{\kappa x} & x < 0 \\
        B e^{ikx} + e^{-ikx}
    ,\end{cases}
\end{eqnarray}
where $\kappa = \sqrt{v_0 - \epsilon}$ and $k = \sqrt{\epsilon}$.
Note that we only have a single solution for a given $\epsilon$.
The $e^{-\kappa x}$ solution in the region $x < 0$ was discarded since we must have a bounded solution.

Imposing BCs at $x = 0$ gives
\begin{gather}
    A = B + 1 \\
    ik(B - 1) - \kappa A = -w_0 A
,\end{gather}
and solving for $B$ gives
\begin{eqnarray}
    B = \frac{k - i(\kappa - w_0)}{k + i(\kappa - w_0)} = -e^{-2i\delta}
,\end{eqnarray}
where
\begin{eqnarray}
    \tan{\delta} = \frac{\kappa - w_0}{k}
.\end{eqnarray}
Notice then that the transmission coefficient is
\begin{eqnarray}
    \eqbox{ T = 1 - R = 1 - |B|^2 = 0 }
.\end{eqnarray}
This matches with the intuition that the transmission coefficient is given by the probability current density at $x = -\infty$ at times far from the interaction at $t = 0$ with the potential step and $\delta$-well.
Since the energy is less than the step size, the wave-function must be exponentially damped at any given energy, and therefore, the probability current would be identically zero at $x = -\infty$.

If we introduce a wave packet, we find the reflected wave to be
\begin{eqnarray}
    \Psi_{R}(x,t) = -\int \frac{\dd{k}}{\sqrt{2 \pi}} g(k) e^{i(kx - \omega t - 2\delta)}
.\end{eqnarray}
The stationary phase method gives the center of the reflected wave packet as
\begin{eqnarray}
    x_{R}(t) = \frac{\hbar k_0}{m} t + 2 \delta'(k_0) = \frac{\hbar k_0}{m} [t - 2\tau]
,\end{eqnarray}
where $k_0$ is the point about which the profile $g(k)$ is localized.

Thus, half the delay time is 
\begin{eqnarray}
    \eqbox{ \tau = \frac{m}{\hbar k_0} \frac{(\kappa_0 - w_0) + k^2/\kappa_0}{\kappa_0^2 + (\kappa_0-w_0)^2} }
.\end{eqnarray}


}



\end{document}
