\def\duedate{\today}
\def\HWnum{\#}
% Document setup
\documentclass[12pt]{article}
\usepackage[margin=1in]{geometry}
\usepackage{fancyhdr}
\usepackage{lastpage}

\pagestyle{fancy}
\lhead{Richard Whitehill}
\chead{PHYS 621 -- HW \HWnum}
\rhead{\duedate}
\cfoot{\thepage \hspace{1pt} of \pageref{LastPage}}

% Encoding
\usepackage[utf8]{inputenc}
\usepackage[T1]{fontenc}

% Math/Physics Packages
\usepackage{amsmath}
\usepackage{amssymb}
\usepackage{mathtools}
\usepackage[arrowdel]{physics}
\usepackage{siunitx}

\AtBeginDocument{\RenewCommandCopy\qty\SI}

% Reference Style
\usepackage{hyperref}
\hypersetup{
    colorlinks=true,
    linkcolor=blue,
    filecolor=magenta,
    urlcolor=cyan,
    citecolor=green
}

\newcommand{\eref}[1]{Eq.~(\ref{eq:#1})}
\newcommand{\erefs}[2]{Eqs.~(\ref{eq:#1})--(\ref{eq:#2})}

\newcommand{\fref}[1]{Fig.~\ref{fig:#1}}
\newcommand{\frefs}[2]{Figs.~\ref{fig:#1}--\ref{fig:#2}}

\newcommand{\tref}[1]{Table~\ref{tab:#1}}
\newcommand{\trefs}[2]{Tables~\ref{tab:#1}-\ref{tab:#2}}

% Figures and Tables 
\usepackage{graphicx}
\usepackage{float}

\newcommand{\bef}{\begin{figure}[h!]\begin{center}}
\newcommand{\eef}{\end{center}\end{figure}}

\newcommand{\bet}{\begin{table}[h!]\begin{center}}
\newcommand{\eet}{\end{center}\end{table}}

% tikz
\usepackage{tikz}
\usetikzlibrary{calc}
\usetikzlibrary{decorations.pathmorphing}
\usetikzlibrary{decorations.markings}
\usetikzlibrary{arrows.meta}
\usetikzlibrary{positioning}

% tcolorbox
\usepackage[most]{tcolorbox}
\usepackage{xcolor}
\usepackage{xifthen}
\usepackage{parskip}

\newcommand*{\eqbox}{\tcboxmath[
    enhanced,
    colback=black!10!white,
    colframe=black,
    sharp corners,
    size=fbox,
    boxsep=8pt,
    boxrule=1pt
]}

% problem-solution macros
% \usepackage{adjustbox}
\usepackage{changepage}

\newtcolorbox{probbox}[1][]{
    breakable,
    enhanced,
    boxrule=0pt,
    frame hidden,
    borderline west={4pt}{0pt}{green!50!black},
    colback=green!5,
    before upper=\textbf{Problem #1) \,},
    % \textbf{Problem #1 \ifthenelse{\isempty{#1}}{}{: #1} \\ },
    sharp corners
}

% \newtcolorbox{ProblemBox}[1][]{%
%   breakable,
%   enhanced,
%   colback=black!10!white,
%   colframe=black,
%   title={\large #1 \hfill}
% }
\newcommand{\prob}[2]{
\begin{probbox}[#1]
#2
\end{probbox}
}

\newenvironment{solution}{\begin{adjustwidth}{8pt}{8pt}}{\end{adjustwidth}}
\newcommand{\sol}[1]{
\begin{solution}
#1
\end{solution}
}
% \textbf{#1)} #2}

% Miscellaneous Definitions/Settings
\newcommand{\reals}{\mathbb{R}}
\newcommand{\integers}{\mathbb{Z}}
\newcommand{\naturals}{\mathbb{N}}
\newcommand{\rationals}{\mathbb{Q}}
\newcommand{\complexs}{\mathbb{C}}

\setlength{\parskip}{\baselineskip}
\setlength{\parindent}{0pt}
\setlength{\headheight}{14.49998pt}
\addtolength{\topmargin}{-2.49998pt}


\begin{document}

\prob{1 -- Chapter 3 \# 3}{

In the $r$-representation, the Schr\"{o}dinger equation for a particle under the influence of a potential $V(\va*{r})$ is given by
\begin{eqnarray}
    i\hbar \pdv{t} \Psi(\va*{r},t) = \Bigg[ -\frac{\hbar^2}{2m} \laplacian + V(\va*{r}) \Bigg] \Psi(\va*{r},t)
.\end{eqnarray}

(a) By using the properties of the Fourier transform show that the Schrodinger equation in the $p$-representation can be wrriten as
\begin{eqnarray}
    \Big( i\hbar \pdv{t} - \frac{\va*{p}^2}{2m} \Big) \tilde{\Psi}(\va*{p},t) = \frac{1}{(2 \pi \hbar)^{3/2}} \int \dd[3]{\va*{p}'} \tilde{V}(\va*{p} - \va*{p}') \tilde{\Psi}(\va*{p}',t)
,\end{eqnarray}
where
\begin{eqnarray}
    \tilde{\Psi}(\va*{p},t) = \int \frac{\dd[3]{\va*{r}}}{(2 \pi \hbar)^{3/2}} e^{-i \va*{p} \cdot \va*{r} / \hbar} \Psi(\va*{r},t), \quad \tilde{\va*{p}} = \int \frac{\dd[3]{\va*{r}}}{(2 \pi \hbar)^{3/2}} e^{-i \va*{p} \cdot \va*{r} / \hbar} V(\va*{r})
.\end{eqnarray}
Assume that $V(\va*{r})$ vanishes as $|\va*{r}| \rightarrow \infty$ fast enough for its Fourier transform to exist.
Note that in the $p$-space Schr\"{o}dinger equation as given above, the term with the potential reduces to a convolution product involving the Fourier transforms $\tilde{V}(\va*{p})$ and $\tilde{\Psi}(\va*{p},t)$.

(b) SHow that the $p$-space Schr\"{o}dinger equation can also be written as
\begin{eqnarray}
    i \hbar \pdv{t} \tilde{\Psi}(\va*{p},t) = \Big[ \frac{\va*{p}^2}{2m} + V(i\hbar \grad_{p}) \Big] \tilde{\Psi}(\va*{p},t)
\end{eqnarray}
as given in the notes.

}

\sol{



}


\prob{2 -- Chapter 4 \# 2}{

This problem deals with the solution of the S.E. in momentum space.

(a) Solve directly in momentum space for the bound-state energy and wave funciton in an attractive $\delta$-function potential, given in coordinate space by $V(x) = -V_0 \delta(x)$ with $V_0 > 0$.

(b) From the momentum space wave function so obtained, calculate the average kinetic energy of the particle.
Repeat the calculation of this average kinetic energy but in coordinate space, using the wave function you derived in the previous problem.
Note that the first derivative of the coordinate-space bound-state wave function is discontinuous at $x = 0$.

}

\sol{}


\prob{3 -- Chapter 5 \# 4}{

Consider a particle in one-dimension under the influence of the even potential given by
\begin{eqnarray}
   V(x) = \begin{cases}
       V_0 & |x| \leq a \\
       0 & |x| > a
   ,\end{cases}
\end{eqnarray}
where $V_0 > 0$.
Assume that the particle's energy $E$ is in the range $0 < E < V_0$.
Write the general solution as 
\begin{eqnarray}
   \psi(x) = \begin{cases}
       Ae^{ikx} + Be^{-ikx} & x < -a \\
       C e^{-\kappa x} + D e^{\kappa x} & |x| < a \\
       Fe^{ikx} + Ge^{-ikx} & x > a
   ,\end{cases}
\end{eqnarray}
where $k = \sqrt{\epsilon}$, $\kappa = \sqrt{v_0 - \epsilon}$, $\epsilon = 2mE/\hbar^2$, and $v_0 = 2mV_0/\hbar^2$.

(a) By imposing the appropriate boundary conditions at $x = \pm a$, show that the coefficients $A$ and $B$ are related to the coefficients $F$ and $G$ in the following way
\begin{eqnarray}
    \begin{pmatrix}
    A \\ B
    \end{pmatrix}
    = 
    \begin{bmatrix}
        (\cosh{2 \kappa a} + i(\alpha/2) \sinh{2 \kappa a})e^{2ika} & i(\beta/2) \sinh{2 \kappa a} \\

        -i(\beta/2) \sinh{2 \kappa a} & (\cosh{2 \kappa a} - i(\alpha/2) \sinh{2 \kappa a})e^{-2ika} \\
    \end{bmatrix}
    \begin{pmatrix}
    F \\ G
    ,\end{pmatrix}
\end{eqnarray}
where
\begin{eqnarray}
   \alpha = \frac{\kappa}{k} - \frac{k}{\kappa},~ \beta = \frac{\kappa}{k} + \frac{k}{\kappa},~\beta^2 - \alpha^2 = 4
.\end{eqnarray}

(b) From the matrix relation obtained in part (a) above, show that the transmission coefficient $T$ for a high and wide barrier with $\kappa a \gg 1$ is approximately given by 
\begin{eqnarray}
    T \approx 16 e^{-4 \kappa a} \Big( \frac{\kappa k}{k^2 + \kappa^2} \Big)^2
.\end{eqnarray}

(c) Consider the case of a very narrow but very high barrier such that $a V_0$ is finite.
Assume $V_0 \gg E$ $\kappa \gg k$, and $\kappa a \ll 1$, but $\kappa^2 a$ is finite.
Show that the transmission coefficient under these conditions is the same as that obtained in the repulsive $\delta$-function potential $V(x) = 2aV_0\delta(x)$, namely
\begin{eqnarray}
    T = \frac{\epsilon}{\epsilon + (2mV_0a/\hbar^2)^2}
.\end{eqnarray}


}

\sol{}


\prob{4 -- Chapter 5 \# 5}{

Consider a particle of mass $m$ under the action of a potential given by
\begin{eqnarray}
    V(x) = \frac{\hbar^2}{2m} [ v_0 \theta(-x) - w_0 \delta(x) ]
,\end{eqnarray}
where $\theta(z)$ is the Heaviside step function.

(a) Show that the presence of the potential step of height $v_0$ does not alter the boundary conditions that the wave function and its first derivative must satisfy at $x = 0$.
Is there a bound state?
If there is, what is its energy?

(b) Assume $\epsilon > v_0$.
Calculate the reflection and transmission coefficients.
What are these coefficients in the limit $\epsilon \gg v_0,w_0$?

(c) Now, assume $0 < \epsilon < v_0$.
Calculate the reflection and transmission coefficients in this case.
Explain why you could have anticipated the result.
Compute the time delay associated with the relected wave packet in this case.

}

\sol{}



\end{document}
