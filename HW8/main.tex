\def\duedate{\today}
\def\HWnum{8}
\documentclass[10pt,a4paper]{book}

% custom section formatting
\usepackage{titlesec}
\titleformat{\chapter}[display]
{\normalfont\Large\filcenter\sffamily}
{\titlerule[1pt]%
\vspace{1pt}%
\titlerule
\vspace{1pc}%
\LARGE\MakeUppercase{\chaptertitlename} \thechapter}
{1pc}
{\titlerule
\vspace{1pc}%
\Huge}

% appendix handling
\usepackage[toc,page]{appendix}
    
% encoding for file and font
\usepackage[utf8]{inputenc}
\usepackage[T1]{fontenc}

% math formatting/tools
\usepackage{amsmath}
\usepackage{amssymb}
\usepackage{mathtools}
\usepackage[arrowdel]{physics}

% unit formatting
\usepackage{siunitx}
\AtBeginDocument{\RenewCommandCopy\qty\SI}

% figure formatting/tools
\usepackage{graphicx}
\usepackage{float}
\usepackage{subcaption}
\usepackage{multirow}
\usepackage{import}
\usepackage{pdfpages}
\usepackage{transparent}
\usepackage{currfile}

\NewDocumentCommand\incfig{O{1} m}{
    \def\svgwidth{#1\textwidth}
    \import{./Figures/\currfiledir}{#2.pdf_tex}
}

\newcommand{\bef}{\begin{figure}[h!tb]\centering}
\newcommand{\eef}{\end{figure}}

\newcommand{\bet}{\begin{table}[h!tb]\centering}
\newcommand{\eet}{\end{table}}

% hyperlink references 
\usepackage{hyperref}
\hypersetup{
    colorlinks=true,
    linkcolor=blue,
    filecolor=magenta,
    urlcolor=cyan,
    pdftitle={Physics 1 Notes},
    pdfauthor={Richard Whitehill},
    pdfpagemode=FullScreen
}
\urlstyle{same}

\newcommand{\eref}[1]{Eq.~(\ref{eq:#1})}
\newcommand{\erefs}[2]{Eqs.~(\ref{eq:#1})--(\ref{eq:#2})}

\newcommand{\fref}[1]{Fig.~(\ref{fig:#1})}
\newcommand{\frefs}[2]{Fig.~(\ref{fig:#1})--(\ref{fig:#2})}

\newcommand{\aref}[1]{Appendix~(\ref{app:#1})}
\newcommand{\sref}[1]{Section~(\ref{sec:#1})}
\newcommand{\srefs}[2]{Sections~(\ref{sec:#1})-(\ref{sec:#2})}

\newcommand{\tref}[1]{Table~(\ref{tab:#1})}
\newcommand{\trefs}[2]{Table~(\ref{tab:#1})--(\ref{tab:#2})}

% tcolorbox formatting/definitions
\usepackage[most]{tcolorbox}
\usepackage{xcolor}
\usepackage{xifthen}
\usepackage{parskip}

\definecolor{peach}{rgb}{1.0,0.8,0.64}

\DeclareTColorBox[auto counter, number within=chapter]{defbox}{O{}}{
    enhanced,
    boxrule=0pt,
    frame hidden,
    borderline west={4pt}{0pt}{green!50!black},
    colback=green!5,
    before upper=\textbf{Definition \thetcbcounter \ifthenelse{\isempty{#1}}{}{: #1} \\ },
    sharp corners
}

\newcommand*{\eqbox}{\tcboxmath[
    enhanced,
    colback=black!10!white,
    colframe=black,
    sharp corners,
    size=fbox,
    boxsep=8pt,
    boxrule=1pt
]}

\newtcolorbox[auto counter, number within=chapter]{exbox}{
    parbox=false,
    breakable,
    enhanced,
    sharp corners,
    boxrule=1pt,
    colback=white,
    colframe=black,
    before upper= \textbf{Example \thetcbcounter:}\,,
    before lower= \textbf{Solution:}\,,
    segmentation hidden
}

\newtcolorbox{resbox}{
    enhanced,
    colback=black!10!white,
    colframe=black,
    boxrule=1pt,
    boxsep=0pt,
    top=2pt,
    ams nodisplayskip,
    sharp corners
}


\begin{document}

\prob{1 -- Chapter 6 \#9}{

A \textit{normal} operator is one that commutes with its adjoint.
A linear algebra theorem assures that (at least for a finite-dimensional state space) a normal operator can be diagonalized and that the eigenvectors form a basis in the state space.

(a) Prove the converse statement that an operator whose eigenstates form a basis in the state space must be normal.

(b) A hermitian operator is an obvious example of a normal operator.
Is a unitary operator also a normal operator?
Show that the eigenvalues of a unitary operator all have unit magnitude.

(c) Show that the eigenvalues of an operator are independent of the basis used to represent this operator.

(d) Prove that if $A$ is a normal matrix representing the (normal) operator $\hat{A}$, then
\begin{eqnarray}
    \det(e^{A}) = e^{\Tr A}, \quad \Tr A = \sum_i A_{ii}
.\end{eqnarray}

}

\sol{

(a) Suppose that $\{ \ket{\psi_n} \}$ is the set of eigenvectors of an operator $\hat{A}$ satisfying $\hat{A} \ket{\psi_{n}} = a_{n} \ket{\psi_{n}}$, where $n$ is a discrete index from a finite set.
Further suppose that any vector
\begin{eqnarray}
    \ket{\phi} = \sum_{n} c_{n} \ket{\psi_{n}}
,\end{eqnarray}
where the coefficients $c_{n} = \bra{\phi}\ket{\psi_{n}}$.
An equivalent, alternative statement of this completeness is
\begin{eqnarray}
    \sum_{n} \bra{\psi_{n}} \ket{\psi_{n}} = \id
.\end{eqnarray}
We want to prove then that the operator $\hat{A}$ is normal: $[\hat{A}^{\dagger},\hat{A}] = 0$.
Observe the following:
\begin{eqnarray}
\begin{aligned}
    [A^{\dagger},A] &= A^{\dagger} A - A A^{\dagger} \\
                    &= \id A^{\dagger} A \id - A \id A^{\dagger} \\
                    &= \sum_{m} \ket{\psi_{m}}\bra{\psi_{m}} A^{\dagger} A \sum_{n} \ket{\psi_{n}} \bra{\psi_{n}} - A \sum_{n} \ket{\psi_{n}} \bra{\psi_{n}} A^{\dagger} \\
                    &= \sum_{m} \sum_{n} a_{m}^{*} a_{n} \ket{\psi_{m}} \underbrace{\bra{\psi_{m}} \ket{\psi_{n}}}_{\delta_{nm}} \bra{\psi_{n}} - \sum_{n} |a_{n}|^2 \ket{\psi_{n}} \bra{\psi_{n}} \\
                    &= \sum_{n} |a_{n}|^2 \bra{\psi_{n}} \ket{\psi_{n}} - \sum_{n} |a_{n}|^2 \bra{\psi_{n}} \ket{\psi_{n}} = 0
.\end{aligned}
\end{eqnarray}
From this, we see that $A$ is a normal operator since it commutes with its adjoint.

(b) A hermitian operator $A$ is clearly a normal operator since $[A^{\dagger},A] = [A,A] = 0$.
On the other hand, a unitary operator $A$ is also normal since
\begin{eqnarray}
    [A^{\dagger},A] = A^{\dagger} A - A A^{\dagger} = \id - \id = 0
.\end{eqnarray}

Suppose that $\ket{\psi}$ is an eigenstate of $A$, which is unitary, with corresponding eigenvalue $\lambda$.
That is,
\begin{eqnarray}
    A \ket{\psi} = \lambda \ket{\psi}
.\end{eqnarray}
Using the fact that $A$ is a normal operator (as shown above), we have
\begin{eqnarray}
\begin{aligned}
    \bra{\psi} [A^{\dagger},A] \ket{\psi} = &\bra{\psi} A^{\dagger} A \ket{\psi} - \bra{\psi} A A^{\dagger} \ket{\psi} = 0 \\
                                          &|\lambda|^2 - 1 = 0 \Rightarrow |\lambda| = 1
.\end{aligned}
\end{eqnarray}

(c) We will show that the spectrum of an operator in a basis which is not the eigenbasis still yields the same eigenvalues for the operator.
Suppose that $\{ \ket{\phi_{n}} \} $ is a basis of the state space which is not the eigenbasis $\{ \ket{\psi_{n}} \} $ of the operator $A$.
Note that different bases must have the same dimension since they are minimal sets that span the state space, so we can label them with the same index set.
The matrix representation of $A$ in the basis $\{ \phi_{n} \}$ is given as
\begin{eqnarray}
    A^{(\phi)} = (A^{(\phi)}_{mn}) = \bra{\phi_{m}} A \ket{\phi_{n}}
,\end{eqnarray}
while in the eigenbasis $\{ \psi_{n} \} $, the matrix is is diagonal:
\begin{eqnarray}
    A^{(\psi)} = (A^{(\psi)}_{mn}) = \bra{\psi_{m}} A \ket{\psi_{n}} = a_{n} \delta_{mn}
.\end{eqnarray}
Since $\{ \ket{\psi_{n}} \} $ is complete, we can change bases using that
\begin{eqnarray}
    \ket{ \phi_{n} } = \sum_{k} c_{nk} \ket{\psi_{k}}
,\end{eqnarray}
where $c_{nk} = \bra{\phi_{n}}\ket{\psi_{k}}$.
The operator $C = (c_{nm})$ then gives the basis transformation from $\{ \ket{ \phi_{n} } \} $ to $\{ \ket{\psi_{n}} \} $.
Furthermore, note that $C$ is unitary\footnote{This is true since we can reverse the argument and write $\ket{\psi_{n}}$ in terms of the vectors $\ket{\phi_{m}}$. Doing so gives that $C C^{\dagger} = \id$.}
That is, if $\va*{\alpha}_{\phi}$ is the vector of coordinates in the basis $\{ \ket{ \phi_{n} } \} $, then $\va*{\alpha}_{\psi} = C \va*{\alpha}_{\phi}$ is the vector of coordinates in the basis $\{ \ket{ \psi_{n} } \} $.
It also then follows that $A^{(\phi)} = C A^{(\psi)} C^{\dagger}$.
Finally, recall that the eigenvalues are just the roots of a matrix's characteristic equation:
\begin{gather}
    \det(A^{(\psi)} - \lambda \id) = 0 \nonumber \\
    \det(C^{\dagger} A^{(\phi)} C - \lambda C^{\dagger} C \id C^{\dagger} C) = 0 \nonumber \\
    \det[ C^{\dagger} ( A^{(\phi)} - \lambda C C^{\dagger} ) C ] = 0 \nonumber \\
    \det(C^{\dagger}) \det(A^{(\phi)} - \lambda \id ) \det(C) = 0 \nonumber \\
    \det(C^{\dagger} C) \det(A^{(\phi)} - \lambda \id) = 0 \nonumber \\
    \det(A^{(\phi)} - \lambda \id) = 0
.\end{gather}
Thus, the eigenvalues $\lambda$ of $A^{(\psi)}$ satisfy the characteristic equation of $A$ in the basis $\{ \ket{\phi_{n}} \}$, meaning that the eigenvalues of $A$ are independent of the basis used to represent $A$.

It is strongly emphasized, though, that this result hinges on the completeness of the basis of the state-space and therefore the normality of the operator $A$.
If $A$ is not normal then the eigenvectors do not necessarily form a complete set, and it is generally not true that there exists an operator $C$ which transforms between the set of eigenvectors of $A$ and a generic basis of the state space.

(d) In part (c), we saw that a normal matrix $A$ can be diagonalized as
\begin{eqnarray}
    A = C D C^{\dagger}
,\end{eqnarray}
where $D$ is a diagonal matrix with $D_{nn} = \lambda_{n}$ being the $n^{\rm th}$ eigenvalue of $A$ and $C$ is the matrix which transforms $A$ from its representation in an arbitrary basis to its eigenbasis.
It follows then that
\begin{eqnarray}
    \det(e^{A}) = \det(e^{C D C^{\dagger}}) = \det(C^{\dagger} e^{D} C) = \det(C^{\dagger} C e^{D}) = \det(e^{D}) = e^{\sum_{n} \lambda_{n}} = e^{\Tr(A)}
.\end{eqnarray}
This line of algebra hinged on two facts.
First,
\begin{eqnarray}
\begin{aligned}
    e^{C D C^{\dagger}} &= \sum_{n=0}^{\infty} \frac{(C D C^{\dagger})^{n}}{n!} = \sum_{n=0}^{\infty} \frac{1}{n!} (C D C^{\dagger}) (C D C^{\dagger}) \ldots (C D C^{\dagger}) (C D C^{\dagger}) = \sum_{n=0}^{\infty} \frac{C D^{n} C^{\dagger}}{n!} \\
                        &= C \Bigg( \sum_{n=0}^{\infty} \frac{D^{n}}{n!} \Bigg) C^{\dagger} = C e^{D} C^{\dagger}
,\end{aligned}
\end{eqnarray}
and second,
\begin{eqnarray}
    \Tr(A) = \Tr(C D C^{\dagger}) = \Tr(C^{\dagger} C D) = \Tr(D)
,\end{eqnarray}
using the cylic behavior of the trace.

}


\prob{2 -- Chapter 7 \# 3}{

Let $\psi(\va*{r})$ be the normalized wave function of a particle.
Express in terms of $\psi(\va*{r})$ the probability for:

(a) A measurement of the momentum component $\hat{p}_{x}$ to yield a result included between $p_1$ and $p_2$.

(b) A simultaneous measurement of the position component $\hat{x}$ and the momentum component $\hat{p}_{z}$ to yield results $x_1 \leq x \leq x_2$ and $p_{z} \geq 0$, respectively.

(c) A simultaneous measurement of the three momentum components $\hat{p}_{x},~\hat{p}_{y},~{\rm and}~\hat{p}_{z}$ to yield results $p_1 \leq p_{x} \leq p_{2}$, $p_{3} \leq p_{y} \leq p_{4}$, and $p_{5} \leq p_{z} \leq p_{6}$.
Show that this probability is the same as obtained in part (a) above when $p_{3},p_{5} \rightarrow -\infty$ and $p_{4},p_{6} \rightarrow \infty$.

}

\sol{

(a) 

}


\prob{3 -- Chapter 7 \# 4}{

Consider three non-commuting observables $\hat{A}$, $\hat{B}$, and $\hat{C}$.
Observable $\hat{A}$ is measured yielding the non-degenerate eigenvalue $a$ with corresponding eigenstate $\ket{\psi_{a}}$.
Then, the following two different experiments are carried out.

(a) In the first experiment, before the system has had any time to evolve after the measurement of $\hat{A}$, observable $\hat{C}$ is measured yielding the non-degenerate eigenvalue $c$ with corresponding eigenstate $\ket{\chi_{c}}$.
Calculate the probability $P_{a}(c)$ to measure $c$.

(b) In the second experiment, observables $\hat{B}$ and $\hat{C}$ are measured in rapid succession, so that the system has not had any time to evolve between the measurements of $\hat{A}$ and $\hat{B}$ and those of $\hat{B}$ and $\hat{C}$.
Suppose these measurements yield, respectively, the non-degenerate eigenvalues $b$ and $c$, where $\ket{\phi_{b}}$ is the eigenstate corresponding to the eigenvalue $b$.
Calculate the probability $P_{a}(b,c)$ to measure $b$ and $c$.

(c) Assuming that the eigenvalues of $\hat{B}$ are all non-degenerate, show that
\begin{eqnarray}
    P_{a}(c) = \sum_{b} P_{a}(b,c) + \mathbf{X}
.\end{eqnarray}
Provide an expression for $\mathbf{X}$.
What is then the essential difference between the two experiments?
What conclusions can you draw?

}

\sol{}


\prob{4 -- Chapter \#7}{

Consider a system with a three-dimensional state space.
The Hamiltonian $\hat{H}$ has a non-degenerate eigenvalue $E_1 = E_0$ with (normalized) eigenstate $\ket{\phi_1}$ and a degenerate eigenvalue $E_2 = 2E_0$ with (orthonormal) eigenstates $\ket{\phi_2}$ and $\ket{\phi_3}$.
Suppose at time $t = 0$ the system is in the normalized state $\ket{\psi(0)}$ given by
\begin{eqnarray}
    \ket{\psi(0)} = \frac{1}{\sqrt{2}} \ket{\phi_1} + \frac{1}{2}( \ket{\phi_2} + \ket{\phi_3} )
.\end{eqnarray}

(a) At $t = 0$ the energy of the system is measured.
What values can be found and with what probabilities?
Calculate $\expval{H}$ and the root-mean-square deviation $\Delta H$ for the system in the state $\ket{\psi(0)}$.

(b) Suppose at $t = 0$, instead of $\hat{H}$, the observable $\hat{A}$, which in the basis $\ket{\phi_1}$, $\ket{\phi_2}$, and $\ket{\phi_3}$ is represented by the following matrix
\begin{eqnarray}
    A = a \begin{pmatrix}
        1 & 0 & 0 \\
        0 & 0 & 1 \\
        0 & 1 & 0
    \end{pmatrix}
,\end{eqnarray}
is measured with $a \in \reals^{+}$.
What results can be found and with what probabilities?

(c) Obtain the state vector $\ket{\psi(t)}$ at time $t$.

(d) In addition to $\hat{A}$ above, consider another observable $\hat{B}$ with respresentation (in the same basis of Hamiltonian eigenstates) given by
\begin{eqnarray}
    B = b \begin{pmatrix}
        0 & 1 & 0 \\
        1 & 0 & 0 \\
        0 & 0 & 1
    \end{pmatrix}
,\end{eqnarray}
where $b \in \reals^{+}$.
What are the mean values $\expval{A(t)}$ and $\expval{B(t)}$?
Any comments?

(e) What results are obtained if $\hat{A}$ is measured at time $t$?
Same question for $\hat{B}$?
Interpret these results.

}

\sol{}


\end{document}
