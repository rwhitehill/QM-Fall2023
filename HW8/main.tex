\def\duedate{\today}
\def\HWnum{8}
\documentclass[10pt,a4paper]{book}

% custom section formatting
\usepackage{titlesec}
\titleformat{\chapter}[display]
{\normalfont\Large\filcenter\sffamily}
{\titlerule[1pt]%
\vspace{1pt}%
\titlerule
\vspace{1pc}%
\LARGE\MakeUppercase{\chaptertitlename} \thechapter}
{1pc}
{\titlerule
\vspace{1pc}%
\Huge}

% appendix handling
\usepackage[toc,page]{appendix}
    
% encoding for file and font
\usepackage[utf8]{inputenc}
\usepackage[T1]{fontenc}

% math formatting/tools
\usepackage{amsmath}
\usepackage{amssymb}
\usepackage{mathtools}
\usepackage[arrowdel]{physics}

% unit formatting
\usepackage{siunitx}
\AtBeginDocument{\RenewCommandCopy\qty\SI}

% figure formatting/tools
\usepackage{graphicx}
\usepackage{float}
\usepackage{subcaption}
\usepackage{multirow}
\usepackage{import}
\usepackage{pdfpages}
\usepackage{transparent}
\usepackage{currfile}

\NewDocumentCommand\incfig{O{1} m}{
    \def\svgwidth{#1\textwidth}
    \import{./Figures/\currfiledir}{#2.pdf_tex}
}

\newcommand{\bef}{\begin{figure}[h!tb]\centering}
\newcommand{\eef}{\end{figure}}

\newcommand{\bet}{\begin{table}[h!tb]\centering}
\newcommand{\eet}{\end{table}}

% hyperlink references 
\usepackage{hyperref}
\hypersetup{
    colorlinks=true,
    linkcolor=blue,
    filecolor=magenta,
    urlcolor=cyan,
    pdftitle={Physics 1 Notes},
    pdfauthor={Richard Whitehill},
    pdfpagemode=FullScreen
}
\urlstyle{same}

\newcommand{\eref}[1]{Eq.~(\ref{eq:#1})}
\newcommand{\erefs}[2]{Eqs.~(\ref{eq:#1})--(\ref{eq:#2})}

\newcommand{\fref}[1]{Fig.~(\ref{fig:#1})}
\newcommand{\frefs}[2]{Fig.~(\ref{fig:#1})--(\ref{fig:#2})}

\newcommand{\aref}[1]{Appendix~(\ref{app:#1})}
\newcommand{\sref}[1]{Section~(\ref{sec:#1})}
\newcommand{\srefs}[2]{Sections~(\ref{sec:#1})-(\ref{sec:#2})}

\newcommand{\tref}[1]{Table~(\ref{tab:#1})}
\newcommand{\trefs}[2]{Table~(\ref{tab:#1})--(\ref{tab:#2})}

% tcolorbox formatting/definitions
\usepackage[most]{tcolorbox}
\usepackage{xcolor}
\usepackage{xifthen}
\usepackage{parskip}

\definecolor{peach}{rgb}{1.0,0.8,0.64}

\DeclareTColorBox[auto counter, number within=chapter]{defbox}{O{}}{
    enhanced,
    boxrule=0pt,
    frame hidden,
    borderline west={4pt}{0pt}{green!50!black},
    colback=green!5,
    before upper=\textbf{Definition \thetcbcounter \ifthenelse{\isempty{#1}}{}{: #1} \\ },
    sharp corners
}

\newcommand*{\eqbox}{\tcboxmath[
    enhanced,
    colback=black!10!white,
    colframe=black,
    sharp corners,
    size=fbox,
    boxsep=8pt,
    boxrule=1pt
]}

\newtcolorbox[auto counter, number within=chapter]{exbox}{
    parbox=false,
    breakable,
    enhanced,
    sharp corners,
    boxrule=1pt,
    colback=white,
    colframe=black,
    before upper= \textbf{Example \thetcbcounter:}\,,
    before lower= \textbf{Solution:}\,,
    segmentation hidden
}

\newtcolorbox{resbox}{
    enhanced,
    colback=black!10!white,
    colframe=black,
    boxrule=1pt,
    boxsep=0pt,
    top=2pt,
    ams nodisplayskip,
    sharp corners
}


\begin{document}

\prob{1 -- Chapter 6 \#9}{

A \textit{normal} operator is one that commutes with its adjoint.
A linear algebra theorem assures that (at least for a finite-dimensional state space) a normal operator can be diagonalized and that the eigenvectors form a basis in the state space.

(a) Prove the converse statement that an operator whose eigenstates form a basis in the state space must be normal.

(b) A hermitian operator is an obvious example of a normal operator.
Is a unitary operator also a normal operator?
Show that the eigenvalues of a unitary operator all have unit magnitude.

(c) Show that the eigenvalues of an operator are independent of the basis used to represent this operator.

(d) Prove that if $A$ is a normal matrix representing the (normal) operator $\hat{A}$, then
\begin{eqnarray}
    \det(e^{A}) = e^{\Tr A}, \quad \Tr A = \sum_i A_{ii}
.\end{eqnarray}

}

\sol{

(a) Suppose that $\{ \ket{\psi_n} \}$ is the set of eigenvectors of an operator $\hat{A}$ satisfying $\hat{A} \ket{\psi_{n}} = a_{n} \ket{\psi_{n}}$, where $n$ is a discrete index from a finite set.
Further suppose that any vector
\begin{eqnarray}
    \ket{\phi} = \sum_{n} c_{n} \ket{\psi_{n}}
,\end{eqnarray}
where the coefficients $c_{n} = \bra{\phi}\ket{\psi_{n}}$.
An equivalent, alternative statement of this completeness is
\begin{eqnarray}
    \sum_{n} \bra{\psi_{n}} \ket{\psi_{n}} = \id
.\end{eqnarray}
We want to prove then that the operator $\hat{A}$ is normal: $[\hat{A}^{\dagger},\hat{A}] = 0$.
Observe the following:
\begin{eqnarray}
\begin{aligned}
    [A^{\dagger},A] &= A^{\dagger} A - A A^{\dagger} \\
                    &= \id A^{\dagger} A \id - A \id A^{\dagger} \\
                    &= \sum_{m} \ket{\psi_{m}}\bra{\psi_{m}} A^{\dagger} A \sum_{n} \ket{\psi_{n}} \bra{\psi_{n}} - A \sum_{n} \ket{\psi_{n}} \bra{\psi_{n}} A^{\dagger} \\
                    &= \sum_{m} \sum_{n} a_{m}^{*} a_{n} \ket{\psi_{m}} \underbrace{\bra{\psi_{m}} \ket{\psi_{n}}}_{\delta_{nm}} \bra{\psi_{n}} - \sum_{n} |a_{n}|^2 \ket{\psi_{n}} \bra{\psi_{n}} \\
                    &= \sum_{n} |a_{n}|^2 \bra{\psi_{n}} \ket{\psi_{n}} - \sum_{n} |a_{n}|^2 \bra{\psi_{n}} \ket{\psi_{n}} = 0
.\end{aligned}
\end{eqnarray}
From this, we see that $A$ is a normal operator since it commutes with its adjoint.

(b) A hermitian operator $A$ is clearly a normal operator since $[A^{\dagger},A] = [A,A] = 0$.
On the other hand, a unitary operator $A$ is also normal since
\begin{eqnarray}
    [A^{\dagger},A] = A^{\dagger} A - A A^{\dagger} = \id - \id = 0
.\end{eqnarray}

Suppose that $\ket{\psi}$ is an eigenstate of $A$, which is unitary, with corresponding eigenvalue $\lambda$.
That is,
\begin{eqnarray}
    A \ket{\psi} = \lambda \ket{\psi}
.\end{eqnarray}
Using the fact that $A$ is a normal operator (as shown above), we have
\begin{eqnarray}
\begin{aligned}
    \bra{\psi} [A^{\dagger},A] \ket{\psi} = &\bra{\psi} A^{\dagger} A \ket{\psi} - \bra{\psi} A A^{\dagger} \ket{\psi} = 0 \\
                                          &|\lambda|^2 - 1 = 0 \Rightarrow |\lambda| = 1
.\end{aligned}
\end{eqnarray}

(c) We will show that the spectrum of an operator in a basis which is not the eigenbasis still yields the same eigenvalues for the operator.
Suppose that $\{ \ket{\phi_{n}} \} $ and $\{ \ket{\psi_{n}} \}$ are two distinct bases of the operator $A$, which are not necessarily the eigenbasis of $A$.
Note that different bases must have the same dimension since they are minimal sets that span the state space, so we can label them with the same index set.
The matrix representation of $A$ in the basis $\{ \phi_{n} \}$ is given as
\begin{eqnarray}
    A^{(\phi)} = (A^{(\phi)}_{mn}) = \bra{\phi_{m}} A \ket{\phi_{n}}
,\end{eqnarray}
while in the eigenbasis $\{ \psi_{n} \} $
\begin{eqnarray}
    A^{(\psi)} = (A^{(\psi)}_{mn}) = \bra{\psi_{m}} A \ket{\psi_{n}}
.\end{eqnarray}
Since $\{ \ket{\psi_{n}} \} $ is complete, we can change bases using that
\begin{eqnarray}
    \ket{ \phi_{n} } = \sum_{k} c_{nk} \ket{\psi_{k}}
,\end{eqnarray}
where $c_{nk} = \bra{\phi_{n}}\ket{\psi_{k}}$.
The operator $C = (c_{nm})$ then gives the basis transformation from $\{ \ket{ \phi_{n} } \} $ to $\{ \ket{\psi_{n}} \} $.
Furthermore, note that $C$ is unitary\footnote{This is true since we can reverse the argument and write $\ket{\psi_{n}}$ in terms of the vectors $\ket{\phi_{m}}$. Doing so gives that $C C^{\dagger} = \id$.}.
That is, if $\va*{\alpha}_{\phi}$ is the vector of coordinates in the basis $\{ \ket{ \phi_{n} } \} $, then $\va*{\alpha}_{\psi} = C \va*{\alpha}_{\phi}$ is the vector of coordinates in the basis $\{ \ket{ \psi_{n} } \} $.
It also then follows that $A^{(\phi)} = C A^{(\psi)} C^{\dagger}$\footnote{This is proven by considering how the matrix elements of $A$ transform between bases.}.
Finally, recall that the eigenvalues $\lambda$ are just the roots of a matrix's characteristic equation:
\begin{gather}
    \det(A^{(\psi)} - \lambda \id) = 0 \nonumber \\
    \det(C^{\dagger} A^{(\phi)} C - \lambda C^{\dagger} C \id C^{\dagger} C) = 0 \nonumber \\
    \det[ C^{\dagger} ( A^{(\phi)} - \lambda C C^{\dagger} ) C ] = 0 \nonumber \\
    \det(C^{\dagger}) \det(A^{(\phi)} - \lambda \id ) \det(C) = 0 \nonumber \\
    \det(C^{\dagger} C) \det(A^{(\phi)} - \lambda \id) = 0 \nonumber \\
    \det(A^{(\phi)} - \lambda \id) = 0
.\end{gather}
Thus, the roots $\lambda$ of the characteristic equation for $A$ in the basis $\{ \ket{\psi_{n}} \}$ also satisfy the characteristic equation of $A$ in the basis $\{ \ket{\phi_{n}} \}$, meaning that the eigenvalues of $A$ are independent of the basis used to represent $A$.

It is strongly emphasized, though, that this result hinges on the completeness of the basis of the state-space and therefore the normality of the operator $A$.
If $A$ is not normal then the eigenvectors do not necessarily form a complete set, and it is generally not true that there exists an operator $C$ which transforms between the set of eigenvectors of $A$ and a generic basis of the state space.

(d) In part (c), we saw that a normal matrix $A$ can be diagonalized as
\begin{eqnarray}
    A = C D C^{\dagger}
,\end{eqnarray}
where $D$ is a diagonal matrix with $D_{nn} = \lambda_{n}$ being the $n^{\rm th}$ eigenvalue of $A$ and $C$ is the matrix which transforms $A$ from its representation in an arbitrary basis to its eigenbasis.
It follows then that
\begin{eqnarray}
    \det(e^{A}) = \det(e^{C D C^{\dagger}}) = \det(C^{\dagger} e^{D} C) = \det(C^{\dagger} C e^{D}) = \det(e^{D}) = e^{\sum_{n} \lambda_{n}} = e^{\Tr(A)}
.\end{eqnarray}
This line of algebra hinged on two facts.
First,
\begin{eqnarray}
\begin{aligned}
    e^{C D C^{\dagger}} &= \sum_{n=0}^{\infty} \frac{(C D C^{\dagger})^{n}}{n!} = \sum_{n=0}^{\infty} \frac{1}{n!} (C D C^{\dagger}) (C D C^{\dagger}) \ldots (C D C^{\dagger}) (C D C^{\dagger}) = \sum_{n=0}^{\infty} \frac{C D^{n} C^{\dagger}}{n!} \\
                        &= C \Bigg( \sum_{n=0}^{\infty} \frac{D^{n}}{n!} \Bigg) C^{\dagger} = C e^{D} C^{\dagger}
,\end{aligned}
\end{eqnarray}
and second,
\begin{eqnarray}
    \Tr(A) = \Tr(C D C^{\dagger}) = \Tr(C^{\dagger} C D) = \Tr(D)
,\end{eqnarray}
using the cylic behavior of the trace.

}


\prob{2 -- Chapter 7 \# 3}{

Let $\psi(\va*{r})$ be the normalized wave function of a particle.
Express in terms of $\psi(\va*{r})$ the probability for:

(a) A measurement of the momentum component $\hat{p}_{x}$ to yield a result included between $p_1$ and $p_2$.

(b) A simultaneous measurement of the position component $\hat{x}$ and the momentum component $\hat{p}_{z}$ to yield results $x_1 \leq x \leq x_2$ and $p_{z} \geq 0$, respectively.

(c) A simultaneous measurement of the three momentum components $\hat{p}_{x},~\hat{p}_{y},~{\rm and}~\hat{p}_{z}$ to yield results $p_1 \leq p_{x} \leq p_{2}$, $p_{3} \leq p_{y} \leq p_{4}$, and $p_{5} \leq p_{z} \leq p_{6}$.
Show that this probability is the same as obtained in part (a) above when $p_{3},p_{5} \rightarrow -\infty$ and $p_{4},p_{6} \rightarrow \infty$.

}

\sol{

(a) The probability to measure $p_{x} \in [p_1,p_2]$ in the state $\ket{\psi}$ is given by
\begin{eqnarray}
    \mathcal{P}(p_1 \leq p_{x} \leq p_2) = \bra{\psi} P(p_1 \leq p_{x} \leq p_2) \ket{\psi}
,\end{eqnarray}
where $P(p_1 \leq p_{x} \leq p_{2})$ is the projection operator from the general state space onto the subspace spanned by momentum eigenvectors with momentum $p_{x}$ in the given range.
This is explictly written
\begin{eqnarray}
    P = \int_{p_1}^{p_2} \dd{p_{x}} \int_{-\infty}^{\infty} \dd{p_{y}} \int_{-\infty}^{\infty} \dd{p_{z}} \ket{\psi_{\va*{p}}} \bra{\psi_{\va*{p}}}
,\end{eqnarray}
where $\ket{\psi_{\va*{p}}} = \ket{\psi_{p_{x}}} \otimes \ket{\psi_{p_{y}}} \otimes \ket{\psi_{p_{z}}}$.
We can insert the spatial state completeness into the projector definition as follows
\begin{align}
    P &= \int_{p_1}^{p_2} \dd{p_{x}} \int_{-\infty}^{\infty} \dd{p_{y}} \int_{-\infty}^{\infty} \dd{p_{z}} \id \ket{\psi_{\va*{p}}} \bra{\psi_{\va*{p}}} \id \nonumber \\
    &= \int_{p_1}^{p_2} \dd{p_{x}} \int_{-\infty}^{\infty} \dd{p_{y}} \int_{-\infty}^{\infty} \dd{p_{z}} \int \dd[3]{\va*{r}} \dd[3]{\va*{r}'} \ket{\phi_{\va*{r}}} \frac{1}{(2 \pi \hbar)^{3}} e^{-i \va*{p} \cdot (\va*{r} - \va*{r}') / \hbar} \bra{\phi_{\va*{r}'}} \nonumber \\
    &= \int \dd[3]{\va*{r}} \dd[3]{\va*{r}'} \int_{p_1}^{p_2} \dd{p_{x}} \frac{1}{2 \pi \hbar} e^{-i p_{x}(x - x')/\hbar} \underbrace{ \Bigg[ \int_{-\infty}^{\infty} \frac{\dd{p_{y}}}{2 \pi \hbar} e^{-ip_{y}(y-y')/\hbar} \Bigg] }_{\delta(y-y')} \underbrace{ \Bigg[ \int_{-\infty}^{\infty} \frac{\dd{p_z}}{2\pi\hbar} e^{-ip_{z}(z-z')/\hbar} \Bigg] }_{\delta(z - z')} \ket{\phi_{\va*{r}}} \bra{\phi_{\va*{r}'}} \nonumber \\
    &= \int_{-\infty}^{\infty} \dd{y} \int_{-\infty}^{\infty} \dd{z} \int_{p_1}^{p_2} \dd{p_x} \int_{-\infty}^{\infty} \dd{x} \int_{-\infty}^{\infty} \dd{x'} \frac{1}{\sqrt{2 \pi \hbar}} e^{-ip_{x}x/\hbar} \frac{1}{\sqrt{2 \pi \hbar}} e^{i p_{x} x' / \hbar} \ket{\phi_{x}} \ket{\phi_{y}} \ket{\phi_{z}} \bra{\phi_{x'}} \bra{\phi_{y}} \bra{\phi_{z}}
\end{align}
If we put this into the expression for the measurement probability above, we have
\begin{eqnarray}
    \mathcal{P}(p_1 \leq p_{x} \leq p_{2}) = \int_{-\infty}^{\infty} \dd{y} \int_{-\infty}^{\infty} \dd{z} \int_{p_1}^{p_2} \dd{p_{x}} |\tilde{\psi}(p_{x},y,z)|^2
,\end{eqnarray}
where
\begin{eqnarray}
    \tilde{\psi}(p_{x},y,z) = \int_{-\infty}^{\infty} \frac{\dd{x}}{\sqrt{2 \pi \hbar}} e^{-i p_{x} x / \hbar} \psi(\va*{r})
.\end{eqnarray}

%{ \color{red}
%Let us digress a bit and take a look at the generic integral
%\begin{eqnarray}
%    \int_{-\infty}^{\infty} \dd{x'} \frac{e^{ia(x'-x)}f(x')}{x'-x}
%.\end{eqnarray}
%It is clear that this cannot be done via elementary integration methods, but we can, however, analytically continue the integrand to the complex plane ($x' \rightarrow z$) and then use contour integration methods.
%Assuming for now that $a > 0$
%\begin{eqnarray}
%\begin{aligned}
%    \int_{C} \frac{e^{ia(z-x)} f(z)}{z-x} &= \int_{-\infty}^{x-\epsilon} \dd{x'} \frac{e^{ia(x'-x)}f(x')}{x'-x} + \int_{|z-x|=\epsilon,\phi \in [\pi,0]} \dd{z} \frac{e^{ia(z-x)}f(z)}{z - x} \\
%    &+ \int_{x+\epsilon}^{\infty} \dd{x'} \frac{e^{ia(x'-x)}f(x')}{x'-x} + \int_{\Gamma} \dd{z} \frac{e^{ia(z-x)}f(z)}{z-x}
%,\end{aligned}
%\end{eqnarray}
%where $\Gamma$ is the semi-circle $|z| = R$ with $\phi \in [0,\pi]$, taking $R \rightarrow \infty$.
%This path was chosen such that there are no poles enclosed by $C$ and that the contribution along $\Gamma$ is identically zero.
%Taking $\epsilon \rightarrow 0$, we have the principal value of the integral above
%\begin{eqnarray}
%\begin{aligned}
%    \int_{-\infty}^{\infty} \dd{x'} \frac{e^{ia(x'-x)}f(x')}{x'-x} &= \lim_{\epsilon \rightarrow 0} \int_{|z-x|=\epsilon,\phi \in [0,\pi]} \dd{z} \frac{e^{ia(z-x)}f(z)}{z-x} \\
%    &= \lim_{\epsilon \rightarrow 0} \int_{0}^{\pi} (i \epsilon e^{i\phi} \dd{\phi}) \frac{1}{\epsilon e^{i\phi}} \sum_{n,m=0}^{\infty} \frac{(ia)^{n}f^{(m)}(x)}{n!m!} \epsilon^{n+m} e^{i (n+m) \phi} \\
%    &= \sum_{n=0}^{\infty} \frac{i f^{(n)}(x)}{n!} \lim_{\epsilon \rightarrow 0} \epsilon^{n+m} \Big[ \int_{0}^{\pi} \dd{\phi} e^{i n \phi} \Big] = i \pi f(x)
%.\end{aligned}
%\end{eqnarray}
%Note that if $a < 0$, we would draw the contour with lots of sign flips.
%The contour $C$ would still have the portions along the real axis which constitute the integral we desire when $\epsilon \rightarrow 0$, but the small half-circle would have $\phi \in [\pi,2\pi]$ and $\Gamma$ would be the semi-circle in the lower half-plane.
%Still, there would be no enclosed poles by $C$, and the line integral along $\Gamma$ would contribute nothing, leaving us with
%\begin{eqnarray}
%    \int_{-\infty}^{\infty} \dd{x'} \frac{e^{ia(x'-x)}f(x')}{x'-x} = -\int_{|z-x|=\epsilon,\phi \in [\pi,2\pi]} \dd{z} \frac{e^{ia(z-x)}f(z)}{z-x} = - i \pi f(x)
%.\end{eqnarray}
%Hence, the integral in general is just $i\pi f(x) {\rm sgn}(a)$.
%
%After a very long-winded argument, we now can simplify the measurement probability expression.
%\begin{eqnarray}
%    \mathcal{P}(p_1 \leq p_{x} \leq p_2) = \int \frac{\dd[3]{\va*{r}}}{2 \pi \hbar} \frac{i\pi \hbar [ {\rm sgn}(p_2) - {\rm sgn}(p_1) ]}{i} |\psi(\va*{r})|^2 = \frac{1}{2} \int \dd[3]{\va*{r}} []
%.\end{eqnarray}
%
%}

(b) In this case, we have something quite similar to the above.
If we insert the projection operator for $p_{z}$ onto momentum states with $p_{z} > 0$ and the projection operator for $x$ onto coordinate states with $x_1 \leq x \leq x_2$, we have
\begin{eqnarray}
    P(x_1 \leq x \leq x_2; p_{z} \geq 0) = \int_{x_1}^{x_2} \dd{x} \int_{-\infty}^{\infty} \dd{y} \int_{0}^{\infty} |\tilde{\psi}(x,y,p_{z})|^2
,\end{eqnarray}
where in this case
\begin{eqnarray}
    \tilde{\psi}(x,y,p_{z}) = \int_{-\infty}^{\infty} \frac{\dd{z}}{\sqrt{2 \pi \hbar}} e^{-i p_{z} z / \hbar} \psi(\va*{r})
.\end{eqnarray}
Again, this is just the Fourier transform of the wave function with respect to only the $z$-coordinate.

(c) If we restrict the momenta ranges of $p_{y}$ and $p_{z}$, we then have
\begin{eqnarray}
    \mathcal{P}(\va*{p} \in [p_1,p_2] \times [p_3,p_4] \times [p_5,p_6]) = \int_{p_1}^{p_2} \dd{p_{x}} \int_{p_3}^{p_4} \dd{p_{y}} \int_{p_5}^{p_6} \dd{p_{z}} |\tilde{\psi}(\va*{p})|^2
,\end{eqnarray}
where
\begin{eqnarray}
    \tilde{\psi}(\va*{p}) = \int \frac{\dd[3]{\va*{r}}}{(2 \pi \hbar)^{3/2}} \psi(\va*{r})
.\end{eqnarray}
This is essentially found by stitching the result from part (a) together three times.
Instead of delta-functions in $y$ and $z$, we keep the exponential factors, which in turn give the Fourier transform with respect to $y$ and $z$.
We recover the simplified version of part (a) taking $p_3,p_5 \rightarrow -\infty$ and $p_4,p_6 \rightarrow \infty$.
In this case, we would effectively have the Fourier transform of the wave function with respect to $y$ and $z$ but also immediately the inverse of these.


}


\prob{3 -- Chapter 7 \# 4}{

Consider three non-commuting observables $\hat{A}$, $\hat{B}$, and $\hat{C}$.
Observable $\hat{A}$ is measured yielding the non-degenerate eigenvalue $a$ with corresponding eigenstate $\ket{\psi_{a}}$.
Then, the following two different experiments are carried out.

(a) In the first experiment, before the system has had any time to evolve after the measurement of $\hat{A}$, observable $\hat{C}$ is measured yielding the non-degenerate eigenvalue $c$ with corresponding eigenstate $\ket{\chi_{c}}$.
Calculate the probability $P_{a}(c)$ to measure $c$.

(b) In the second experiment, observables $\hat{B}$ and $\hat{C}$ are measured in rapid succession, so that the system has not had any time to evolve between the measurements of $\hat{A}$ and $\hat{B}$ and those of $\hat{B}$ and $\hat{C}$.
Suppose these measurements yield, respectively, the non-degenerate eigenvalues $b$ and $c$, where $\ket{\phi_{b}}$ is the eigenstate corresponding to the eigenvalue $b$.
Calculate the probability $P_{a}(b,c)$ to measure $b$ and $c$.

(c) Assuming that the eigenvalues of $\hat{B}$ are all non-degenerate, show that
\begin{eqnarray}
    P_{a}(c) = \sum_{b} P_{a}(b,c) + \mathbf{X}
.\end{eqnarray}
Provide an expression for $\mathbf{X}$.
What is then the essential difference between the two experiments?
What conclusions can you draw?

}

\sol{

(a) The probability to measure $c$ is just
\begin{eqnarray}
    P_a(c) = \bra{\psi_{a}} \mathcal{P}_{c} \ket{\psi_{a}} = \bra{\psi_{a}} (\ket{\chi_{c}} \bra{\chi_{c}}) \ket{\psi_{a}} = |\bra{\chi_{c}}\ket{\psi_{a}}|^2
.\end{eqnarray}

(b) The probability to measure $b$ is
\begin{eqnarray}
    P_{a}(b) = \bra{\psi_{a}} \mathcal{P}_{b} \ket{\psi_{a}} = |\bra{\phi_{b}}\ket{\psi_{a}}|^2
.\end{eqnarray}
After this measurement, the system collapses to $\ket{\phi_{b}}$\footnote{It may sound odd to say that the system ``collapses'' to another state. How can a system which is already in an eigenstate collapse to another eigenstate? This is because $\ket{\psi_{a}} = \sum_{b} \ket{\phi_{}} \bra{\phi_{b}}\ket{\psi_{a}}$ since $\ket{\psi_{a}}$ is not an eigenstate of $B$, given that $[A,B] \ne 0$. The eigenstates are of distinct, non-commuting observables, which are not simultaneously diagonalizable.}
The probability to measure $c$ in the state $\ket{\phi_{b}}$ is then just
\begin{eqnarray}
    P_{b}(a) = \bra{\phi_{b}} \mathcal{P}_{c} \ket{\phi_{b}} = |\bra{\chi_{c}}\ket{\phi_{b}}|^2
.\end{eqnarray}
The probability to measure $b$ then $c$ is then just
\begin{eqnarray}
    P_{a}(b,c) = P_{a}(b)P_{b}(c) = |\bra{\phi_{b}}\ket{\psi_{a}}|^2 |\bra{\chi_{c}}\ket{\phi_{b}}|^2 = |\bra{\chi_{c}} \mathcal{P}_{b} \ket{\psi_{a}}|^2
.\end{eqnarray}

(c) If the eigenvalues of $B$ are non-degenerate, we can then write
\begin{eqnarray}
\begin{aligned}
    P_{a}(c) &= |\bra{\chi_{c}} \sum_{b} \mathcal{P}_{b} \ket{\psi_{a}}|^2 = \sum_{b,b'} \bra{\psi_{a}} \mathcal{P}_{b} \ket{\chi_{c}} \bra{\chi_{c}} \mathcal{P}_{b'} \ket{\psi_{a}} \\
             &= \sum_{b=b'} P_{a}(b,c) + \underbrace{ \sum_{b \ne b'} \bra{\psi_{a}} \mathcal{P}_{b} \mathcal{P}_{c} \mathcal{P}_{b'} \ket{\psi_{a}} }_{\mathbf{X}}
.\end{aligned}
\end{eqnarray}
The essential difference is that the measurement of $B$ then $C$ forces the system to assume the eigenstate $\ket{\phi_{b}}$ before measurement of $C$.
In the process where $C$ is measured, the particle is in state $\ket{\psi_{a}}$, which is a superposition of the states $\{ \ket{\phi_{b}} \} $.
We can then imagine repeating second measurement an infinite number of times.
To recover the probability $P_{a}(c)$ in this case, we have to sum over all the events (all possibilite transitions from an eigenstate of $B$ to the final eigenstate with eigenvalue $c$ of $C$).

}


\prob{4 -- Chapter \#7}{

Consider a system with a three-dimensional state space.
The Hamiltonian $\hat{H}$ has a non-degenerate eigenvalue $E_1 = E_0$ with (normalized) eigenstate $\ket{\phi_1}$ and a degenerate eigenvalue $E_2 = 2E_0$ with (orthonormal) eigenstates $\ket{\phi_2}$ and $\ket{\phi_3}$.
Suppose at time $t = 0$ the system is in the normalized state $\ket{\psi(0)}$ given by
\begin{eqnarray}
    \ket{\psi(0)} = \frac{1}{\sqrt{2}} \ket{\phi_1} + \frac{1}{2}( \ket{\phi_2} + \ket{\phi_3} )
.\end{eqnarray}

(a) At $t = 0$ the energy of the system is measured.
What values can be found and with what probabilities?
Calculate $\expval{H}$ and the root-mean-square deviation $\Delta H$ for the system in the state $\ket{\psi(0)}$.

(b) Suppose at $t = 0$, instead of $\hat{H}$, the observable $\hat{A}$, which in the basis $\ket{\phi_1}$, $\ket{\phi_2}$, and $\ket{\phi_3}$ is represented by the following matrix
\begin{eqnarray}
    A = a \begin{pmatrix}
        1 & 0 & 0 \\
        0 & 0 & 1 \\
        0 & 1 & 0
    \end{pmatrix}
,\end{eqnarray}
is measured with $a \in \reals^{+}$.
What results can be found and with what probabilities?

(c) Obtain the state vector $\ket{\psi(t)}$ at time $t$.

(d) In addition to $\hat{A}$ above, consider another observable $\hat{B}$ with respresentation (in the same basis of Hamiltonian eigenstates) given by
\begin{eqnarray}
    B = b \begin{pmatrix}
        0 & 1 & 0 \\
        1 & 0 & 0 \\
        0 & 0 & 1
    \end{pmatrix}
,\end{eqnarray}
where $b \in \reals^{+}$.
What are the mean values $\expval{A(t)}$ and $\expval{B(t)}$?
Any comments?

(e) What results are obtained if $\hat{A}$ is measured at time $t$?
Same question for $\hat{B}$?
Interpret these results.

}

\sol{

(a) The energy $E_1$ has probability
\begin{eqnarray}
    P_{1} = |\bra{\phi_1}\ket{\psi(0)}|^2 = \frac{1}{2}
\end{eqnarray}
to be measured at time $t = 0$, and similarly
\begin{eqnarray}
    P_{2} = \bra{\psi(0)} ( \ket{\phi_2}\bra{\phi_2} + \ket{\phi_3}\bra{\phi_3} ) \ket{\psi(0)} = \frac{1}{4} + \frac{1}{4} = \frac{1}{2}
\end{eqnarray}
is the proability to measure $E_2$, which is as expected since the probability of measuring $E_1$ or $E_2$ must be unity.

The expectation value of the energy in this state is
\begin{eqnarray}
    \expval{H} = \bra{\psi(0)} \Big[ E_1 \ket{\phi_1}\bra{\phi_1} + E_2 ( \ket{\phi_2}\bra{\phi_2} + \ket{\phi_3}\bra{\phi_3} ) \Big] \ket{\psi(0)} = \frac{1}{2} E_1 + \frac{1}{2} E_2 = \frac{3}{2} E_0
.\end{eqnarray}
Similarly,
\begin{eqnarray}
    \expval{H^2} = \frac{1}{2} E_1^2 + \frac{1}{2} E_2^2 = \frac{5}{2} E_0^2
,\end{eqnarray}
which gives
\begin{eqnarray}
    \Delta H = \sqrt{ \expval{H}^2 - \expval{H}^2 } = \sqrt{ \frac{10 - 9}{4} } E_0 = \frac{1}{2} E_0
.\end{eqnarray}

(b) We can write
\begin{eqnarray}
    A = a ( \ket{\phi_1} \bra{\phi_1} + \ket{\phi_2} \bra{\phi_3} + \ket{\phi_3} \bra{\phi_2} )
.\end{eqnarray}
The possible measured values of $A$ are $\pm a$.
Note that the eigenvectors of $A$ corresponding to $a$ are $\ket{\phi_1}$ and $(\ket{\phi_2} + \ket{\phi_3})/\sqrt{2}$ while for $-a$ we have $(\ket{\phi_2} - \ket{\phi_3})/\sqrt{2}$.
The probability to measure $a$ is then
\begin{eqnarray}
\begin{aligned}
    P_{+a} &= \bra{\psi(0)} [ \ket{\phi_1}\bra{\phi_1} + \frac{1}{2}(\ket{\phi_2} + \ket{\phi_3})( \bra{\phi_{2}} + \bra{\phi_3} ) ] \ket{\psi(0)} \\
           &= \frac{1}{2} + \frac{1}{2}(1 + 1)\frac{1}{2} (1 + 1) = 1
\end{aligned}
.\end{eqnarray}
The proability to measure $-a$ is then zero:
\begin{eqnarray}
    P_{-a} = \bra{\psi(0)} \frac{1}{2} ( \ket{\phi_2} - \ket{\phi_3} )( \bra{\phi_2} - \bra{\phi_3} ) \ket{\psi(0)} = \frac{1}{2} \Big[ \frac{1}{2}\frac{1}{2} - \frac{1}{2}\frac{1}{2} - \frac{1}{2}\frac{1}{2} + \frac{1}{2}\frac{1}{2} \Big] = 0
.\end{eqnarray}

(c) Since the Hamiltonian is time-independent, we can write the unitary time evolution operator (with initial time $t = 0$) as
\begin{eqnarray}
    \mathcal{U}(t) = e^{-i H t / \hbar}
,\end{eqnarray}
which gives
\begin{eqnarray}
    \ket{\psi(t)} = \mathcal{U}(t) \ket{\psi(0)} = \frac{1}{\sqrt{2}} e^{-i E_1 t / \hbar} \ket{\phi_1} + \frac{1}{2} e^{-i E_2 t / \hbar} ( \ket{\phi_2} + \ket{\phi_3} )
.\end{eqnarray}

(d) The operator $B$ has eigenvalues $\pm b$ with eigenstates $( \ket{\phi_1} + \ket{\phi_2} )/\sqrt{2}$ and $\ket{\phi_3}$ corresponding to $b$ and eigenstate $( \ket{\phi_1} - \ket{\phi_2} )/\sqrt{2}$ corresponding to $-b$.

The mean value
\begin{eqnarray}
    \expval{A(t)} = \bra{\psi(t)} A \ket{\psi(t)} = \bra{\psi(t)} a \ket{\psi(t)} = a
\end{eqnarray}
This is expected since $\ket{\psi(t)}$ is always in the subspace of the state space spanned by the eigenvectors of $A$ corresponding to eigenvalue $a$.

On the other hand, observe that we can write
\begin{eqnarray}
\begin{aligned}
    \expval{B(t)} &= \bra{\psi(t)} B \ket{\psi(t)} = b \bra{\psi(t)} \Big[ \frac{1}{\sqrt{2}} e^{-i E_1 t /\hbar} \ket{\phi_2} + \frac{1}{2} e^{-i E_2 t /\hbar} ( \ket{\phi_1} + \ket{\phi_3} ) \Big] \\
                  &= b \Bigg[ \frac{1}{2\sqrt{2}} ( e^{i E_2 t /\hbar} e^{-iE_1t/\hbar} + e^{i E_1 t /\hbar} e^{-i E_2 t /\hbar} ) + \frac{1}{4} \Bigg] \\
                  &= \Big[ \frac{1}{4} + \frac{1}{\sqrt{2}} \cos[(E_2-E_1)t/\hbar] \Big] b \\
                  &= \big[ 1 + 2 \sqrt{2} \cos(E_0 t /\hbar) \big] \frac{b}{4}
\end{aligned}
.\end{eqnarray}

(e) The interpretation of $\expval{A(t)} = a$ is just that if one prepares the state $\ket{\psi(0)}$ and allows it to evolve to any aribitrary time $t$ and measures $A$, the result would always be $a$.
Actually, this should be true for subsequent measurements of $A$ too\footnote{This is true because the system collapses to an eigenstate of $a$. \textbf{Question}: Does the system's state change after measurement of $A$? It is already in the subspace spanned by the degenerate eigenstates of $A$. It would not spontaneously assume another state in the same space, would it?}.

Notice that $\expval{B(t)}$ oscillates between $\pm b/4$.
This is an interference phenomenon which is present because $\ket{\psi(t)}$ is a linear combination of the eigenstates of $B$, containing terms correspoding to eigenvalues $\pm b$, respectively.
It is the relative phases of these terms and the fact that the probabilities are square-amplitudes that generates the cross-terms between the eigenstates of $B$, which collectively generate a sinusoidal behavior in the expectation value of $B$ over time.

}


\end{document}
