\def\duedate{\today}
\def\HWnum{3}
\documentclass[10pt,a4paper]{book}

% custom section formatting
\usepackage{titlesec}
\titleformat{\chapter}[display]
{\normalfont\Large\filcenter\sffamily}
{\titlerule[1pt]%
\vspace{1pt}%
\titlerule
\vspace{1pc}%
\LARGE\MakeUppercase{\chaptertitlename} \thechapter}
{1pc}
{\titlerule
\vspace{1pc}%
\Huge}

% appendix handling
\usepackage[toc,page]{appendix}
    
% encoding for file and font
\usepackage[utf8]{inputenc}
\usepackage[T1]{fontenc}

% math formatting/tools
\usepackage{amsmath}
\usepackage{amssymb}
\usepackage{mathtools}
\usepackage[arrowdel]{physics}

% unit formatting
\usepackage{siunitx}
\AtBeginDocument{\RenewCommandCopy\qty\SI}

% figure formatting/tools
\usepackage{graphicx}
\usepackage{float}
\usepackage{subcaption}
\usepackage{multirow}
\usepackage{import}
\usepackage{pdfpages}
\usepackage{transparent}
\usepackage{currfile}

\NewDocumentCommand\incfig{O{1} m}{
    \def\svgwidth{#1\textwidth}
    \import{./Figures/\currfiledir}{#2.pdf_tex}
}

\newcommand{\bef}{\begin{figure}[h!tb]\centering}
\newcommand{\eef}{\end{figure}}

\newcommand{\bet}{\begin{table}[h!tb]\centering}
\newcommand{\eet}{\end{table}}

% hyperlink references 
\usepackage{hyperref}
\hypersetup{
    colorlinks=true,
    linkcolor=blue,
    filecolor=magenta,
    urlcolor=cyan,
    pdftitle={Physics 1 Notes},
    pdfauthor={Richard Whitehill},
    pdfpagemode=FullScreen
}
\urlstyle{same}

\newcommand{\eref}[1]{Eq.~(\ref{eq:#1})}
\newcommand{\erefs}[2]{Eqs.~(\ref{eq:#1})--(\ref{eq:#2})}

\newcommand{\fref}[1]{Fig.~(\ref{fig:#1})}
\newcommand{\frefs}[2]{Fig.~(\ref{fig:#1})--(\ref{fig:#2})}

\newcommand{\aref}[1]{Appendix~(\ref{app:#1})}
\newcommand{\sref}[1]{Section~(\ref{sec:#1})}
\newcommand{\srefs}[2]{Sections~(\ref{sec:#1})-(\ref{sec:#2})}

\newcommand{\tref}[1]{Table~(\ref{tab:#1})}
\newcommand{\trefs}[2]{Table~(\ref{tab:#1})--(\ref{tab:#2})}

% tcolorbox formatting/definitions
\usepackage[most]{tcolorbox}
\usepackage{xcolor}
\usepackage{xifthen}
\usepackage{parskip}

\definecolor{peach}{rgb}{1.0,0.8,0.64}

\DeclareTColorBox[auto counter, number within=chapter]{defbox}{O{}}{
    enhanced,
    boxrule=0pt,
    frame hidden,
    borderline west={4pt}{0pt}{green!50!black},
    colback=green!5,
    before upper=\textbf{Definition \thetcbcounter \ifthenelse{\isempty{#1}}{}{: #1} \\ },
    sharp corners
}

\newcommand*{\eqbox}{\tcboxmath[
    enhanced,
    colback=black!10!white,
    colframe=black,
    sharp corners,
    size=fbox,
    boxsep=8pt,
    boxrule=1pt
]}

\newtcolorbox[auto counter, number within=chapter]{exbox}{
    parbox=false,
    breakable,
    enhanced,
    sharp corners,
    boxrule=1pt,
    colback=white,
    colframe=black,
    before upper= \textbf{Example \thetcbcounter:}\,,
    before lower= \textbf{Solution:}\,,
    segmentation hidden
}

\newtcolorbox{resbox}{
    enhanced,
    colback=black!10!white,
    colframe=black,
    boxrule=1pt,
    boxsep=0pt,
    top=2pt,
    ams nodisplayskip,
    sharp corners
}


\begin{document}

\prob{1 -- Chapter 3 \# 4}{

The time-dependent Schr\"{o}dinger equation of a charged particle in an electromagnetic field reads
\begin{eqnarray}
    i \hbar \pdv{t} \Psi(\va*{r},t) = \Bigg\{ \frac{1}{2m} \Big[ -i \hbar \grad - \frac{q}{c} \va*{A}(\va*{r},t) \Big]^2 + q U(\va*{r},t) \Bigg\} \Psi(\va*{r},t)
\end{eqnarray}
where $U(\va*{r},t)$ and $\va*{A}(\va*{r},t)$ are the (real) scalar and vector potential, respectively, c is the speed of light, and $q$ is the charge of the particle.
Show that the probability density $\rho(\va*{r},t)$ and the probability current density $\va*{j}(\va*{r},t)$ are given in this case by
\begin{align}
    \rho(\va*{r},t) &= | \Psi(\va*{r},t) |^2 \\
    \va*{j}(\va*{r},t) &= \frac{\hbar}{2 m i} \Big[ \Psi^{*}(\va*{r},t) \grad \Psi(\va*{r},t) - \Psi(\va*{r},t) \grad \Psi^{*}(\va*{r},t) \Big] - \frac{q}{mc} \va*{A}(\va*{r},t) | \Psi(\va*{r},t) |^2
\end{align}
with
\begin{eqnarray}
    \pdv{t} \rho(\va*{r},t) + \div{\va*{j}(\va*{r},t)} = 0
.\end{eqnarray}


}

\sol{}


\prob{2 -- Chapter 3 \# 8}{

Consider a particle in a potential $V(\va*{r})$ with associated wave function satisfying the time-dependent Schr\"{o}dinger equation
\begin{eqnarray}
    i \hbar \pdv{t} \Psi(\va*{r},t) = \Big[ -\frac{\hbar^2}{2m} \laplacian + V(\va*{r}) \Big] \Psi(\va*{r},t)
.\end{eqnarray}

(a) Show that 
\begin{eqnarray}
    \dv{t} \expval{\va*{r}(t)} = \int \dd[3]{\va*{r}} \, \va*{j}(\va*{r},t)
,\end{eqnarray}
where $\expval{\va*{r}(t)}$ is the average position of the particle (notation as in notes) and $\va*{j}(\va*{r},t)$ is the probability current density.
Using the definition of $\va*{j}(\va*{r},t)$, show that the equation above can also be written as
\begin{eqnarray}
    m \dv{t} \expval{\va*{r}(t)} = \expval{\va*{p}(t)}
.\end{eqnarray}

(b) Show that 
\begin{eqnarray}
    \begin{aligned}
        \dv{t} \expval{\va*{p}(t)} &= - \expval{\grad{V}} = - \int \dd[3]{\va*{r}} \Psi^{*}(\va*{r},t) [\grad{V(\va*{r})}] \Psi(\va*{r},t) \\
                                   &= \int \dd[3]{\va*{r}} \Psi^{*}(\va*{r},t) \va*{F}(\va*{r}) \Psi(\va*{r},t)
    \end{aligned}
,\end{eqnarray}
where we have introduced the force $\va*{F}(\va*{r})$.

\textbf{Hint}: Consider, say, the $x$-component and, by using the Schr\"{o}dinger equation and its complex conjugate, obtain
\begin{eqnarray}
    \begin{aligned}
        \dv{t} \expval{p_{x}(t)} &= -\frac{\hbar^2}{2m} \int \dd[3]{\va*{r}} \Bigg[ (\laplacian{\Psi^{*}}) \pdv{\Psi}{x} - \Psi^{*} \laplacian{\pdv{\Psi}{x}} \Bigg] \\
                                 &+ \int \dd[3]{\va*{r}} \Bigg[ V \Psi^{*} \pdv{\Psi}{x} - \Psi^{*} \pdv{(V \Psi)}{x} \Bigg]
    ,\end{aligned}
\end{eqnarray}
where the dependence on $\va*{r}$ and $t$ on the r.h.s. has been suppressed for brevity.
Next examine these two terms.

(c) The equation above looks like Newton's second law but for average values.
As a matter of fact if $\expval{\va*{F}} = \va*{F}(\expval{\va*{r}})$, then $\expval{\va*{r}(t)}$ changes in time as the position of a classical particle under the action of the force $\va*{F}(\va*{r})$.
Under what condition(s) can this happen?
Obtain $\expval{\va*{r}(t)}$ and $\expval{\va*{p}(t)}$ for a particle in a harmonic potential
\begin{eqnarray}
   V(\va*{r}) = \frac{m \omega^2}{2} \va*{r}^2
.\end{eqnarray}


}

\sol{}


\prob{3 -- Chapter 4 \# 1}{

Consider the problem of a particle in an attractive $\delta$-function potential given by
\begin{eqnarray}
   V(x) = -V_0 \delta(x) \quad V_0 > 0
.\end{eqnarray}

(a) Obtain the energy and wave-function of the bound state.
Sketch the wave function and provide an estimate for $\Delta x$.

(b) Calculate the probability $\dd{P}(p)$ that a measurement of the momentum in this bound state will give a result included between $p$ and $p + \dd{p}$.
For what value of $p$ is this probability largest?
Provide an estimate for $\Delta p$ and an order of magnitude for $\Delta x \Delta p$.

}

\sol{}


\prob{4 -- Chapter 4 \# 5}{

Consider a particle in the one-dimensional potential $V(x)$, such that $V(x) = \infty$ for $x < 0$ and
\begin{eqnarray}
   V(x) = - V_0 \, \delta(x-a) ~{\rm for}~ x > 0
\end{eqnarray}
where $V_0 > 0$.
Determine whether this potential admits any bound states.

}

\sol{}

\end{document}
