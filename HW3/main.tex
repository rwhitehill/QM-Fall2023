\def\duedate{\today}
\def\HWnum{3}
% Document setup
\documentclass[12pt]{article}
\usepackage[margin=1in]{geometry}
\usepackage{fancyhdr}
\usepackage{lastpage}

\pagestyle{fancy}
\lhead{Richard Whitehill}
\chead{PHYS 621 -- HW \HWnum}
\rhead{\duedate}
\cfoot{\thepage \hspace{1pt} of \pageref{LastPage}}

% Encoding
\usepackage[utf8]{inputenc}
\usepackage[T1]{fontenc}

% Math/Physics Packages
\usepackage{amsmath}
\usepackage{amssymb}
\usepackage{mathtools}
\usepackage[arrowdel]{physics}
\usepackage{siunitx}

\AtBeginDocument{\RenewCommandCopy\qty\SI}

% Reference Style
\usepackage{hyperref}
\hypersetup{
    colorlinks=true,
    linkcolor=blue,
    filecolor=magenta,
    urlcolor=cyan,
    citecolor=green
}

\newcommand{\eref}[1]{Eq.~(\ref{eq:#1})}
\newcommand{\erefs}[2]{Eqs.~(\ref{eq:#1})--(\ref{eq:#2})}

\newcommand{\fref}[1]{Fig.~\ref{fig:#1}}
\newcommand{\frefs}[2]{Figs.~\ref{fig:#1}--\ref{fig:#2}}

\newcommand{\tref}[1]{Table~\ref{tab:#1}}
\newcommand{\trefs}[2]{Tables~\ref{tab:#1}-\ref{tab:#2}}

% Figures and Tables 
\usepackage{graphicx}
\usepackage{float}

\newcommand{\bef}{\begin{figure}[h!]\begin{center}}
\newcommand{\eef}{\end{center}\end{figure}}

\newcommand{\bet}{\begin{table}[h!]\begin{center}}
\newcommand{\eet}{\end{center}\end{table}}

% tikz
\usepackage{tikz}
\usetikzlibrary{calc}
\usetikzlibrary{decorations.pathmorphing}
\usetikzlibrary{decorations.markings}
\usetikzlibrary{arrows.meta}
\usetikzlibrary{positioning}

% tcolorbox
\usepackage[most]{tcolorbox}
\usepackage{xcolor}
\usepackage{xifthen}
\usepackage{parskip}

\newcommand*{\eqbox}{\tcboxmath[
    enhanced,
    colback=black!10!white,
    colframe=black,
    sharp corners,
    size=fbox,
    boxsep=8pt,
    boxrule=1pt
]}

% problem-solution macros
% \usepackage{adjustbox}
\usepackage{changepage}

\newtcolorbox{probbox}[1][]{
    breakable,
    enhanced,
    boxrule=0pt,
    frame hidden,
    borderline west={4pt}{0pt}{green!50!black},
    colback=green!5,
    before upper=\textbf{Problem #1) \,},
    % \textbf{Problem #1 \ifthenelse{\isempty{#1}}{}{: #1} \\ },
    sharp corners
}

% \newtcolorbox{ProblemBox}[1][]{%
%   breakable,
%   enhanced,
%   colback=black!10!white,
%   colframe=black,
%   title={\large #1 \hfill}
% }
\newcommand{\prob}[2]{
\begin{probbox}[#1]
#2
\end{probbox}
}

\newenvironment{solution}{\begin{adjustwidth}{8pt}{8pt}}{\end{adjustwidth}}
\newcommand{\sol}[1]{
\begin{solution}
#1
\end{solution}
}
% \textbf{#1)} #2}

% Miscellaneous Definitions/Settings
\newcommand{\reals}{\mathbb{R}}
\newcommand{\integers}{\mathbb{Z}}
\newcommand{\naturals}{\mathbb{N}}
\newcommand{\rationals}{\mathbb{Q}}
\newcommand{\complexs}{\mathbb{C}}

\setlength{\parskip}{\baselineskip}
\setlength{\parindent}{0pt}
\setlength{\headheight}{14.49998pt}
\addtolength{\topmargin}{-2.49998pt}


\begin{document}

\prob{1 -- Chapter 3 \# 4}{

The time-dependent Schr\"{o}dinger equation of a charged particle in an electromagnetic field reads
\begin{eqnarray}
    i \hbar \pdv{t} \Psi(\va*{r},t) = \Bigg\{ \frac{1}{2m} \Big[ -i \hbar \grad - \frac{q}{c} \va*{A}(\va*{r},t) \Big]^2 + q U(\va*{r},t) \Bigg\} \Psi(\va*{r},t)
\end{eqnarray}
where $U(\va*{r},t)$ and $\va*{A}(\va*{r},t)$ are the (real) scalar and vector potential, respectively, c is the speed of light, and $q$ is the charge of the particle.
Show that the probability density $\rho(\va*{r},t)$ and the probability current density $\va*{j}(\va*{r},t)$ are given in this case by
\begin{align}
    \rho(\va*{r},t) &= | \Psi(\va*{r},t) |^2 \\
    \va*{j}(\va*{r},t) &= \frac{\hbar}{2 m i} \Big[ \Psi^{*}(\va*{r},t) \grad \Psi(\va*{r},t) - \Psi(\va*{r},t) \grad \Psi^{*}(\va*{r},t) \Big] - \frac{q}{mc} \va*{A}(\va*{r},t) | \Psi(\va*{r},t) |^2
\end{align}
with
\begin{eqnarray}
    \pdv{t} \rho(\va*{r},t) + \div{\va*{j}(\va*{r},t)} = 0
.\end{eqnarray}


}

\sol{

    As in our previous derivations of $\rho$ and $\va*{j}$ we define $\eqbox{ \rho(\va*{r},t) = |\Psi|^2}$, and $\va*{j}$ such that $\pdv{\rho}{t} = - \div{\va*{j}}$.
Taking the time derivative of $\rho$, we have
\begin{eqnarray}
    \label{eq:t-deriv-rho}
    \pdv{\rho}{t} = \Psi^{*} \pdv{\Psi}{t} + \Psi \pdv{\Psi^{*}}{t}
.\end{eqnarray}
Again, we get the time-derivative of the wave function from S.E. (which requires a bit more massaging than in the previous cases):
\begin{eqnarray}
    \begin{aligned}
        \pdv{\Psi}{t} &= \frac{1}{2 m i \hbar} \Bigg[ -\hbar^2 \laplacian + \frac{i \hbar q}{c} \div{\va*{A}} + \frac{i \hbar q}{c} \va*{A} \cdot \grad + \frac{q^2}{c^2} \va*{A}^2 \Bigg] \Psi + \frac{q}{i \hbar} U \Psi \\
                      &= -\frac{\hbar}{2mi} \laplacian{\Psi} + \frac{q}{2mc} \underbrace{ \Big[ \Psi \div{\va*{A}} + \va*{A} \cdot \grad{\Psi} \Big] }_{\div{\va*{A} \Psi}} + \frac{q^2}{2 m i \hbar c} \va*{A}^2 \Psi + \frac{q}{i \hbar} U \Psi
    \end{aligned}
.\end{eqnarray}
Observe that the last two terms are purely imaginary, and therefore cancel in \eref{t-deriv-rho}, leaving us with\footnote{We use the fact that $\div{\va*{A}|\Psi|^2} = |\Psi|^2 \div{\va*{A}} + \va*{A} \cdot \grad{|\Psi|^2} = |\Psi|^2 \div{\va*{A}} + \va*{A} \cdot ( \Psi^{*} \grad{\Psi} + \Psi \grad{\Psi^{*}} )$ and $\Psi^{*} \div{\va*{A} \Psi} = \div{\va*{A} |\Psi|^2} - \Psi \va*{A} \cdot \grad{\Psi^{*}}$.}
\begin{eqnarray}
    \eqbox{
    \begin{aligned}
        \pdv{\rho}{t} &= - \Bigg\{ \grad \cdot \frac{\hbar}{2mi} \Big[ \Psi^{*} \grad{\Psi} - \Psi \grad{\Psi^{*}} \Big] + \frac{q}{2mc} \Big[ \Psi^{*} \div{\va*{A} \Psi} - \Psi \div{\va*{A} \Psi^{*}} \Big]  \Bigg\} \\
                      &= - \grad \cdot \underbrace{ \Bigg\{ \Big[ \Psi^{*} \grad{\Psi} - \Psi \grad{\Psi^{*}} \Big] + \frac{q}{mc} \va*{A} |\Psi|^2 \Bigg\} }_{\va*{j}(\va*{r},t)}
    \end{aligned}
}
.\end{eqnarray}
It is then manifestly obvious that
\begin{eqnarray}
    \eqbox{ \pdv{\rho}{t} + \div{\va*{j}} = 0 } 
.\end{eqnarray}

}


\prob{2 -- Chapter 3 \# 8}{

Consider a particle in a potential $V(\va*{r})$ with associated wave function satisfying the time-dependent Schr\"{o}dinger equation
\begin{eqnarray}
    i \hbar \pdv{t} \Psi(\va*{r},t) = \Big[ -\frac{\hbar^2}{2m} \laplacian + V(\va*{r}) \Big] \Psi(\va*{r},t)
.\end{eqnarray}

(a) Show that 
\begin{eqnarray}
    \dv{t} \expval{\va*{r}(t)} = \int \dd[3]{\va*{r}} \, \va*{j}(\va*{r},t)
,\end{eqnarray}
where $\expval{\va*{r}(t)}$ is the average position of the particle (notation as in notes) and $\va*{j}(\va*{r},t)$ is the probability current density.
Using the definition of $\va*{j}(\va*{r},t)$, show that the equation above can also be written as
\begin{eqnarray}
    m \dv{t} \expval{\va*{r}(t)} = \expval{\va*{p}(t)}
.\end{eqnarray}

(b) Show that 
\begin{eqnarray}
    \begin{aligned}
        \dv{t} \expval{\va*{p}(t)} &= - \expval{\grad{V}} = - \int \dd[3]{\va*{r}} \Psi^{*}(\va*{r},t) [\grad{V(\va*{r})}] \Psi(\va*{r},t) \\
                                   &= \int \dd[3]{\va*{r}} \Psi^{*}(\va*{r},t) \va*{F}(\va*{r}) \Psi(\va*{r},t)
    ,\end{aligned}
\end{eqnarray}
where we have introduced the force $\va*{F}(\va*{r})$.

\textbf{Hint}: Consider, say, the $x$-component and, by using the Schr\"{o}dinger equation and its complex conjugate, obtain
\begin{eqnarray}
    \begin{aligned}
        \dv{t} \expval{p_{x}(t)} &= -\frac{\hbar^2}{2m} \int \dd[3]{\va*{r}} \Bigg[ (\laplacian{\Psi^{*}}) \pdv{\Psi}{x} - \Psi^{*} \laplacian{\pdv{\Psi}{x}} \Bigg] \\
                                 &+ \int \dd[3]{\va*{r}} \Bigg[ V \Psi^{*} \pdv{\Psi}{x} - \Psi^{*} \pdv{(V \Psi)}{x} \Bigg]
    ,\end{aligned}
\end{eqnarray}
where the dependence on $\va*{r}$ and $t$ on the r.h.s. has been suppressed for brevity.
Next examine these two terms.

(c) The equation above looks like Newton's second law but for average values.
As a matter of fact if $\expval{\va*{F}} = \va*{F}(\expval{\va*{r}})$, then $\expval{\va*{r}(t)}$ changes in time as the position of a classical particle under the action of the force $\va*{F}(\va*{r})$.
Under what condition(s) can this happen?
Obtain $\expval{\va*{r}(t)}$ and $\expval{\va*{p}(t)}$ for a particle in a harmonic potential
\begin{eqnarray}
   V(\va*{r}) = \frac{m \omega^2}{2} \va*{r}^2
.\end{eqnarray}


}

\sol{

(a) Observe that
\begin{eqnarray}
    \label{eq:prob2a-1}
    \eqbox{ \dv{t} \expval{\va*{r}(t)} = \int \dd[3]{\va*{r}} \pdv{t} \rho(\va*{r},t) = - \int \dd[3]{\va*{r}} \va*{j}(\va*{r},t) }
,\end{eqnarray}
where we have used the continuity equation for the probability density and current density.

Using $\va*{j} = (\hbar/2mi)[\Psi^{*} \grad{\Psi} - \Psi \grad{\Psi^{*}}]$, we can also rewrite \eref{prob2a-1} as
\begin{eqnarray}
    \label{eq:prob2a-2}
    \eqbox{ m \dv{t} \expval{\va*{r}(t)} = \frac{1}{2} \int \dd[3]{\va*{r}} [ \Psi \va*{p} \Psi^{*} + \Psi^{*} \va*{p} \Psi ] = \int \dd[3]{\va*{r}} \Psi^{*} \va*{p} \Psi = \expval{\va*{p}(t)} }
.\end{eqnarray}
Note that we have used the fact that $\va*{p}$ is hermitian to rewrite $\Psi \va*{p} \Psi^{*} = \Psi^{*} \va*{p} \Psi$ under the integral sign.

(b) We can do this as follows:
\begin{eqnarray}
    \begin{aligned}
        \dv{t} \expval{\va*{p}(t)} &= -i \hbar \int \dd[3]{\va*{r}} \pdv{t} \Psi^{*} \grad \Psi \\
                                   &= -i\hbar \int \dd[3]{\va*{r}} \Bigg[ \pdv{\Psi^{*}}{t} \grad \Psi + \Psi^{*} \grad \pdv{\Psi}{t} \Bigg] \\
                                   &= -i \hbar \int \dd[3]{\va*{r}} \Bigg[ -\frac{1}{i \hbar} \Big( -\frac{\hbar^2}{2m} \laplacian \Psi^{*} + V \Psi^{*} \Big) \grad \Psi + \frac{1}{i \hbar} \Psi^{*} \grad \Big( \frac{-\hbar^2}{2m} \laplacian \Psi + V \Psi \Big) \Bigg] \\
                                   &= \int \dd[3]{\va*{r}} \Bigg\{ -\frac{\hbar^2}{2m} \Big[ (\laplacian \Psi^{*})(\grad \Psi) - \Psi^{*} \grad (\laplacian \Psi) \Big] + \Big[ V \Psi^{*} \grad \Psi - \Psi^{*} \grad (V \Psi) \Big] \Bigg\} \\
                                   &= \int \dd[3]{\va*{r}} ~ ? + \expval{- \grad V}
    .\end{aligned}
\end{eqnarray}


}


\prob{3 -- Chapter 4 \# 1}{

Consider the problem of a particle in an attractive $\delta$-function potential given by
\begin{eqnarray}
   V(x) = -V_0 \delta(x) \quad V_0 > 0
.\end{eqnarray}

(a) Obtain the energy and wave-function of the bound state.
Sketch the wave function and provide an estimate for $\Delta x$.

(b) Calculate the probability $\dd{P}(p)$ that a measurement of the momentum in this bound state will give a result included between $p$ and $p + \dd{p}$.
For what value of $p$ is this probability largest?
Provide an estimate for $\Delta p$ and an order of magnitude for $\Delta x \Delta p$.

}

\sol{}


\prob{4 -- Chapter 4 \# 5}{

Consider a particle in the one-dimensional potential $V(x)$, such that $V(x) = \infty$ for $x < 0$ and
\begin{eqnarray}
   V(x) = - V_0 \, \delta(x-a) ~{\rm for}~ x > 0
\end{eqnarray}
where $V_0 > 0$.
Determine whether this potential admits any bound states.

}

\sol{}

\end{document}
