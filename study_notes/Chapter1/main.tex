\chapter{The Failure of Classical Physics}

The following is a quick summary of some of the phenomena which classical mechanics and electromagnetic theory could not properly model and explain.

\section{Black-body radiation}

The primary quantity of interest here is the energy density of some box which is kept at temperature $T$, denoted as $u(\nu,T)$.
The total energy that strikes an area $A$ of the wall of the box in time $t$ is
\begin{eqnarray}
    \int_{0}^{2 \pi} \dd{\phi} \int_{0}^{\pi/2} \dd{\theta} \sin{\theta} \int_{0}^{ct} r^2 \frac{A \cos{\theta}}{4 \pi r^2} u(\nu,T) \dd{\nu} = \frac{c t A}{4} u(\nu,T) \dd{\nu}
.\end{eqnarray}
If the box absorbs some fraction $f(\nu,T)$ of this energy, then the total energy absorbed by the box at temperature $T$ from light at frequency $\nu$ is
\begin{eqnarray}
    E(\nu,T) = \frac{c}{4} f(\nu,T) u(\nu,T)
.\end{eqnarray}
A body is called ``black'' if $f \equiv 1$, meaning that all light is perfectly absorbed.

\subsection{Classical treatment}

We can solve Maxwell's equations (no sources!) inside the box with periodic boundary conditions, which gives
\begin{eqnarray}
    \pdv[2]{\tilde{\bf{E}}}{t} + c^2 k^2 \tilde{\bf{E}} = 0
,\end{eqnarray}
where $\bf{\tilde{E}}(k)$ are the Fourier expansion coefficients of the electric field such that
\begin{eqnarray}
    \va*{E} = \sum_{k} e^{i \va*{k} \cdot \va*{r}} \bf{\tilde{E}}(k)
.\end{eqnarray}
A similar equation holds for the magnetic field.
We can thus model the electromagnetic (EM) radiation by an infinite set of uncoupled harmonic oscillators.

Next, we find the number of modes in a small volume of the reciprocal space to be 
\begin{eqnarray}
    \rho(\va*{k}) \dd[3]{\va*{k}} = 2 \frac{V}{(2\pi)^3} \dd{\va*{k}}
.\end{eqnarray}
The factor of two comes from the fact that there are two independent components in $\tilde{E}$ (in solving MEs, there is a condition that $\va*{E}$ be perpendicular to $\va*{k}$), and the second factor comes from the volume per mode in the reciprocal space.

Using the equipartition theorem (which states that $U = \frac{1}{2} k_{B} T$ is the energy contribution from each quadratic degree of freedom in the Hamiltonian -- of which there are two for a given harmonic oscillator), the relation $\dd[3]{\va*{k}} = 4 \pi k^2 \dd{k}$ (assuming angular symmetry), and the dispersion relation $c = \nu \lambda$ ($k = 2\pi/\lambda$), we recover the Rayleigh-Jeans law for the energy density of a blackbody:
\begin{eqnarray}
    \eqbox{ \frac{(k_{B} T / 2 ) \rho(k) \dd{k}}{V} = \underbrace{ 8 \pi \frac{k_{B} T}{c^3} \nu^2 }_{u(\nu,T)} \dd{\nu} }
,\end{eqnarray}
Clearly the energy density diverges to $\infty$ for more energetic light (which is where the term ``ultraviolet catastrophe'' originates).

\subsection{Quantum treatment}

The quantum nature comes from using Einstein's formula for the energy of a photon of light at frequency $\nu$: $E = n h \nu$ ($n = 0,1,2,\ldots$).
We can use the partition function to derive the fact that the average energy of a harmonic oscillator is
\begin{eqnarray}
    \expval{E} = \frac{h \nu}{e^{\beta h \nu} - 1}
,\end{eqnarray}
where $\beta = 1/k_{B}T$.
Using this fact instead of the equipartition theorem, we find
\begin{eqnarray}
    \eqbox{ u(\nu,T) = \frac{8 \pi h}{c^3} \frac{\nu^3}{e^{\beta h \nu} - 1} }
.\end{eqnarray}
Notice that this resolves the ultraviolet catastrophe since the energy density now is bounded at all frequencies of light, peaking at some $\nu_0$ (is there a well-known formula for this?).


\section{Photo-electric effect}

The photo-electric effect is the phenomenon that if light is shined on a metal and set up a potential difference between the metal and some other metallic plate, say, that there is a current between the plates (in some circumstances).

\subsection{Classical Treatment}

We can treat an electron at the surface of the metal, which absorbs the light, to be a harmonic oscillator driven by the light (i.e. the force that the electric field of the light wave exerts on the electron).
That is,
\begin{eqnarray}
    m \ddot{x}(t) + k x(t) = -e \mathcal{E}_{0} \cos(\omega t)
.\end{eqnarray}
Solving gives,
\begin{eqnarray}
    x(t) = x_0 \cos(\omega_0 t + \phi_0) -\frac{e \mathcal{E}_{0}}{m} \frac{\cos(\omega t)}{\omega_0^2 - \omega^2}
,\end{eqnarray}
where $\omega_0^2 = k/m$, which depends on the metal in consideration.
The first term is just the homogeneous solution of the equation with $x_0$ and $\phi_0$ determined by initial conditions.
Notice that the energy of the electron is (considering only the contribution associated with the transient term -- i.e. that from the driving force)
\begin{eqnarray}
    E(t) = \frac{1}{2} m \dot{x}^2 + \frac{k}{2} x \approx \frac{e^2 \mathcal{E}_{0}^2}{2m} \frac{1}{(\omega_0^2 - \omega^2)^2} [ \omega^2 \sin^2(\omega t) + \omega_0^2 \cos^2(\omega t) ]
.\end{eqnarray}
The average energy over one period of oscillation $T = 2 \pi / \omega$  is just
\begin{eqnarray}
    \expval{E} = \frac{e^2 \mathcal{E}_0^2}{4 m} \frac{\omega_0^2 + \omega^2}{(\omega_0^2 - \omega^2)^2}
.\end{eqnarray}
Recall that the intensity of light is proportional to $\mathcal{E}_{0}^2$, so the average energy of the electron is predicted to be proportional to the intensity of the light shone on the metal.
Above some threshold energy, we should observe a current, and by increasing the intensity of the light, we would expect classically that the current is proportional in some way (would have to work out the proportionality) to the intensity.
In fact, for any color of light, we could observe a current if we have intense enough light.

\subsection{Quantum Treatment}

The classical treatment is inconsistent with experimental observations.
In reality, it is found that no matter how intense we make the light, some wavelengths are simply incapable of generating any current.
Again, in this case, we quantize the energy of a photon as $E = h \nu$.
For a metal with work function $W = h \nu_0$, where $\nu_0$ is a convenient parameter with which to compare the frequency of the light, the kinetic energy of an electron which is freed from the metal by this photon is
\begin{eqnarray}
    K = E - W = h (\nu - \nu_0)
.\end{eqnarray}
Clearly then, we have to have $\nu > \nu_0$ at the very least to observe a current.
If we satisfy this condition, then increasing the intensity of light -- which corresponds to increasing the number of photons -- does in fact lead to larger observed currents.


\section{Compton scattering}

In classical mechanics, treating light as a wave, if we shine light on a charged particle, say an electron, we would find that the electron is excited by the light similar to the above argument.
In quantum mechanics, however, treating the photon as a particle, the light has momentum proportional to $p = h/\lambda = h \nu / c$, and the photon interacts ``concretely'' with the electron by exchanging momentum and scattering at some angle $\theta$ relative to the axis of the incoming light.
Quantitatively, we have the Compton relation
\begin{eqnarray}
    \Delta \lambda = \frac{h}{m c} (1 - \cos{\theta})
.\end{eqnarray}
Note that $h/mc$ is known as the Compton wavelength of the electron.

We can derive this relation as follows.
First, let us work in the rest frame of the electron, and the photon approach the electron with momentum $\va*{p}_{\gamma} = h/\lambda \vu*{x}$.
After the collision, the photon has momentum $\va*{p}'_{\gamma} = \frac{h}{\lambda'} \big[ \sin{\theta} \vu*{x} + \cos{\theta} \vu*{y} \big]$ while the electron has momentum $\va*{p}'_{e}$.
Note that energy conservation gives
\begin{eqnarray}
    p_{\gamma} c + m c^2 = p_{\gamma}' c + \sqrt{m^2 c^4 + p_{e}'^2 c^2}
.\end{eqnarray}
If we subtract $p'_{\gamma} c$  from both sides and square, we have
\begin{gather}
    [ (p_{\gamma} - p'_{\gamma}) c + m c^2 ]^2 = m^2 c^4 + |\va*{p}_{\gamma}' - \va*{p}_{\gamma}|^2 c^2 \nonumber \\
    (p_{\gamma}' - p_{\gamma})^2 + 2(p_{\gamma} - p_{\gamma}') m c = p_{\gamma}'^2 + p_{\gamma}^2 - 2 p_{\gamma}' p_{\gamma} \cos{\theta} \nonumber \\
    \frac{1}{p_{\gamma}'} - \frac{1}{p_{\gamma}} = \frac{1 - \cos{\theta}}{mc} \nonumber \\
    \lambda' - \lambda = \Delta \lambda = \frac{h}{mc} (1 - \cos{\theta})
.\end{gather}



\section{Atom stability and spectral lines}

In the homework, we completed a problem showing that based on classical mechanics an electron should collapses into the nucleus of an atom very quickly (because of the EM radiation from its acceleration).
Clearly this is in contrast with the stability of the universe in general, and can be broached with the introduction of a few different models, although the Schr\"{o}dinger equation is the most correct treatment and is the subject of much of the topic of non-relativistic QM.

There is also the issue of the spectral lines from atoms, which are distinct wavelengths of light emitted and absorbed by atoms.
Note that the emission and absorption spectra are identical (corresponding to the energy levels of an atom).
This observation is in stark contradiction to the classical expectation, which is that the spectra are continuous.
For emission, the electron radiates light of all frequencies within a certain range, given by its acceleration at any time.
On the other hand, there is no classical restriction on which frequencies an electron is permitted to interact with, so it absorbs any light happily.






