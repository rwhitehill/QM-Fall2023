\chapter{Scattering in One Dimension}

\section{Scattering in a $\delta$-function potential}

Consider a repulsive $\delta$-function potential of the form $v(x) = v_0 \delta(x)$.
We are concerned here with the so-called scattering states, which exist for $\epsilon > 0$.
Recall that we have two linearly independent solutions for such an energy regime since the energy is unbounded for $x \rightarrow \pm \infty$.
We can write them as follows
\begin{eqnarray}
    \psi_{k}^{(1)}(x) = \begin{cases}
        e^{ikx} + A^{(1)} e^{-ikx} & x < 0 \\
        B^{(1)} e^{ikx} & x > 0
    \end{cases}
    ~{\rm and}~
    \psi_{k}^{(2)}(x) = \begin{cases}
        B^{(2)} e^{-ikx} & x < 0 \\
        e^{-ikx} + A^{(2)} e^{ikx} & x > 0
    .\end{cases}
\end{eqnarray}
A general solution is obtained by taking a linear combination $\psi_{k}(x) = \alpha_{k} \psi_{k}^{(1)}(x) + \beta_{k} \psi_{k}^{(2)}(x)$, but for our purposes we will stick to analyzing the first since this corresponds to particles coming in from the left at $x = -\infty$, interacting with the $\delta$-function potential at $x = 0$ and either reflecting back to $x = -\infty$ or being transmitted and continuing to $x = \infty$.
Note that this is quite distinct from the classical expectation, where the potential here acts as essentially a very hard wall and sends the particle back elastically without fail.

The BCs at $x = 0$ give (dropping the superscript for brevity)
\begin{gather}
    1 + A = B \\
    ik[B - (1 - A)] = v_0 B
.\end{gather}
Solving gives
\begin{align}
    A &= \frac{v_0}{-v_0 + 2ik} \\
    B &= \frac{2ik}{-v_0 + 2ik}
.\end{align}

At this point, we motivate how to compute the reflection and transmission probabilities by considering a wave-packet of the form\footnote{Usually there is a factor $1/\sqrt{2\pi}$ because of the wave-packet's connection to the Fourier transform, but we absorb it into $g(k)$ because it would only be brought along for the ride and has no bearing on the physics.}
\begin{align}
    \Psi(x,t) &= \int \dd{k} g(k) \psi_{k}(x) e^{-i\omega t} = \begin{cases}
        \int \dd{k} g(k) e^{i(kx - \omega t)} + \int \dd{k} g(k) A(k) e^{-i(kx + \omega t)} & x < 0 \\
        \int \dd{k} g(k) B(k) e^{i(kx - \omega t)} & x > 0
    \end{cases} \nonumber \\
    &= \begin{cases}
        \Psi_{I}(x,t) + \Psi_{R}(x,t) & x < 0 \\
        \Psi_{T}(x,t) & x > 0
    ,\end{cases}
\end{align}
where $\Psi_{I,R,T}$ are the incident, reflected, and transmitted wave-packets, $\omega = \hbar k^2/2m$, and as usual $g(k)$ is strongly localized around some $k = k_0$.
Using the stationary phase method of the second chapter, the incident wave packet
\begin{eqnarray}
    \Psi_{I}(x,t) \approx e^{i(k_0 x - \omega_0 t)} \int \dd{k} g(k) e^{i[x - x_{I}(t)][k - k_0]}
,\end{eqnarray}
where $\dv{k} [kx - \omega t]_{x = x_{I}(t)} = 0$ or $x_{I}(t) = (\hbar k / m) t$.
Observe from this that the incident wave packet's center reaches the potential at $t = 0$, and for $t \gg 0$, $x_{I}(t) \gg 0$ and the exponential is strongly oscillatory, implying that $|\Psi_{I}(x,t)| \approx |\Psi(x - x_{I}(t),0)| \approx 0$.

Next, we analyze the reflected wave packet:
\begin{eqnarray}
    \Psi_{R}(x,t) = \int \dd{k} g(k) |A(k)| e^{i(kx + \omega t + \alpha(k))}
.\end{eqnarray}
We have written
\begin{eqnarray}
    A(k) = |A(k)|e^{i\alpha(k)}
,\end{eqnarray}
where
\begin{eqnarray}
    |A(k)| = \frac{v_0}{\sqrt{ v_0^2 + 4 k^2 }} ~{\rm and}~ \alpha(k) = \arctan( \frac{2k}{v_0} )
.\end{eqnarray}
Thus, the stationary phase method gives
\begin{eqnarray}
    \Psi_{R}(x,t) \approx |A(k_0)| e^{i(k_0 x + \omega_0 t - \alpha(k_0))} \int \dd{k} g(k) e^{i(k - k_0)(x - x_{R}(t))}
,\end{eqnarray}
where
\begin{eqnarray}
    x_{R}(t) = - \frac{\hbar k_0}{m} [t - \tau]
\end{eqnarray}
with
\begin{eqnarray}
    \tau = \frac{m}{\hbar k_0} \alpha'(k_0) = \frac{m}{\hbar k_0} \frac{2 v_0}{v_0^2 + 4k^2}
.\end{eqnarray}
This is interpreted as the time-delay between the interaction with the potential and reflection.

Finally, we analyze the transmitted wave packet:
\begin{eqnarray}
    \Psi_{T}(x,t) \approx |B(k_0)| e^{i(k_0x - \omega_0 + \beta(k))} \int \dd{k} g(k) e^{i(k - k_0)(x - x_{T}(t))}
,\end{eqnarray}
where $|B(k_0)| = 4k^2/\sqrt{v_0^2 + 4k^2}$ and $\beta(k) = \arctan(\frac{v_0}{2k})$, which gives
\begin{eqnarray}
    x_{T}(t) = \frac{\hbar k_0}{m} [t - \tau ]
\end{eqnarray}
with
\begin{eqnarray}
    \tau = \frac{m}{\hbar k_0} \frac{2v_0}{v_0^2 + 4k^2}
,\end{eqnarray}
which is the same as for the reflected packet.

Let us define
\begin{eqnarray}
    G(x) = \int g(k) e^{i(k-k_0)x}
.\end{eqnarray}
This allows us to write the wave-packet as
\begin{align}
    \Psi(x,t) \approx \begin{cases}
        e^{i(k_0 x - \omega_0 t)} G(x - x_{I}(t)) + A(k_0) e^{-i(k_0x + \omega_0 t)} G(x - x_{R}(t)) & x < 0 \\
        B(k_0) e^{i(kx - \omega_0 t)} G(x - x_{T}(t)) & x > 0
    .\end{cases}
\end{align}

Observe that for $|x| \gg 0$, $G(x) \approx 0$.
For times much before the interaction $t \rightarrow -\infty$, $\Psi(x,t) \approx \Psi_{I}(x,t) \theta(-x)$, and for times much after the interaction $t \rightarrow \infty$, $\Psi(x,t) \approx \Psi_{R}(x,t)\theta(-x) + \Psi_{T}(x,t)\theta(x)$.
The conservation of proability gives
\begin{eqnarray}
    \int_{-\infty}^{0} |\Psi_{I}(x,t \rightarrow -\infty)| \dd{x} = \int_{-\infty}^{0} |\Psi_{R}(x,t\rightarrow \infty)|^2 + \int_{0}^{\infty} |\Psi_{T}(x,t \rightarrow \infty)|^2
,\end{eqnarray}
and inserting the definitions
\begin{eqnarray}
    1 = |A(k_0)|^2 + |B(k_0)|^2
.\end{eqnarray}
We thus call the reflection and transmission coefficients
\begin{eqnarray}
    R = |A(k_0)|^2 \quad T = |B(k_0)|^2
.\end{eqnarray}


\section{Scattering: An Alternative Treatment}

Generally, the wave-packet treatment is the most rigorous, but the algebra for more complicated potentials can become unwieldy, but we can avoid this mess to obtain the reflection and transmission coefficients.
In the previous section, we had
\begin{eqnarray}
    \psi_{k}(x) = \begin{cases}
        e^{ikx} + A(k) e^{-ikx} & x < 0 \\
        B(k)e^{ikx} & x > 0
    .\end{cases}
\end{eqnarray}
The probability current density of this is
\begin{eqnarray}
    j(x) = \frac{\hbar k}{m} \begin{cases}
        1 - |A(k)|^2 & x < 0 \\
        |B(k)|^2
    \end{cases} = \begin{cases}
        j_{I} + j_{R} & x < 0 \\
        j_{T}
    \end{cases}
.\end{eqnarray}
The reflection and transmission probabilities then are just ratios of the reflected and transmitted probability currents to the incident probability current:
\begin{eqnarray}
    R = \Big| \frac{j_{R}}{j_{I}} \Big| = |A(k)|^2 \quad T = \Big| \frac{j_{T}}{j_{I}} \Big| = |B(k)|^2
.\end{eqnarray}


\section{Scattering in a parity-invariant potential: phase shift method}

Let us consider the repulsive $\delta$-potential.
The even and odd scattering states are
\begin{eqnarray}
    \psi_{k}^{e}(x) = \begin{cases}
        A e^{ikx} + B e^{-ikx} & x < 0 \\
        B e^{ikx} + A e^{-ikx} & x > 0
    \end{cases}
    \quad
    \psi_{k}^{o}(x) = \begin{cases}
        C e^{ikx} + D e^{-ikx} & x < 0 \\
        -D e^{ikx} - C e^{-ikx} & x > 0
    .\end{cases}
\end{eqnarray}
Imposing BCs on the even solutions gives
\begin{eqnarray}
    \frac{B}{A} = \frac{2ik - v_0}{2ik + v_0}
.\end{eqnarray}

For the odd solutions, the BCs give $C = -D$.
We can write
\begin{eqnarray}
    \frac{2ik - v_0}{2ik + v_0} = e^{2i\delta(k)} ~{\rm with}~ \delta(k) = \frac{1}{2} \arctan(\frac{4 k v_0}{4 k^2 - v_0})
.\end{eqnarray}
Thus, the even and odd wave-functions can be expressed as
\begin{eqnarray}
    \psi_{k}^{e}(x) = Ae^{i\delta} \begin{cases}
        e^{i(kx - \delta)} + e^{-i(kx + \delta)} & x < 0 \\
        e^{i(kx + \delta)} + e^{-i(kx - \delta)} & x < 0
    \end{cases}
    = 2Ae^{i\delta} \cos[k|x| - \delta(k)]
\end{eqnarray}
and
\begin{eqnarray}
    \psi_{k}^{0}(x) = C \begin{cases}
        e^{ikx} - e^{-ikx} & x < 0 \\
        e^{ikx} - e^{-ikx} & x > 0
    \end{cases}
    = 2i C \sin(kx)
.\end{eqnarray}

The reflection and transmission coefficients are ({\color{red} can do the derivation later})
\begin{eqnarray}
    R(k) = \sin^2{\delta(k)} \quad T(k) = \cos^2{\delta(k)}
.\end{eqnarray}


\section{Reflection and Transmission in a Generic Potential: General Considerations}
{\color{red} Pressed for time so only the gist is given here}

The asymptotic regions are considered with both solutions
\begin{eqnarray}
    \psi_{k}(x) = \begin{cases}
        A e^{i k_{-} x} + B e^{-i k_{-} x} & x \rightarrow -\infty \\
        C e^{i k_{+} x} + D e^{i k_{+} x} & x \rightarrow \infty
    .\end{cases}
\end{eqnarray}
The probaility current is then
\begin{eqnarray}
    j(x) = \begin{cases}
        \frac{\hbar k_{-}}{m} (|A|^2 - |B|^2) & x < 0 \\
        \frac{\hbar k_{+}}{m} (|C^2| - |D|^2) & x > 0
    \end{cases}
.\end{eqnarray}
Defining $A' = \sqrt{k_{-}} A$, $B' = \sqrt{k_{-}} B$, $C' = \sqrt{k_{+}} C$, and $D' = \sqrt{k_{+}} D$, so that
\begin{eqnarray}
    \eref{eq:continuity}
    |A'|^2 + |B'|^2 = |C'|^2 + |D'|^2
.\end{eqnarray}
The matching conditions generally give
\begin{align}
    B' &= S_{11} A' + S_{12} C' \\
    D' &= S_{21} A' + S_{22} C'
,\end{align}
which can be written in matrix form as
\begin{eqnarray}
    \begin{pmatrix}
    B' \\ D'
    \end{pmatrix}
    =
    \underbrace{
    \begin{pmatrix}
        S_{11} & S_{12} \\ S_{21} & S_{22}
    \end{pmatrix}
}_{S}
    \begin{pmatrix}
        A' & D'
    \end{pmatrix} 
.\end{eqnarray}
This $S$ is the so-called scattering matrix.
Notice it is unitary because of \eref{continuity}.
From this, we can read off the reflection and transmission coefficients
\begin{align}
    R^{(1)} = |S_{11}|^2 \quad T^{(1)} = |S_{12}|^2 \\
    R^{(2)} = |S_{21}|^2 \quad T^{(2)} = |S_{22}|^2
.\end{align}

Note that $S$ is also symmetric (i.e. $S^{\rm T} = S$).





