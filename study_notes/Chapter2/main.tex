\chapter{Wave-Particle Duality and Wave Mechanics}

In the previous chapter, we have observed that light can be treated as both a particle in some instances and a wave in others.
It is natural then to extend this to matter particles such as electrons, which have in some instances as wave-like properties.
De Broglie introduced the wavelength of a particle as $\lambda = h/p$, and from this formula, we can reproduce the different observations which corroborate this wave-particle duality of matter.
For instance, a couple well known results are the two-slit experiment and Bragg reflection, both of which give interference patterns that can only be attributed to the wave nature of particles.

\section{Wave packets}

A particle's wave-function can be written
\begin{eqnarray}
    \label{eq:wave-packet}
    \Psi(\va*{r},t) = \int \frac{\dd[3]{\va*{k}'}}{(2 \pi)^{3/2}} g(\va*{k}') e^{i [ \va*{k}' \cdot \va*{r} - \omega(\va*{k}') t ]}
,\end{eqnarray}
where $g$ is some complex profile function whose magnitude $|g|$ is strongly localized around $\va*{k}$.
This is a superposition of plane waves corresponding to different momenta (we will see later that we cannot consider a particle with definite momentum $\va*{p} = \hbar \va*{k}$) with time dependence given by a phase with $\omega = E/\hbar = \hbar k'^2/2m$.

We can analyze this wave-packet using the method of stationary phase.
Defining $\phi(\va*{k}') = \va*{k}' \cdot \va*{r} - \omega t$, we expand around the peak of $|g|$, which gives
\begin{eqnarray}
    \phi(\va*{k}') \approx \phi(\va*{k}) + (\va*{k}' - \va*{k}) \cdot \grad_{\va*{k}'} \phi(\va*{k}')|_{\va*{k}' = \va*{k}} + \ldots
.\end{eqnarray}
Putting this into \eref{wave-packet}, we have
\begin{eqnarray}
    \Psi(\va*{r},t) \approx e^{i [\va*{k} \cdot \va*{r} - \omega(\va*{k}) t]} \int \frac{\dd[3]{\va*{k}'}}{(2 \pi)^{3/2}} g(\va*{k}') e^{i [ \va*{r} - \frac{\hbar \va*{k}}{m} t ] \cdot [\va*{k}' - \va*{k}]}
.\end{eqnarray}
Notice that the first factor is just a phase factor, and the line of constant phase $\phi = \phi_0$ is given by
\begin{eqnarray}
    x_{i} = \frac{\phi_0}{k_{i}} + \frac{\omega(\va*{k})}{k_{i}} t
.\end{eqnarray}
The coefficient in front of the second term is called the phase velocity.

Now, turning to the phase factor inside the integral, we have some argument that depends on the distance between $\va*{r}$ and $\va*{r}_{0} = (\hbar \va*{k} / m) t$.
We can identify $\va*{r}_{0}$ with the ``center'' of the wave-packet since for $\va*{r}$ far from $\va*{r}_0$, the exponential is very oscillatory and hence damps the amplitude of $\Psi$, but for $\va*{r} \approx \va*{r}_{0}$, there are no such cancellations and a peak at $\va*{r} = \va*{r}0$.
It is clear then that the wave-packet moves globally with this center at the group velocity $\va*{v}_{\rm group} = \hbar \va*{k}/m$.


\section{Mathematical Digression: $\delta$-Function Definition and Properties}

In the hand-wavey approach, we can regard delta-functions as the limit of a sequence of functions $\delta(x) = \lim_{n \rightarrow \infty} f_{n}(x)$ such that $\delta(x) = 0$ for all $x \ne 0$ and 
\begin{eqnarray}
    \int_{-\infty}^{\infty} \phi(x) \delta(x) \dd{x} = \phi(0)
.\end{eqnarray}
A few examples (certainly not exhaustive) of $(f_{n})$ include
\begin{itemize}
    \item $\displaystyle f_{n}(x) = \begin{cases} n & |x| < 1/2n \\ 0 & |x| > 1/2n\end{cases}$
    \item $\displaystyle f_{n}(x) = \frac{n}{2} e^{-n |x|}$
    \item $\displaystyle f_{n}(x) = \frac{n}{\pi} \frac{1}{(nx)^2 + 1}$
    \item $\displaystyle f_{n}(x) = \frac{n}{\sqrt{\pi}} e^{-(nx)^2}$
    \item $\displaystyle f_{n}(x) = \frac{1}{\pi} \frac{\sin(nx)}{x}$
    \item $\displaystyle f_{n}(x) = \frac{1}{n \pi} \frac{\sin^2(nx)}{x^2}$
\end{itemize}

There are a couple of properties of $\delta$-functions that should be enumerated\footnote{Note that by equality of two dirac functions $D_{1}$ and $D_{2}$, we really mean that $\int \dd{x} \phi(x) D_1(x) = \int \dd{x} \phi(x) D_2(x)$.}
\begin{enumerate}
    \item The $\delta$-function is even: $\delta(x) = \delta(-x)$, which can be easily shown by a substitution $x \rightarrow -x$.
    \item $\delta(cx) = \delta(x)/|c|$ for $c \in \mathbb{R}$ (substitute $u = cx$ and consider $c > 0$ and $c < 0$ separately).
    \item For a function $f(x)$ with simple roots $\{ x_{i} ~:~ f(x_{i}) = 0 \}$, $\delta(f(x)) = \sum_{i} \delta(x - x_{i})/|f'(x_{i})|$ (Break the integral into small pieces around each $x_{i}$ and Taylor expand $f(x)$ about $x_{i}$.).
\end{enumerate}

Generalizing to $n$ dimensions, we can define $\delta^{(n)}(\va*{x}) = 0$ for all $\va*{x} \ne 0$ and
\begin{eqnarray}
    \int_{\mathcal{D}} \phi(\va*{x}) \delta(\va*{x} - \va*{x}_{0}) \dd{x} = \begin{cases}
        \phi(\va*{x}_{0}) & \va*{x}_{0} \in \mathcal{D} \\
        0 & \va*{x}_{0} \not\in \mathcal{D}
    \end{cases}
.\end{eqnarray}
Focusing on three dimensions, we have
\begin{eqnarray}
    \delta^{(3)}(\va*{r}) = \delta(x) \delta(y) \delta(z)
.\end{eqnarray}
If we want to change to curvilinear coordinates $(u,v,w)$, then
\begin{eqnarray}
    \delta^{(3)}(\va*{r}) = |{\rm Jac}[(x,y,z) \rightarrow (u,v,w)]| ~ \delta(u) \delta(v) \delta(w)
.\end{eqnarray}
For example, changing from Cartesian to spherical, we have
\begin{eqnarray}
    \delta^{(3)}(\va*{r} - \va*{r}_{0}) = \frac{1}{r^2} \delta(r - r_0) \delta(\cos{\theta} - \cos{\theta_0}) \delta(\phi - \phi_0)
.\end{eqnarray}


\section{Mathematical Digression: Fourier Transform Definition and Properties}

Generically, an integral transform is of the form
\begin{eqnarray}
    \tilde{f}(y) = \int_{a}^{b} \dd{x} K(x,y) f(x)
.\end{eqnarray}
The function $K$ is called the kernel of the transform.

The Fourier transform is defined with $K(x,y) = e^{-iyx}/\sqrt{2\pi}$ such that
\begin{eqnarray}
    \tilde{f}(y) = \int_{-\infty}^{\infty} \frac{\dd{x}}{\sqrt{2\pi}} e^{-iyx} f(x)
.\end{eqnarray}
Note that only functions $f$ which are square integrable\footnote{$\int_{-\infty}^{\infty} |f(x)|^2 \dd{x} \in \mathbb{R}$} have a fourier transform.

If we know the transform of $f$, we can invert the Fourier transform as follows to obtain
\begin{eqnarray}
    f(x) = \int_{-\infty}^{\infty} \frac{\dd{x}}{\sqrt{2 \pi}} e^{iyx} \tilde{f}(y)
.\end{eqnarray}

There are a few properties that are useful to enumerate:
\begin{enumerate}
    \item Parseval's identity: $\displaystyle \int_{-\infty}^{\infty} |f(x)|^2 \dd{x} = \int_{-\infty}^{\infty} |\tilde{f}(y)|^2 \dd{y}$
    \item $\displaystyle \mathcal{F}\Big\{ \dv[n]{f}{x} \Big\} (y) = (iy)^{n} \tilde{f}(y)$.
    \item The Fourier transform of $h(x) = \int_{-\infty}^{\infty} \dd{x'} f(x - x')g(x')$ (called a convolution) is just the product of the transforms of $f$ and $g$ separately: $\tilde{h}(y) = \sqrt{2 \pi} \tilde{f}(y) \tilde{g}(y)$.
\end{enumerate}


